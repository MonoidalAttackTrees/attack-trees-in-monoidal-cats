

%% Suppose we are the computer security team of a small hospital, and we
%% need to assess the potential for an intruder to gain root access of the
%% all important medical records Unix server.  So we build an attack tree
%% to assess all of the potential ways one could gain root access to the
%% server.  Such an attack tree might look something like the following:
%% \[
%% \small
%%   \begin{array}{lll}
%%     \sqcup\,\text{''Obtain Root Privileges''}\,10\\
%%     \,\,\,\,\,\,\,(\sqcup\,\text{''Access System Console''}\,30\\
%%     \,\,\,\,\,\,\,\,\,\,\,\,\,\,\begin{array}{lll}
%%     (\sqcup\,\text{''Enter Computer Center''}\,30\\
%%     \,\,\,\,\,\,\,\,\,\,\,\,\,\,\begin{array}{lll}
%%     (\mathsf{leaf}\,\text{''Break In to Computer Center''}\,80)\\
%%     (\mathsf{leaf}\,\text{''Unattended Guest''}\,30))\\
%%     \end{array}\\
%%     (\mathsf{leaf}\,\text{''Corrupt Operator''}\,50))\\
%%     \end{array}\\
%%     \,\,\,\,\,\,\,(\sqcup\,\text{''Obtain Root Password''}\,10\\
%%     \,\,\,\,\,\,\,\,\,\,\,\,\,\,\begin{array}{lll}
%%     (\sqcup\,\text{''Naive Approach''}\,40\\
%%     \,\,\,\,\,\,\,\,\,\,\,\,\,\,\begin{array}{lll}
%%     (\rhd\,\text{''Guess Password''}\,50\\
%%     \,\,\,\,\,\,\,\,\,\,\,\,\,\begin{array}{lll}
%%     (\mathsf{leaf}\,\text{''Obtain Password File''}\,30)\\
%%     (\mathsf{leaf}\,\text{''Encounter Guessable Password''}\,20))\\
%%     \end{array}\\
%%     (\mathsf{leaf}\,\text{''Look Over Sys. Admin. Shoulder''}\,40))\\
%%     \end{array}\\
%%     (\sqcup\,\text{''Sophisticated Approach''}\,10\\
%%     \,\,\,\,\,\,\,\,\,\,\,\,\,\,\begin{array}{lll}
%%     (\mathsf{leaf}\,\text{''Trojan Horse Sys. Admin. Account''}\,10)\\
%%     (\mathsf{leaf}\,\text{''Corrupt Sys. Admin.''}\,80))\\
%%     \end{array}
%%     \end{array}
%%   \end{array}
%% \]
%% In the interest of saving space we present our attack trees in script
%% form.  Each node is labeled with a goal of the attacker and a cost for
%% executing such an attack. In the example above there are two types of
%% branching nodes, those labeled with $\sqcup$ which stands for a choice
%% between attacks, and those labeled with $\rhd$ which stands for
%% sequential composition of attacks.  The above attack tree shows that
%% the only reasonable attack for an intruder is to try and send a trojan
%% horse to the system administrator with a cost of $10$.

%% The example above implies that attack trees are a modeling tool,
%% originally proposed by Bruce Schneier \cite{Schneier:1999}, which are
%% used to assess the threat potential of a security critical system.
%% Attack trees have since been used to analyze the threat potential of
%% many types of security critical systems, for example, cybersecurity of
%% power grids \cite{Ten:2007}, wireless networks \cite{Reinhardt:2012},
%% and many others.  Attack trees consist of several goals, usually
%% specified in English prose, for example, ``compromise safe'' or
%% ``obtain administrative privileges'', where the root is the ultimate
%% goal of the attack and each node coming off of the root is a
%% refinement of the main goal into a subgoal.  Then each subgoal can be
%% further refined.  The leaves of an attack tree make up the set of base
%% attacks.

%% Extensions Of ATREES:
%%   - Attack nets
%%   - Sequential conjunction
%% There have been a number of extensions of attack trees to include new
%% operators on goals.  One such extension recasts attack trees into
%% attack nets which have all of the benefits of attack trees with the
%% additional benefit of being able to include the flaw hypothesis model
%% for penetration testing \cite{McDermott:2001}.  A second extension
%% adds sequential conjunction of attacks, that is, suppose $A_1$ and
%% $A_2$ are attacks, then $A_1;A_2$ is the attack obtained by performing
%% $A_1$, and then executing attack $A_2$ directly after $A_1$ completes
%% \cite{Jhawar:2015}.

%% Attack trees for real-world security scenarios can grow to be quite
%% complex.  The attack tree presented in \cite{Ten:2007} to access the
%% security of power grids has twenty-nine nodes with sixty counter
%% measures attached to the nodes throughout the tree.  The details of
%% the tree spans several pages of appendix.  The attack tree developed
%% for the border gateway protocol has over a hundred nodes
%% \cite{Convey:2003}, and the details of the tree spans ten pages.
%% Manipulating such large trees without a formal semantics can be
%% dangerous.

%% Semantics of ATREES:
%%   - Boolean logic
%%   - Attack nets (2000)
%%   - Multisets (2006)
%%   - Series-parallel graphs (extension of multisets) (2015)
%% One of the leading questions the field is seeking to answer is ``what
%% is an attack tree?''  That is, what is a mathematical foundation of
%% attack trees? There have been numerous attempts at answering this
%% question.  For example, attack trees have been based on propositional
%% logic and De Morgan Algebras
%% \cite{Kordy:2014,Kordy:2012,Pietre-Cambacedes:2010}, multisets
%% \cite{Mauw:2006}, Petri nets \cite{McDermott:2001}, tree automata
%% \cite{Camtepe:2007}, and series parallel graphs
%% \cite{Jhawar:2015}. \textbf{There is currently no known semantics of
%%   attack trees based in category theory}.

%% By far the most intuitive foundation of attack trees is propositional
%% logic or De Morgan algebras, however, neither of these properly
%% distinguish between attack trees with repeated subgoals.  If we
%% consider each subgoal as a \textbf{resource} then the attack tree
%% using a particular resource twice is different than an attack tree
%% where it is used only once.  The multiset semantics of attack trees
%% was developed precisely to provide a resource conscious foundation
%% \cite{Mauw:2006}. The same can be said for the Petri nets semantics
%% \cite{McDermott:2001}.  A second benefit of a semantics based in
%% multisets, Petri nets, and even tree automata is that operators on
%% goals in attack trees are associated with concurrency operators from
%% process algebra.  That is, the goals of an attack tree should be
%% thought of as being run concurrently.  Furthermore, these alternate
%% models lack \textbf{the intuitiveness and elegance of the
%%   propositional logic semantics}.  Lastly, existing work has focused
%% on specifically what an attack tree is, and has not sought to
%% understand what the theory of attack trees is.

%% By modeling attack trees in monoidal categories we obtain a sound
%% mathematical model, a resource conscious logic for reasoning about
%% attack trees, and the means of constructing a functional programming
%% language for defining attack trees (as types), and constructing
%% semantically valid transformations (as programs) of attack trees.

%% Linear logic was first proposed by Girard \cite{Girard:1987} and was
%% quickly realized to be a theory of resources.  In linear logic, every
%% hypothesis must be used exactly once.  Thus, formulas like $A \otimes
%% A$ and $A$ are not logically equivalent -- here $\otimes$ is linear
%% conjunction.  This resource perspective of linear logic has been very
%% fruitful in computer science and lead to linear logic being a logical
%% foundation of processes and concurrency where formulas may be
%% considered as processes.  Treating attack trees as concurrent
%% processes is not new; they have been modeled by event-based models of
%% concurrency like Petri nets and partially-ordered multisets (pomsets)
%% \cite{Jhawar:2015,Mauw:2006}.  In fact, pomsets is a model in which
%% events (the resources) can be executed exactly once, and thus, has a
%% relationship with linear logic \cite{Retore:1997}.  However,
%% connecting linear logic as a theory of attack trees is novel, and
%% strengthens this perspective.

%% In this short paper I introduce a newly funded research
%% project\footnote{This material is based upon work supported by the
%%   National Science Foundation CRII CISE Research Initiation grant,
%%   ``CRII:SHF: A New Foundation for Attack Trees Based on Monoidal
%%   Categories``, under Grant No. 1565557.}  investigating founding
%% attack trees in monoidal categories, and through the
%% Curry-Howard-Lambek correspondence deriving a new domain-specific
%% functional programming language called Lina for Linear Threat
%% Analysis.  Note that this paper introduces an ongoing research
%% project, and thus, we do not currently have the complete story, but we
%% feel that the community will find this project of interest, and the
%% project would benefit from the feedback of the community.  I begin by
%% defining an extension -- inspired by our semantics -- of the attack
%% trees given in \cite{Jhawar:2015} in Section~\ref{sec:attack_trees}.
%% Then I introduce a new semantics of attack trees in dialectica spaces,
%% which depends on a novel result in
%% Section~\ref{sec:concrete_semantics_of_attack_trees_in_dialectica_spaces}.
%% The final section,
%% Section~\ref{sec:lina:_a_domain_specific_pl_for_threat_analysis},
%% discusses Lina and some of the current problems the project seeks to
%% answer.
% section introduction (end)

\section{Attack Trees}
\label{sec:attack_trees}
In this section I introduce an extension to attack trees with
sequential composition first proposed by Jhawar et
al. \cite{Jhawar:2015}.  One of the projects ultimate goals is to
extend attack trees with even more operators driven by our choice of
semantics.  The syntax for attack trees is defined in the following
definition.  Note that in this paper we only consider the logical
aspects of attack trees and leave reasoning about the quantitative
aspects to future work.
\begin{definition}
  \label{def:atrees}
  The following defines the syntax of \textbf{Attack Trees} given a set
  of base attacks $b \in \mathsf{B}$:
  \[
  \begin{array}{lll}
    t ::= [[b]] \mid [[t1 + t2]] \mid [[t1 (+) t2]] \mid [[t1;t2]] \mid [[t1 (x) t2]]\\
  \end{array}
  \]
  I denote unsynchronized non-communicating parallel composition of
  attacks by $[[t1 + t2]]$, choice between attacks by $[[t1 (+) t2]]$,
  sequential composition of attacks by $[[t1;t2]]$, and a new
  operator called unsynchronized interacting parallel composition
  denoted $[[t1 (x) t2]]$.

  \ \\
  \noindent
  The following rules define the attack tree equivalence relation:
  \begin{center}
    \vspace{-14px}
    \footnotesize
    \begin{math}
      \begin{array}{cccccccc}
        \begin{array}{c}
        \inferrule* [right={\scriptsize assoc}] {
          \,
        }{[[(t1 op t2) op t3]] = [[t1 op (t2 op t3)]]}
        \\
        \inferrule* [right={\scriptsize sym}] {
          \,
        }{[[t1 opS t2]] = [[t2 opS t1]]}
      \end{array}
        & \quad &
        \begin{array}{lll}
          \inferrule* [right={\scriptsize $\textsc{dist}_1$}] {
          \,
        }{[[(t1 (+)  t2) + t3]] = [[(t1 + t3) (+) (t2 + t3)]]}
        \\
        \inferrule* [right={\scriptsize $\textsc{dist}_2$}] {
          \,
        }{[[(t1 (+)  t2) ; t3]] = [[(t1 ; t3) (+) (t2 ; t3)]]}
        \end{array}
      \end{array}
    \end{math}    
  \end{center}
  where $\mathsf{op} \in \{\odot, \otimes, \rhd, \sqcup\}$ and $\mathsf{op_S}
  \in \{\odot, \otimes, \sqcup\}$.  
\end{definition}
The set base attacks are the leafs of the attack tree.  Then each
operator composes attacks together depending on the intended meaning
of the attack. The syntax given in the previous definition differs
from the syntax used by Jhawar et al. \cite{Jhawar:2015}.  First, I
use infix binary operations, while they use prefix $n$-ary
operations. However, it does not sacrifice any expressivity, because
each operation is associative, and parallel composition, choice, and
interacting parallel composition are symmetric.  Thus, Jhawar et al.'s
definition of attack trees can be embedded into the ones defined
here.%% The hard
%% part of this embedding is realizing that the $n$-ary version of
%% sequential composition can be modeled by the binary version, but if we
%% have $\mathsf{SAND}(t_1,t_2,\ldots,t_n)$, then it is understood that
%% $t_1$ is executed, then $t_2$, and so on until $t_n$ is executed, but
%% this is exactly the same as $t_1;t_2;\cdots;t_n$.  

The second major difference is that the typical parallel composition
operator found in attack trees is modeled here by unsynchronized
non-communicating parallel composition.  This operator is a symmetric
tensor product, and not a disjunction.  This is contrary to the
literature, for example, the parallel operation of Jhawar et
al. defined on source sink graphs \cite{Jhawar:2015} can be proven to
be a coproduct -- see
Appendix~\ref{sec:source_sink_graphs_are_symmetric_monoidal} -- and
coproducts categorically model disjunctions.  Furthermore, parallel
composition is modeled by multiset union in the multiset semantics,
but we can model this as a coproduct.  However, in the semantics given
in the next section if we took parallel composition to be a coproduct,
then the required isomorphisms necessary to model attack trees would
not hold.

The third difference is that I denote the choice between executing
attack $[[t1]]$ or attack $[[t2]]$, but not both, by $[[t1 (+) t2]]$
instead of using a symbol that implies that it is a disjunction.  This
fits very nicely with the semantics of Jhawar et al., where they
collect the attacks that can be executed into a set.  The semantics I
give in the next section models choice directly.

The fourth difference is that I extend the syntax with a new operator
called unsynchronized communicating parallel composition.  The attack
$[[t1 (x) t2]]$ states that $[[t1]]$ interacts with the attack
$[[t2]]$ in the sense that processes interact.  Modeling interacting
attacks allows for the more refined modeling of security critical
systems, for example, it can be used to bring social engineering into
the analysis where someone communicates malicious information or
commands to a unsuspecting party.  As a second example, interacting
parallel composition could be used to model interacting bot nets.

Finally, the equivalence relation is essentially the equivalence given
in Jhaware et al. \cite{Jhawar:2015} -- Theorem 1.  Conducting an
analysis of the threat potential of a complex system might yield an
attack tree that must be rearranged to be understood, but this
rearrangement should not compromise the meaning of the tree.  The
attack tree equivalence relation defines the valid transformations
that one can do to a tree without compromising its meaning.
% section attack_trees (end)

\section{Semantics of Attack Trees in Dialectica Spaces}
\label{sec:concrete_semantics_of_attack_trees_in_dialectica_spaces}

I now introduce a new semantics of attack trees that connects their
study with a new perspective of attack trees that could highly impact
future research: intuitionistic linear logic, but it also strengthens
their connection to process calculi.  This section has been formalized
in the proof assistant Agda\footnote{The complete formalization can be
  found at
  \url{https://github.com/heades/dialectica-spaces/tree/PLAS16}
  which is part of a general library for working with dialectica
  spaces in Agda developed with Valeria de Paiva.}.  The semantics is
based on the notion of a dialectica space:

\begin{definition}
  \label{def:dialectica-space}
  A \textbf{dialectica space} is a triple $(A, Q, \delta)$ where $A$
  and $Q$ are sets and $\delta : A \times Q \to \mathsf{3}$ is a
  multi-relation where $\mathsf{3} = \{0,\perp,1\}$ and $\perp$
  represents undefined.
\end{definition}

Dialectica spaces can be seen as the intuitionistic cousin
\cite{dePaiva:2006b} of Chu spaces \cite{Pratt:1999}.  The latter have
be used extensively to study process algebra and as a model of
classical linear logic, while dialectica spaces and their morphisms
form a categorical model of intuitionistic linear logic called $\dial$
(originally due to \cite{dePaiva:1987}); I do not introduce
dialectica space morphisms here, but the curious reader can find the
definition in the formal development. I will use the intuitions often
used when explaining Chu spaces as processes to explain dialectica
spaces as processes, but it should be known that these intuitions are
due to Pratt and Gupta \cite{Gupta:1994}.

Intuitively, a dialectica space, $(A , Q, \delta)$, can be thought of
as a process where $A$ is the set of actions the process will execute,
$Q$ is the set of states the process can enter, and for $a \in A$ and
$q \in Q$, $\delta(a , q)$ indicates whether action $a$ can be
executed in state $q$.

The interpretation of attack trees into dialectica spaces is as
follows: {\vspace{-8pt}\ \\
\begin{changemargin}{2pt}{2pt}\noindent
\textbf{Parallel Composition.} Suppose $\cat{A} = (A , Q, \alpha)$ and
$\cat{B} = (B , R , \beta)$ are two dialectica spaces.  Then we can
construct -- due to de Paiva \cite{dePaiva:2014a} -- the dialectica
space $\cat{A} \odot \cat{B} = (A \times B, Q \times R, \alpha \odot
\beta)$ where $(\alpha \odot \beta)((a,b) , (q,r)) = \alpha(a , q)
\otimes_3 \beta(b , r)$ and $\otimes_3$ is the symmetric tensor
product definable on $\mathsf{3}$\footnote{See the formal development
  for the full definition:
  \url{https://github.com/heades/dialectica-spaces/blob/PLAS16/concrete-lineales.agda\#L328}}.
Thus, from a process perspective we can see that $\cat{A} \odot
\cat{B}$ executes actions of $\cat{A}$ and actions of $\cat{B}$ in
parallel.  Parallel composition is associative and symmetric.
\end{changemargin}

\begin{changemargin}{2pt}{2pt}\noindent
\textbf{Choice.} Suppose $\cat{A} = (A , Q, \alpha)$ and $\cat{B} = (B
, R , \beta)$ are two dialectica spaces.  Then we can construct the
dialectica space $\cat{A} \sqcup \cat{B} = (A + B, Q + R, \alpha
\sqcup \beta)$ where $(\alpha \sqcup \beta)(i , j) = \alpha(i , j)$ if
$i \in A$ and $j \in Q$, $(\alpha \sqcup \beta)(i , j) = \beta(i , j)$ if
$i \in B$ and $j \in R$, otherwise $(\alpha \sqcup \beta)(i , j) = 0$.
Thus, from a process perspective we can see that $\cat{A} \sqcup
\cat{B}$ executes either an action of $\cat{A}$ or an action of
$\cat{B}$, but not both.  Choice is symmetric and associative, but it
is not a coproduct, because it is not possible to define the
corresponding injections.  %% Brown et. al. show that Petri nets can
%% be modeled in dialectica spaces \cite{Brown:1991}, but they use the
%% coproduct as choice.  The operator given here is actually the
%% definition given for Chu spaces \cite{Gupta:1994}. If we were to
%% use the coproduct, then we would not be able to prove that choice
%% distributes over parallel composition nor over sequential
%% composition. As far as I am aware, this is the first time this has
%% been pointed out.
\end{changemargin}

\begin{changemargin}{2pt}{2pt}\noindent
\textbf{Sequential Composition.} Suppose $\cat{A} = (A , Q, \alpha)$
and $\cat{B} = (B , R , \beta)$ are two dialectica spaces.  Then we can
construct -- due to de Paiva \cite{dePaiva:2014a} -- the dialectica
space $\cat{A} \rhd \cat{B} = (A \times B, Q \times R, \alpha \rhd
\beta)$ where $(\alpha \rhd \beta)((a,b) , (q , r)) = \alpha(a , q)
\mathop{\mathsf{land}} \beta(i , r)$, and $\mathop{\mathsf{land}}$ is
lazy conjunction defined for $\mathsf{3}$\footnote{See the formal
  development for the full definition:
  \url{https://github.com/heades/dialectica-spaces/blob/PLAS16/concrete-lineales.agda\#L648}}.
This is a non-symmetric conjunctive operator, and thus, sequential
composition is \\non-symmetric.  This implies that from a process
perspective $\cat{A} \rhd \cat{B}$ will first execute the actions of
$\cat{A}$ and then execute actions of $\cat{B}$ in that order.
Sequential composition is associative.
\end{changemargin}

\begin{changemargin}{2pt}{2pt}\noindent
\textbf{Interacting Parallel Composition.} Suppose $\cat{A} = (A , Q,
\alpha)$ and $\cat{B} = (B , R , \beta)$ are two dialectica spaces.
Then we can construct the dialectica space $\cat{A} \otimes \cat{B} =
(A \times B, (B \rightarrow Q) \times (A \rightarrow R), \alpha
\otimes \beta)$ where $B \to Q$ and $A \to R$ denote function spaces,
and $(\alpha \otimes \beta)((a, b) , (f, g)) = \alpha(a , f(b)) \land
\beta(b,g(a))$.  From a process perspective the actions of $\cat{A}
\otimes \cat{B}$ are actions from $\cat{A}$ and actions of $\cat{B}$,
but the states are pairs of maps $f : B \to Q$ and $g : A \to R$ from
actions to states.  This is the point of interaction between the
processes.  This operator is symmetric and associative.
\end{changemargin}
} At this point it is straightforward to define an interpretation
$\interp{[[t]]}$ of attack trees into $\dial$.  Soundness with respect
to this model would correspond to the following theorem.

\begin{theorem}[Soundness]
  \label{thm:soundness}
  If $[[t1]] = [[t2]]$, then $\interp{[[t1]]} \cong \interp{[[t2]]}$ in $\dial$.
\end{theorem}

Those familiar with Chu spaces and their application to process
algebra may be wondering how treating dialectica spaces as processes
differs.  The starkest difference is that in this model process
simulation is modeled by morphisms of the model, but this is not
possible in Chu spaces.  In fact, to obtain the expected properties of
processes a separate notion of bi-simulation had to be developed for
Chu spaces \cite{Gupta:1994}.  However, I took great care to insure
that the morphisms of our semantics capture the desired properties of
process simulation, and hence, attack trees.

The ability to treat morphisms as process simulation was not easy to
achieve.  The definition of choice in the semantics presented here
actually is the definition given for Chu spaces \cite{Gupta:1994}, but
Brown et al. use the coproduct defined for dialectica spaces to model
choice in Petri nets.  However, taking the coproduct for choice here
does not lead to the isomorphisms $(\cat{A} \sqcup \cat{B}) \rhd
\cat{C} \cong (\cat{A} \rhd \cat{C}) \sqcup (\cat{B} \rhd \cat{C})$
and $(\cat{A} \sqcup \cat{B}) \odot \cat{C} \cong (\cat{A} \odot
\cat{C}) \sqcup (\cat{B} \odot \cat{C})$, thus, we will not be able to
soundly model attack trees.  I have found that if choice is modeled
using the definition from Chu spaces \cite{Gupta:1994} then we obtain
these isomorphisms which is a novel result\footnote{For the proofs see
  the formal development:
  \url{https://github.com/heades/dialectica-spaces/blob/PLAS16/concurrency.agda\#L70}
  and
  \url{https://github.com/heades/dialectica-spaces/blob/PLAS16/concurrency.agda\#L150}}.

This semantics can be seen as a generalization of some existing
models.  Multisets, pomsets, and Petri nets can all be modeled by
dialectica spaces \cite{Brown:1991,Gupta:1994}.  However, there is a
direct connection between dialectica spaces and linear logic which
may lead to a logical theory of attack trees.
% section concrete_semantics_of_attack_trees_in_dialectica_spaces (end)

\section{Lina: A Domain Specific PL for Threat Analysis}
\label{sec:lina:_a_domain_specific_pl_for_threat_analysis}

The second major part of this project is the development of a
staticly-typed domain-specific linear functional programming language
for specifying and reasoning about attack trees called Lina for Linear
Threat Analysis.  Lina will consist of a core language and a surface
language.

The project views attack trees as consisting of two layers: a logic
layer and a quantitative layer.  The former is described by the
definition of attack trees in the previous section, but the latter is
the layer added on top of the logical layer, which is used when
conducting threat analysis, for example, computing the set of attacks
with minimal cost.  Thus, there are two types of proofs about attack
trees.  Proofs about the logical layer will be checked using a linear
type system, but proofs about the quantitative layer will be mostly
numerical. In this section I largely concentrate on the logical layer
which is the current focus of the project.

Lina's core will consist of a language for defining attack trees, and
this language will consist of the two layers, but one benefit of the
layered view of attack trees is that the logical layer can be
projected out, and hence, reasoning about the logical layer using a
linear type system may completely ignore the quantitative layer.  The
following two sections describe both of these concepts.

\subsection{Lina's Core: Defining Attack Trees}
\label{subsec:linas_core_defining_attack_trees}

\input{def-lang-output}

% subsection linas_core_defining_attack_trees (end)

\subsection{Lina's Core: A Linear Type System}
\label{subsec:linas_core_linear_type_system}

The current main focus of the project is the design and analysis of
Lina's core type system.  Types in Lina will correspond to attack
trees while programs correspond to semantically valid transformations
of attack trees, thus, a question we must answer then is \textbf{how
  do we sufficiently represent the model of attack trees in $\dial$ as
  a linear logic?}  The problem is the fact that Lina will require
both commutative (parallel composition and choice) and
non-communicative monoidal operators (sequencing).

Supporting both commutative and non-communicative operators within the
same linear logic has been a long standing question.  A starting point
might be with Reedy's LLMS which has already been shown to have a
categorical model in $\dial$ by de Paiva \cite{dePaiva:2014a}.  In
fact, the definition of non-interacting parallel composition given
here is due to her model.  However, we have taken a new path which we
also approach categorically, and then syntactically.

%% We have obtained a great insight from our model in dialectica
%% categories, but it is not obvious how to abstract it into a model we
%% can easily extract a type system from.  So instead of working directly
%% with the dialectica model we have found a more abstract model with the
%% ability to have both a commutative tensor product and and
%% non-communicative tensor product within the same model, but is more
%% straightforwardly transformed into a type system.

%% Girard's genius behind linear logic was that he isolated the
%% structural rules -- weakening and contraction -- by treating them as
%% an effect and putting them inside a comonad called the of-course
%% exponential denoted $!A$.  In fact, $!A \otimes !A$ is logically
%% equivalent to $!A$, and thus, by staying in the comonad we become
%% non-linear.  However, in Girard's linear logic exchange was not
%% isolated like weakening and contraction.  Thus, the tensor product was
%% symmetric.

To accommodate both a commutative and non-commutative tensor product
we isolate exchange in the same way that Girard isolated weakening and
contraction using a comonad.  In this section I give a brief overview
of how this is done.  I begin with the notion of a Lambek category.
\begin{definition}
  \label{def:Lambek-category}
  A \textbf{Lambek category} is a monoidal category \\ $(\cat{C}, I,
  \rhd, \alpha, \lambda, \rho)$ where $\rhd : \cat{C} \times \cat{C}
  \mto \cat{C}$ is the non-commutative tensor product, and
  $\alpha_{A,B,C} : (A \rhd B) \rhd C \mto A \rhd (B \rhd C)$,
  $\lambda_A : I \rhd A \mto A$, and $\rho_A : A \rhd I \mto A$ are
  all natural transformations subject to several coherence
  diagrams\footnote{The coherence diagrams are equivalent to the ones
    in the definition of a symmetric monoidal category modulo
    symmetry; see the appendix for the complete definition of a
    symmetric monoidal category.}.
\end{definition}
\noindent
We call the previous category a Lambek category to pay homage to
Joachim Lambek and his work on the Lambeck calculus \cite{Lambek1958}
which is a non-commutative substructural logic that predates linear
logic.  The traditional definition of a Lambek category also requires
that the monoidal category be biclosed, but we do not concern
ourselves here with closed categories.

In our model exchange will be considered as an effect, and so we
isolate it inside a comonad.  This will allow for the definition of a
commutative tensor product.
\begin{definition}
  \label{def:exchange-comonad}
  A \textbf{Lambek category with exchange} is a Lambek category\\
  $(\cat{C}, I, \rhd, \alpha, \lambda, \rho)$ equipped with a monoidal
  comonad $(e, \varepsilon, \delta)$ where $e : \cat{C} \mto \cat{C}$
  is a monoidal endofunctor\footnote{For the full definition of a
    monoidal functor see Definition~\ref{def:MCFUN} in
    Appendix~\ref{sec:symmetric_monoidal_categories}.}, and
  $\varepsilon_A : eA \mto A$ and $\delta_A : eA \mto e^2A$ are
  monoidal natural transformations.  In addition, there is a monoidal natural
  transformation $\mathsf{ex}_{A,B} : eA \rhd eB \mto eB \rhd
  eA$. Each of these morphisms are subject to the several coherence
  diagrams which I omit due to space.  Most importantly, the
  following diagram must commute:
  \[
  \bfig
  \qtriangle|amm|/->`=`->/<1000,500>[
    eA \otimes eB`
    eB \otimes eA`
    eA \otimes eB;
    \mathsf{ex}_{A,B}``
    \mathsf{ex}_{B,A}]
  \efig
  \]  
  \noindent
  The previous diagram can is invertibility for $\mathsf{ex}_{A,B}$.
  There are a few other coherence diagrams that must hold to insure
  that the Eilenberg-Moore category is symmetric, but we omit these for
  space.
\end{definition}
The previous definition is largely based on how weakening and
contraction are modeled by the of-course exponential.  To be sure that
exchange is defined correctly it should be the case that the
Eilenberg-Moore category becomes symmetric monoidal.  Thus, when working inside the comonad we are working in a symmetric monoidal category. 
\begin{lemma}
  \label{lemma:cokleisli_of_the_exchange_monad_is_symmetric}
  Suppose $\cat{C}$ is a Lambek category with exchange.  Then the
  Eilenberg-Moore category of the exchange comonad is symmetric
  monoidal.
\end{lemma}
It is now more straightforward to construct a term assignment from
this model.  The following define a preliminary definition of a
natural deduction term assignment:
\begin{mathpar}
  \small
  \ATreesdrulevar{}    \and
  \ATreesdruleunitI{}  \and
  \ATreesdruleunitE{}  \and
  \ATreesdruletenI{}   \and
  \ATreesdruletenE{}   \and
  \ATreesdruleex{}   \and
  \ATreesdruledereliction{}  \and
  \ATreesdrulepromotion{}  
\end{mathpar}
The previous rules are based on the natural deduction formalization of
intuitionistic linear logic due to \cite{Benton:1992}.  In future work
we hope to shift to a term assignment based on Barber's
\cite{Barber:1997} dual context formalization.

We can interpret both sequential and interacting parallel composition
as types.  That is, we can interpret $[[t1 ; t2]]$ as $[[T1 ; T2]]$,
and $[[t1 (x) t2]]$ as $[[e T1 ; e T2]]$.  However, accommodating the
other attack tree branching connectives in addition to these in linear
logic is left for future work.
% subsection linas_core_linear_type_system (end)

\subsection{Lina's Surface Language}
\label{subsec:linas_surface_language}

So far I have only mentioned Lina's core language, because this has
been the focus of the first year of the project.  The surface language
will have two main objectives: simplicity and automation.  It will
also be paired with an IDE specifically geared toward threat analysis.
I can only give a brief account of the projects ideas for the surface
language, but experiments will need to be carried out before its
final design will be settled.

\textbf{Simplicity.}  At the IDE level the project plans to capitalize
on the graphical languages category theory provides
\cite{Selinger:2009} to allow for the registration of semantically
valid transformations of attack trees.  Thus, preventing a security
specialist from having to learn the Lina programming language.  The
surface language will rely on local type inference to make programming
easier, but this may require a new local type inference algorithm for
intuitionistic linear logic.

\textbf{Automation.}  Reasoning at the quantitative level could be
highly benefited from automated theorem proving which we plan to
interface with Lina.  Furthermore, there has been recent work
\cite{essay69399,Sheyner:2002:AGA:829514.830526,6957121,essay69402} on
exploring automatically constructing attack trees from a specification
which the project plans to investigate interfacing with Lina, because
once you have the tree one may want to analyze it themselves.
Furthermore, these automated approaches use sophisticated graph
rewriting algorithms, it might be possible to reframe this work in
terms of linear logic and potentially find new ways to automatically
construct attack trees using automated theorem proving.

%% Another goal of this project is to make using Lina as close as
%% possible to functional programming as usual to prevent a large
%% overhead of using the language as well as the tool.  As a programming
%% language simplicity is of the utmost importance, and I think with the
%% semantics given here Lina will not require very advanced syntactic
%% features.  This cannot be said for some of the existing work that is
%% similar to Lina.  For example, Vigo et al. \cite{6957121} proposed the
%% Quality Tree Generator which requires the user to program in a process
%% calculus which is a non-trivial overhead. At the tool level the goal
%% is to have a completely graphical environment for creating attack
%% trees and reasoning about them by capitalizing on existing graphical
%% reasoning tools from category theory.  Thus, at the tool level the
%% user will not have to write any programs at all unless they want to
%% extend the environment.

%% I do not consider this project, particularly Lina, to be at odds with
%% existing work on using automated theorem proving to synthesize and
%% analyze attack trees; see for example
%% \cite{essay69399,Sheyner:2002:AGA:829514.830526,6957121,essay69402}.
%% In fact, this project can benefit from automated generation of attack
%% trees.  Lina's primary goal is to make reasoning about attack trees
%% safer by having a tight correspondence with the semantics of attack
%% trees, and thus, will allow and help with the creation of attack
%% trees.  Lina will offer a manual way for one to create an attack tree,
%% but by leveraging this existing work could allow for their automatic
%% generation, but then could be used to restructure the tree and conduct
%% further analysis in a semantically valid fashion.  In addition, Lina
%% can be seen as an interactive theorem prover for attack trees, and so
%% could be used as a proof checker \cite{Stump:2012} for proof producing
%% SMT backed automated generation of attack trees, thus, potentially
%% allowing for some of the analysis of attack trees in Lina to be
%% automated.
% subsection linas_surface_language (end)
% section lina:_a_domain_specific_pl_for_threat_analysis (end)

\section{Conclusion and Future Work}
\label{sec:conclusion}

The project described here seeks to develop a new semantics to attack
trees that can be leveraged to design a new domain-specific
programming language for reasoning and analyzing attack trees to assess
the threat of security critical systems.

I showed that attack trees (Section~\ref{sec:attack_trees}) can be
given a semantics in dialectica spaces
(Section~\ref{sec:concrete_semantics_of_attack_trees_in_dialectica_spaces}),
thus, relating the study of attack trees to the study of
intuitionistic linear linear logic.  Then I showed that this model can
be abstracted into an alternate categorical model in Lambek categories
that is easier to translate into a type system by exploiting the
Curry-Howard-Lambek correspondence
(Section~\ref{subsec:linas_core_linear_type_system}).  Finally, I
introduced the preliminary design for a domain-specific functional
programming language called Lina for Linear Threat Analysis
(Section~\ref{sec:lina:_a_domain_specific_pl_for_threat_analysis}) to
be used to develop a new tool to conduct threat analysis using attack
trees.  This tool will include the ability to design and formally
reason about attack trees using interactive and automated theorem
proving as well as graphical reasoning tools.

There is still a lot of work to be done.  A forth coming paper will
fully explore the alternate categorical model in Lambek categories
both categorically and syntactically.  In addition, Lina's core design
needs to be further developed and case studies need to be conducted to
assess its effectiveness, and to understand where automation will be
most useful.  Finally, Lina's surface language and accompanying IDE
still needs to be designed and developed.
% section conclusion_and_future_work (end)



%% \begin{thebibliography}{10}
%% \bibitem[Barr(1991)Barr]{MSC:4439568}
%% Michael Barr.
%% \newblock *-autonomous categories and linear logic.
%% \newblock {\em Mathematical Structures in Computer Science}, 1:159--178, 7
%%   1991.

%% \bibitem[Benton et al.(1992)Benton, Bierman, de Paiva, Hyland]{Benton:1992}
%% Nick Benton, Gavin Bierman, Valeria de~Paiva, and Martin Hyland.
%% \newblock Term assignment for intuitionistic linear logic (preliminary report).
%% \newblock Technical report, University of Cambridge, August 1992.  
  
%% \bibitem[Brown et~al.(1991)Brown, Gurr, de Paiva]{Brown:1991}
%% Carolyn Brown, Doug Gurr, and Valeria de Paiva.
%% \newblock A linear specification language for petri nets.
%% \newblock {\em DAIMI Report Series}, 20(363), 1991.

%% \bibitem[Camtepe and Yener(2007)Camtepe, Yener]{Camtepe:2007}
%% S.A. Camtepe and B.~Yener.
%% \newblock Modeling and detection of complex attacks.
%% \newblock In {\em Security and Privacy in Communications Networks and the
%%   Workshops, 2007. SecureComm 2007. Third International Conference on}, pages
%%   234--243, Sept 2007.

%% \bibitem[Convery et~al.(2003)Convery, Cook, Franz]{Convey:2003}
%% S.~Convery, D.~Cook, and M.~Franz.
%% \newblock An attack tree for the border gateway protocol.
%% \newblock 2003.
%% \newblock \url{https://tools.ietf.org/html/draft-ietf-rpsec-bgpattack-00}.

%% \bibitem[de Paiva(1987)de Paiva]{dePaiva:1987}
%% Valeria de~Paiva.
%% \newblock Dialectica categories.
%% \newblock In J.~Gray and A.~Scedrov, editors, {\em Categories in Computer
%%   Science and Logic}, volume~92, pages 47--62. Amerian Mathemaitcal Society,
%%   1989.

%% \bibitem[de Paiva(2006)de Paiva]{dePaiva:2006b}
%% Valeria de~Paiva.
%% \newblock Dialectica and chu constructions: Cousins?
%% \newblock {\em Theory and Applications of Categories}, 17(7):127--152, 2006.

%% \bibitem[de Paiva(2014)de Paiva]{dePaiva:2014a}
%% Valeria de~Paiva.
%% \newblock Linear logic model of state revisited.
%% \newblock {\em Logic Journal of IGPL}, 22(5):791--804, 2014.

%% \bibitem[Fiore and Campos(2013)Fiore, Campos]{Fiore:2013}
%% Marcelo Fiore and Marco Devesas~Campos.
%% \newblock {\em Computation, Logic, Games, and Quantum Foundations. The Many
%%   Facets of Samson Abramsky: Essays Dedicated to Samson Abramsky on the
%%   Occasion of His 60th Birthday}, chapter The Algebra of Directed Acyclic
%%   Graphs, pages 37--51.
%% \newblock Springer Berlin Heidelberg, Berlin, Heidelberg, 2013.

%% \bibitem[Albasini et al.(2010)Albasini, Sabadini, Walters]{FrancescoAlbasini2010}
%% Luisa Francesco~Albasini, Nicoletta Sabadini, and Robert F.~C. Walters.
%% \newblock The compositional construction of markov processes.
%% \newblock {\em Applied Categorical Structures}, 19(1):425--437, 2010.

%% \bibitem[Girard(1987)Girard]{Girard:1987}
%% Jean-Yves Girard.
%% \newblock Linear logic.
%% \newblock {\em Theoretical Computer Science}, 50(1):1 -- 101, 1987.

%% \bibitem[Gupta(1994)Gupta]{Gupta:1994}
%% Vineet Gupta.
%% \newblock {\em Chu Spaces: a Model of Concurrency}.
%% \newblock PhD thesis, Stanford University, 1994.

%% \bibitem[Huistra(2016)Huistra]{essay69399}
%% D.J. {Huistra}.
%% \newblock Automated generation of attack trees by unfolding graph
%%   transformation systems, Online: \url{http://essay.utwente.nl/69399/}, March 2016.

%% \bibitem[Jhawar et al.(2015)Jhawar, Kordy, Mauw, Radomirovi{\"A}‡, Trujillo-Rasua]{Jhawar:2015}
%% Ravi Jhawar, Barbara Kordy, Sjouke Mauw, Sa{\AA}{!'}a Radomirovi{\"A}‡, and
%%   Rolando Trujillo-Rasua.
%% \newblock Attack trees with sequential conjunction.
%% \newblock In Hannes Federrath and Dieter Gollmann, editors, {\em ICT Systems
%%   Security and Privacy Protection}, volume 455 of {\em IFIP Advances in
%%   Information and Communication Technology}, pages 339--353. Springer
%%   International Publishing, 2015.

%% \bibitem[Kordy et al.(2014)Kordy,Pi{\'e}tre-Cambac{\'e}d{\'e}s,Schweitzer]{Kordy:2014a}
%% Barbara Kordy, Ludovic Pi{\'e}tre-Cambac{\'e}d{\'e}s, and Patrick Schweitzer.
%% \newblock Dag-based attack and defense modeling: Don't miss the forest for the
%%   attack trees.
%% \newblock {\em Computer Science Review}, 13{\^a}€“14:1 -- 38, 2014.

%% \bibitem[Kordy et al(2012)Kordy, Pouly, Schweitzer]{Kordy:2012}
%% Barbara Kordy, Marc Pouly, and Patrick Schweitzer.
%% \newblock Computational aspects of attack--defense trees.
%% \newblock In Pascal Bouvry, Mieczys{\l}awA. K{\l}opotek, Franck Lepr{\'e}vost,
%%   Ma{\l}gorzata Marciniak, Agnieszka Mykowiecka, and Henryk Rybi{\'n}ski,
%%   editors, {\em Security and Intelligent Information Systems}, volume 7053 of
%%   {\em Lecture Notes in Computer Science}, pages 103--116. Springer Berlin
%%   Heidelberg, 2012.

%% \bibitem[Kordy et al.(2014)Kordy, Pouly, Schweitzer]{Kordy:2014}
%% Barbara Kordy, Marc Pouly, and Patrick Schweitzer.
%% \newblock A probabilistic framework for security scenarios with dependent
%%   actions.
%% \newblock In Elvira Albert and Emil Sekerinski, editors, {\em Integrated Formal
%%   Methods}, volume 8739 of {\em Lecture Notes in Computer Science}, pages
%%   256--271. Springer International Publishing, 2014.

%% \bibitem[Lambek(1958)Lambek]{Lambek1958}
%% Joachim Lambek.
%% \newblock The mathematics of sentence structure.
%% \newblock {\em American Mathematical Monthly}, pages 154--170, 1958.  
  
%% \bibitem[Mauw and Oostdijk(2006)Mauw and Oostdijk]{Mauw:2006}
%% Sjouke Mauw and Martijn Oostdijk.
%% \newblock Foundations of attack trees.
%% \newblock In DongHo Won and Seungjoo Kim, editors, {\em Information Security
%%   and Cryptology - ICISC 2005}, volume 3935 of {\em Lecture Notes in Computer
%%   Science}, pages 186--198. Springer Berlin Heidelberg, 2006.

%% \bibitem[McDermott(2001)McDermott]{McDermott:2001}
%% J.~P. McDermott.
%% \newblock Attack net penetration testing.
%% \newblock In {\em Proceedings of the 2000 Workshop on New Security Paradigms},
%%   NSPW '00, pages 15--21, New York, NY, USA, 2000. ACM.

%% \bibitem[Pierce and Turner(2000)Pierce, Turner]{Pierce:2000}
%% Benjamin~C. Pierce and David~N. Turner.
%% \newblock Local type inference.
%% \newblock {\em ACM Trans. Program. Lang. Syst.}, 22(1):1--44, January 2000.

%% \bibitem[Pi{\`e}tre-Cambac{\'e}d{\`e}s and
%%   Bouissou(2010)Pi{\`e}tre-Cambac{\'e}d{\`e}s, Bouissou]{Pietre-Cambacedes:2010}
%% L.~Pi{\`e}tre-Cambac{\'e}d{\`e}s and M.~Bouissou.
%% \newblock Beyond attack trees: Dynamic security modeling with boolean logic
%%   driven markov processes (bdmp).
%% \newblock In {\em Dependable Computing Conference (EDCC), 2010 European}, pages
%%   199--208, April 2010.

%% \bibitem[Pratt(1999)Pratt]{Pratt:1999}
%% Vaughan Pratt.
%% \newblock Chu spaces.
%% \newblock Notes for the School on Category Theory and Applications University
%%   of Cimbra, July 1999.

%% \bibitem[Reinhardt et al.(2012)Reinhardt, Seither, Konig, Steinmetz, Hollick]{Reinhardt:2012}
%% A.~Reinhardt, D.~Seither, A.~Konig, R.~Steinmetz, and M.~Hollick.
%% \newblock Protecting ieee 802.11s wireless mesh networks against insider
%%   attacks.
%% \newblock In {\em Local Computer Networks (LCN), 2012 IEEE 37th Conference on},
%%   pages 224--227, Oct 2012.

%% \bibitem[Retor{\'e}(1997)Retor{\'e}]{Retore:1997}
%% Christian Retor{\'e}.
%% \newblock {\em Typed Lambda Calculi and Applications: Third International
%%   Conference on Typed Lambda Calculi and Applications TLCA '97 Nancy, France,
%%   April 2--4, 1997 Proceedings}, chapter Pomset logic: A non-commutative
%%   extension of classical linear logic, pages 300--318.
%% \newblock Springer Berlin Heidelberg, Berlin, Heidelberg, 1997.

%% \bibitem[Schneier(1999)Schneier]{Schneier:1999}
%% Bruce Schneier.
%% \newblock Attack trees: Modeling security threats.
%% \newblock {\em Dr. Dobb's journal}, December 1999.

%% \bibitem[Selinger(2009)Selinger]{Selinger:2009}
%% Peter {Selinger}.
%% \newblock {A survey of graphical languages for monoidal categories}.
%% \newblock {\em ArXiv e-prints}, August 2009.

%% \bibitem[Sheyner et al.(2002)Sheyner, Haines, Jha, Lippmann, Wing]{Sheyner:2002:AGA:829514.830526}
%% Oleg Sheyner, Joshua Haines, Somesh Jha, Richard Lippmann, and Jeannette~M.
%%   Wing.
%% \newblock Automated generation and analysis of attack graphs.
%% \newblock In {\em Proceedings of the 2002 IEEE Symposium on Security and
%%   Privacy}, SP '02, pages 273--, Washington, DC, USA, 2002. IEEE Computer
%%   Society.

%% \bibitem[Ten et al.(2007)Ten, Liu, Govindarasu]{Ten:2007}
%% Chee-Wooi Ten, Chen-Ching Liu, and Manimaran Govindarasu.
%% \newblock Vulnerability assessment of cybersecurity for scada systems using
%%   attack trees.
%% \newblock In {\em Power Engineering Society General Meeting, 2007. IEEE}, pages
%%   1--8, June 2007.

%% \bibitem[Tzouvaras(1998)Tzouvaras]{Tzouvaras:1998}
%% A~Tzouvaras.
%% \newblock The linear logic of multisets.
%% \newblock {\em Logic Journal of IGPL}, 6(6):901--916, 1998.

%% \bibitem[Vigo et al.(2014)Vigo, Nielson, Nielson]{6957121}
%% R.~Vigo, F.~Nielson, and H.~R. Nielson.
%% \newblock Automated generation of attack trees.
%% \newblock In {\em Computer Security Foundations Symposium (CSF), 2014 IEEE
%%   27th}, pages 337--350, July 2014.

%% \bibitem[Wolters(2016)Wolters]{essay69402}
%% N.H. {Wolters}.
%% \newblock Analysis of attack trees with timed automata (transforming formalisms
%%   through metamodeling), Online: \url{http://essay.utwente.nl/69402/},
%%   March 2016.  
  
%% \end{thebibliography}
