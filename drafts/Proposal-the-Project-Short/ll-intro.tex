Before defining attack trees and the logical theory behind them we
first give a brief introduction to linear logic to aid the readers
understanding.  Linear logic is due to the work of
Girard~\cite{Girard:1987}, and has since seen many applications in
computer science, natural languages, mathematics, and philosophy.

We call a logical formula \textbf{linear} if and only if all of its
hypotheses are used exactly once.  In this paper, we will only
consider the following linear logic formulas:
\[
A ::= b \mid [[A (x) B]] \mid [[A -o B]]\\
\]
Atomic formulas are denoted by $[[b]]$, linear conjunction is denoted
by $[[A (x) B]]$, and linear implication is denoted by $[[A -o B]]$.


Linear logic is then characterized by the logic of linear formulas.
It can come in various flavors: intuitionistic, classical, modal, etc.
Furthermore, it can come in several different formulations: natural
deduction, sequent calculus, axiomatic, etc.  Every logic in this
paper will be intuitionistic and in natural deduction form.


