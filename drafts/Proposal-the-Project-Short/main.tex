\documentclass{llncs}

\usepackage{amssymb,amsmath}
\usepackage{cmll}
\usepackage{stmaryrd}
\usepackage{todonotes}
\usepackage{mathpartir}
\usepackage{hyperref}
\usepackage[barr]{xy}
\usepackage{mdframed}

%% This renames Barr's \to to \mto.  This allows us to use \to for imp
%% and \mto for a inline morphism.
\let\mto\to
\let\to\relax
\newcommand{\to}{\rightarrow}

% Commands that are useful for writing about type theory and programming language design.
%% \newcommand{\case}[4]{\text{case}\ #1\ \text{of}\ #2\text{.}#3\text{,}#2\text{.}#4}
\newcommand{\interp}[1]{\llbracket #1 \rrbracket}
\newcommand{\normto}[0]{\rightsquigarrow^{!}}
\newcommand{\join}[0]{\downarrow}
\newcommand{\redto}[0]{\rightsquigarrow}
\newcommand{\nat}[0]{\mathbb{N}}
\newcommand{\fun}[2]{\lambda #1.#2}
\newcommand{\CRI}[0]{\text{CR-Norm}}
\newcommand{\CRII}[0]{\text{CR-Pres}}
\newcommand{\CRIII}[0]{\text{CR-Prog}}
\newcommand{\subexp}[0]{\sqsubseteq}
%% Must include \usepackage{mathrsfs} for this to work.
\newcommand{\powerset}[0]{\mathscr{P}}

\date{}

% Cat commands.
\newcommand{\cat}[1]{\mathcal{#1}}
\newcommand{\catop}[1]{\cat{#1}^{\mathsf{op}}}
\newcommand{\Hom}[3]{\mathsf{Hom}_{\cat{#1}}(#2,#3)}
\newcommand{\limp}[0]{\multimap}
\newcommand{\dial}[0]{\mathsf{Dial_3}(\mathsf{Sets})}
\newcommand{\dialSets}[1]{\mathsf{Dial_{#1}}(\mathsf{Sets})}
\newcommand{\dcSets}[1]{\mathsf{DC_{#1}}(\mathsf{Sets})}
\newcommand{\sets}[0]{\mathsf{Sets}}
\newcommand{\obj}[1]{\mathsf{Obj}(#1)}
\newcommand{\mor}[1]{\mathsf{Mor(#1)}}
\newcommand{\id}[0]{\mathsf{id}}
\newcommand{\lett}[0]{\mathsf{let}\,}
\newcommand{\inn}[0]{\,\mathsf{in}\,}
\newcommand{\cur}[1]{\mathsf{cur}(#1)}
\newcommand{\curi}[1]{\mathsf{cur}^{-1}(#1)}

\newenvironment{changemargin}[2]{%
  \begin{list}{}{%
    \setlength{\topsep}{0pt}%
    \setlength{\leftmargin}{#1}%
    \setlength{\rightmargin}{#2}%
    \setlength{\listparindent}{\parindent}%
    \setlength{\itemindent}{\parindent}%
    \setlength{\parsep}{\parskip}%
  }%
  \item[]}{\end{list}}

%% % Theorems
%% \newtheorem{theorem}{Theorem}
%% \newtheorem{lemma}[theorem]{Lemma}
%% \newtheorem{fact}[theorem]{Fact}
%% \newtheorem{corollary}[theorem]{Corollary}
%% \newtheorem{definition}[theorem]{Definition}
%% \newtheorem{remark}[theorem]{Remark}
%% \newtheorem{proposition}[theorem]{Proposition}
%% \newtheorem{notn}[theorem]{Notation}
%% \newtheorem{observation}[theorem]{Observation}

%% Ott
% generated by Ott 0.24 from: atrees.ott
\newcommand{\ATreesdrule}[4][]{{\displaystyle\frac{\begin{array}{l}#2\end{array}}{#3}\quad\ATreesdrulename{#4}}}
\newcommand{\ATreesusedrule}[1]{\[#1\]}
\newcommand{\ATreespremise}[1]{ #1 \\}
\newenvironment{ATreesdefnblock}[3][]{ \framebox{\mbox{#2}} \quad #3 \\[0pt]}{}
\newenvironment{ATreesfundefnblock}[3][]{ \framebox{\mbox{#2}} \quad #3 \\[0pt]\begin{displaymath}\begin{array}{l}}{\end{array}\end{displaymath}}
\newcommand{\ATreesfunclause}[2]{ #1 \equiv #2 \\}
\newcommand{\ATreesnt}[1]{\mathit{#1}}
\newcommand{\ATreesmv}[1]{\mathit{#1}}
\newcommand{\ATreeskw}[1]{\mathbf{#1}}
\newcommand{\ATreessym}[1]{#1}
\newcommand{\ATreescom}[1]{\text{#1}}
\newcommand{\ATreesdrulename}[1]{\textsc{#1}}
\newcommand{\ATreescomplu}[5]{\overline{#1}^{\,#2\in #3 #4 #5}}
\newcommand{\ATreescompu}[3]{\overline{#1}^{\,#2<#3}}
\newcommand{\ATreescomp}[2]{\overline{#1}^{\,#2}}
\newcommand{\ATreesgrammartabular}[1]{\begin{supertabular}{llcllllll}#1\end{supertabular}}
\newcommand{\ATreesmetavartabular}[1]{\begin{supertabular}{ll}#1\end{supertabular}}
\newcommand{\ATreesrulehead}[3]{$#1$ & & $#2$ & & & \multicolumn{2}{l}{#3}}
\newcommand{\ATreesprodline}[6]{& & $#1$ & $#2$ & $#3 #4$ & $#5$ & $#6$}
\newcommand{\ATreesfirstprodline}[6]{\ATreesprodline{#1}{#2}{#3}{#4}{#5}{#6}}
\newcommand{\ATreeslongprodline}[2]{& & $#1$ & \multicolumn{4}{l}{$#2$}}
\newcommand{\ATreesfirstlongprodline}[2]{\ATreeslongprodline{#1}{#2}}
\newcommand{\ATreesbindspecprodline}[6]{\ATreesprodline{#1}{#2}{#3}{#4}{#5}{#6}}
\newcommand{\ATreesprodnewline}{\\}
\newcommand{\ATreesinterrule}{\\[5.0mm]}
\newcommand{\ATreesafterlastrule}{\\}
\newcommand{\ATreesmetavars}{
\ATreesmetavartabular{
 $ \ATreesmv{term\_var} ,\, \ATreesmv{b} ,\, \ATreesmv{w} ,\, \ATreesmv{x} ,\, \ATreesmv{y} ,\, \ATreesmv{z} ,\, \ATreesmv{v} $ &  \\
 $ \ATreesmv{index\_var} ,\, \ATreesmv{i} ,\, \ATreesmv{j} ,\, \ATreesmv{k} $ &  \\
}}

\newcommand{\ATreesatree}{
\ATreesrulehead{\ATreesnt{atree}  ,\ \ATreesnt{t}}{::=}{}\ATreesprodnewline
\ATreesfirstprodline{|}{\ATreesmv{b}}{}{}{}{}\ATreesprodnewline
\ATreesprodline{|}{ \ATreesnt{t_{{\mathrm{1}}}}  \odot  \ATreesnt{t_{{\mathrm{2}}}} }{}{}{}{}\ATreesprodnewline
\ATreesprodline{|}{ \ATreesnt{t_{{\mathrm{1}}}}  \sqcup  \ATreesnt{t_{{\mathrm{2}}}} }{}{}{}{}\ATreesprodnewline
\ATreesprodline{|}{ \ATreesnt{t_{{\mathrm{1}}}}  \otimes  \ATreesnt{t_{{\mathrm{2}}}} }{}{}{}{}\ATreesprodnewline
\ATreesprodline{|}{ \ATreesnt{t_{{\mathrm{1}}}}  \rhd  \ATreesnt{t_{{\mathrm{2}}}} }{}{}{}{}\ATreesprodnewline
\ATreesprodline{|}{ \textcopyright  \ATreesnt{t} }{}{}{}{}\ATreesprodnewline
\ATreesprodline{|}{\ATreessym{(}  \ATreesnt{t}  \ATreessym{)}} {\textsf{S}}{}{}{}\ATreesprodnewline
\ATreesprodline{|}{ \ATreesnt{t} } {\textsf{M}}{}{}{}\ATreesprodnewline
\ATreesprodline{|}{ \ATreesnt{t_{{\mathrm{1}}}}  \mathop{\mathsf{op} }  \ATreesnt{t_{{\mathrm{2}}}} } {\textsf{M}}{}{}{}\ATreesprodnewline
\ATreesprodline{|}{ \ATreesnt{t_{{\mathrm{1}}}}  \mathop{\mathsf{op_S} }  \ATreesnt{t_{{\mathrm{2}}}} } {\textsf{M}}{}{}{}}

\newcommand{\ATreesformula}{
\ATreesrulehead{\ATreesnt{formula}}{::=}{}\ATreesprodnewline
\ATreesfirstprodline{|}{\ATreesnt{judgement}}{}{}{}{}}

\newcommand{\ATreesjudgement}{
\ATreesrulehead{\ATreesnt{judgement}}{::=}{}}

\newcommand{\ATreesuserXXsyntax}{
\ATreesrulehead{\ATreesnt{user\_syntax}}{::=}{}\ATreesprodnewline
\ATreesfirstprodline{|}{\ATreesmv{term\_var}}{}{}{}{}\ATreesprodnewline
\ATreesprodline{|}{\ATreesmv{index\_var}}{}{}{}{}\ATreesprodnewline
\ATreesprodline{|}{\ATreesnt{atree}}{}{}{}{}}

\newcommand{\ATreesgrammar}{\ATreesgrammartabular{
\ATreesatree\ATreesinterrule
\ATreesformula\ATreesinterrule
\ATreesjudgement\ATreesinterrule
\ATreesuserXXsyntax\ATreesafterlastrule
}}

% defnss
\newcommand{\ATreesdefnss}{
}

\newcommand{\ATreesall}{\ATreesmetavars\\[0pt]
\ATreesgrammar\\[5.0mm]
\ATreesdefnss}


\input{ll-attack-ott}
\renewcommand{\LLAdrule}[4][]{{\displaystyle\frac{\begin{array}{l}#2\end{array}}{#3}\,\LLAdrulename{#4}}}
\renewcommand{\LLAdrulename}[1]{#1}
\renewcommand{\LLAdruleTXXvarName}{\text{id}_{\odot}}
\renewcommand{\LLAdruleTXXvarCName}{\text{id}_{\rhd}}
\renewcommand{\LLAdruleTXXparaName}{\odot}
\renewcommand{\LLAdruleTXXseqName}{\rhd}
\renewcommand{\LLAdruleTXXchoiceName}{\sqcup}

\renewcommand{\LLAdruleEXXvarName}{\text{id}_{\odot}}
\renewcommand{\LLAdruleEXXvarCName}{\text{id}_{\rhd}}
\renewcommand{\LLAdruleEXXchoiceContName}{\text{cont}_\sqcup}
\renewcommand{\LLAdruleEXXchoiceSymName}{\text{sym}_\sqcup}
\renewcommand{\LLAdruleEXXchoiceAssocName}{\text{assoc}_\sqcup}
\renewcommand{\LLAdruleEXXdistParaName}{\text{dis}_\odot}
\renewcommand{\LLAdruleEXXdistSeqName}{\text{dis}_\rhd}
\renewcommand{\LLAdruleEXXparaIName}{\odot_I}
\renewcommand{\LLAdruleEXXparaEName}{\odot_E}
\renewcommand{\LLAdruleEXXseqIName}{\rhd_I}
\renewcommand{\LLAdruleEXXseqEName}{\rhd_E}
\renewcommand{\LLAdruleEXXexName}{\text{sym}_\odot}
\renewcommand{\LLAdruleEXXchoiceName}{\sqcup}
\renewcommand{\LLAdruleEXXimpIName}{\multimap_I}
\renewcommand{\LLAdruleEXXimpEName}{\multimap_E}

\begin{document}

%% \conferenceinfo{PLAS '16}{October 24, 2016, Vienna, Austria}
%% \copyrightyear{2016}
%% \copyrightdata{}
%% \copyrightdoi{}
%% \titlebanner{}

\title{Project Update: A New Foundation of Attack Trees in Linear Logic}

\author{Harley Eades III}
\institute{Computer Science\\Augusta University \\ \href{mailto:heades@augusta.edu}{harley.eades@gmail.com}}

\maketitle 

\begin{abstract}
  In this short paper I provide an update on the status of newly
  funded research project investigating founding attack trees in the
  resource conscious theory called linear logic.  
\end{abstract}

\section{Introduction}
\label{sec:introduction}
%% the problem?
What is a mathematical model of attack trees?  There have been
numerous proposed answers to this question.  Some examples are
propositional logic, multisets, directed acyclic graphs, source sink
graphs (or parallel-series pomsets), Petri nets, and Markov processes.
Is there a unifying foundation in common to each of these proposed
models?  Furthermore, can this unifying foundation be used to further
the field of attack trees and build new tools for conducting threat
analysis?

The answer to the first question is positive, but the answer to the
second question is open.  Each of the proposed models listed above
have something in common.  They can all be modeled in some form of a
symmetric monoidal category\footnote{I provide a proof that the
  category of source sink graphs is monoidal in
  Appendix~\ref{sec:source_sink_graphs_are_symmetric_monoidal}.}
\cite{Tzouvaras:1998,Brown:1991,Fiore:2013,FrancescoAlbasini2010} --
for the definition of a symmetric monoidal category see
Appendix~\ref{sec:symmetric_monoidal_categories}.  That is all well
and good, but what can we gain from monoidal categories?

Monoidal categories are a mathematical model of linear logic as
observed through the beautiful Curry-Howard-Lambek correspondence
\cite{Mellies:2009}.  In linear logic every hypothesis must be used
exactly once, and hence, if we view a hypothesis as a resource, then
this property can be stated as every resource must be consumed.  This
linearity property is achieved by removing the structural rules for
weakening and contraction from classical or intuitionistic logic.
Thus, from a resource perspective, resources cannot be spontaneously
created or duplicated -- hence, propositional logic can be viewed as a
degenerate, from a resource perspective, form of linear logic.

Multisets and Petri nets both capture the idea that the nodes of an
attack tree capture both the attack action and the state -- the
resource -- of the system being analyzed. As it turns out, linear
logic has been shown to be a logical foundation for multisets
\cite{Tzouvaras:1998} and Petri Nets \cite{Brown:1991}.  Thus, linear
logic has the ability to model the state as well as attack actions of
the goals of an attack tree.  We propose that linear logic be used as
the logical foundation of attack trees.

This projects\footnote{This material is based upon work supported by
  the National Science Foundation CRII CISE Research Initiation grant,
  ``CRII:SHF: A New Foundation for Attack Trees Based on Monoidal
  Categories``, under Grant No. 1565557.} main goal is to determine
the suitability of linear logic as a foundation for attack trees.  The
type of attack trees we consider in this paper are attack trees with
sequential composition similar to Jhawar et al.~\cite{Jhawar:2015}.
Attack trees consist of two layers: the logical layer or process tree,
and the quantitative layer.  One interesting aspect of our proposed
foundation is that the logical layer can come to the aid of the
quantitative layer.

At the logical layer the base attacks of an attack tree will
correspond to atomic formulas in linear logic, and each branching node
of an attack tree will correspond to a binary operator in linear
logic.  Adding costs corresponds to annotating the atomic formulas
with some base cost, and then annotating the binary operators with a
cost computed from the costs of their respective left operand (left
subtree) and right operand (right subtree).  The most interesting
aspect of adding costs is that computing the costs at the branching
nodes is done during the construction of a derivation using the
inference rules of the logic.

The inference rules of linear logic provide a number of benefits when
constructing attack trees.  First, the inference rules will certify
that an attack tree is constructed correctly and all costs on
branching nodes will be computed from the costs of the left and right
subtrees.  On top of that the inference rules offer a means of proving
when two attack trees are equivalent. Leveraging the fact that
language of attack trees corresponds to a very simple fragment of
linear logic, e.g. linear implication is not needed, I conjecture
that proving equivalence of attack trees can be automated.  Thus, this
could be used in practice to certify the correctness of
transformations of attack trees maintaining the resource based
semantics.

\section{A Short Introduction to Ordered Linear Logic}
\label{sec:a_short_introduction_to_ordered_linear_logic}
\input{ll-intro-output}
% section a_short_introduction_to_ordered_linear_logic (end)


\section{Attack Trees}
\label{sec:attack_trees}
\input{attack-trees-output}
% section attack_trees (end)

\section{A Linear Logic for Attack Trees}
\label{sec:a_linear_logic_for_attack_trees}
\input{ll-attack-output}
% section a_linear_logic_for_attack_trees (end)

\bibliographystyle{plain}
\bibliography{ref}

\appendix

\section*{Appendix}
\label{sec:appendix}
\section{Symmetric Monoidal Categories}
\label{sec:symmetric_monoidal_categories}
This appendix provides the definitions of both categories in general,
and, in particular, symmetric monoidal closed categories.  We begin
with the definition of a category:
\begin{definition}
  \label{def:category}
  A \textbf{category}, $\cat{C}$, consists of the following data:
  \begin{itemize}
  \item A set of objects $\cat{C}_0$, each denoted by $A$, $B$, $C$, etc.
  \item A set of morphisms $\cat{C}_1$, each denoted by $f$, $g$, $h$, etc.
  \item Two functions $\mathsf{src}$, the source of a morphism, and
    $\mathsf{tar}$, the target of a morphism, from morphisms to
    objects.  If $\mathsf{src}(f) = A$ and $\mathsf{tar}(f) = B$, then
    we write $f : A \to B$.
  \item Given two morphisms $f : A \to B$ and $g : B \to C$, then the
    morphism $f;g : A \to C$, called the composition of $f$ and $g$,
    must exist.
  \item For every object $A \in \cat{C}_0$, the there must exist a
    morphism $\id_A : A \to A$ called the identity morphism on $A$.

  \item The following axioms must hold:
    \begin{itemize}
    \item (Identities) For any $f : A \to B$, $f;\id_B = f = \id_A;f$.
    \item (Associativity) For any $f : A \to B$, $g : B \to C$, and $h
      : C \to D$, $(f;g);h = f;(g;h)$.
    \end{itemize}
  \end{itemize}
\end{definition}

Categories are by definition very abstract, and it is due to this that
makes them so applicable.  The usual example of a category is the
category whose objects are all sets, and whose morphisms are
set-theoretic functions.  Clearly, composition and identities exist,
and satisfy the axioms of a category.  A second example is preordered
sets, $(A , \leq)$, where the objects are elements of $A$ and a
morphism $f : a \to b$ for elements $a, b \in A$ exists iff $a \leq
b$. Reflexivity yields identities, and transitivity yields
composition.  See the usual introductions for more examples \cite{?}.


Symmetric monoidal categories pair categories with a commutative
monoid like structure called the tensor product.  They are a
categorical semantics of linear logic \cite{?}.
\begin{definition}
  \label{def:monoidal-category}
  A \textbf{symmetric monoidal category (SMC)} is a category, $\cat{M}$,
  with the following data:
  \begin{itemize}
  \item An object $I$ of $\cat{M}$,
  \item A bi-functor $\otimes : \cat{M} \times \cat{M} \to \cat{M}$,
  \item The following natural isomorphisms:
    \[
    \begin{array}{lll}
      \lambda_A : I \otimes A \to A\\
      \rho_A : A \otimes I \to A\\      
      \alpha_{A,B,C} : (A \otimes B) \otimes C \to A \otimes (B \otimes C)\\
    \end{array}
    \]
  \item A symmetry natural transformation:
    \[
    \beta_{A,B} : A \otimes B \to B \otimes A
    \]
  \item Subject to the following coherence diagrams:
    \begin{mathpar}
      \bfig
      \vSquares|ammmmma|/->`->```->``<-/[
        ((A \otimes B) \otimes C) \otimes D`
        (A \otimes (B \otimes C)) \otimes D`
        (A \otimes B) \otimes (C \otimes D)``
        A \otimes (B \otimes (C \otimes D))`
        A \otimes ((B \otimes C) \otimes D);
        \alpha_{A,B,C} \otimes \id_D`
        \alpha_{A \otimes B,C,D}```
        \alpha_{A,B,C \otimes D}``
        \id_A \otimes \alpha_{B,C,D}]      
      
      \morphism(1433,1000)|m|<0,-1000>[
        (A \otimes (B \otimes C)) \otimes D`
        A \otimes ((B \otimes C) \otimes D);
        \alpha_{A,B \otimes C,D}]
      \efig
      \and
      \bfig
      \hSquares|aammmaa|/->`->`->``->`->`->/[
        (A \otimes B) \otimes C`
        A \otimes (B \otimes C)`
        (B \otimes C) \otimes A`
        (B \otimes A) \otimes C`
        B \otimes (A \otimes C)`
        B \otimes (C \otimes A);
        \alpha_{A,B,C}`
        \beta_{A,B \otimes C}`
        \beta_{A,B} \otimes \id_C``
        \alpha_{B,C,A}`
        \alpha_{B,A,C}`
        \id_B \otimes \beta_{A,C}]
      \efig      
    \end{mathpar}
    \begin{mathpar}
      \bfig
      \Vtriangle[
        (A \otimes I) \otimes B`
        A \otimes (I \otimes B)`
        A \otimes B;
        \alpha_{A,I,B}`
        \rho_{A}`
        \lambda_{B}]
      \efig
      \and
      \bfig
      \btriangle[
        A \otimes B`
        B \otimes A`
        A \otimes B;
        \beta_{A,B}`
        \id_{A \otimes B}`
        \beta_{B,A}]
      \efig
      \and
      \bfig
      \Vtriangle[
        I \otimes A`
        A \otimes I`
        A;
        \beta_{I,A}`
        \lambda_A`
        \rho_A]
      \efig
    \end{mathpar}    
  \end{itemize}
\end{definition}


%% \begin{definition}
%%   \label{def:SMCC}
%%   A \textbf{symmetric monoidal closed category (SMCC)} is a symmetric
%%   monoidal category, $(\cat{M},I,\otimes)$, such that, for any object
%%   $B$ of $\cat{M}$, the functor $- \otimes B : \cat{M} \to \cat{M}$
%%   has a specified right adjoint.  Hence, for any objects $A$ and $C$
%%   of $\cat{M}$ there is an object $A \limp B$ of $\cat{M}$ and a
%%   natural bijection:
%%   \[
%%   \Hom{\cat{M}}{A \otimes B}{C} \cong \Hom{\cat{M}}{A}{B \limp C}
%%   \]
%% \end{definition}

% section symmetric_monoidal_categories (end)

\section{Source Sink Graphs are Symmetric Monoidal}
\label{sec:source_sink_graphs_are_symmetric_monoidal}
In this appendix I show that the category of source-sink graphs
defined by Jhawar et al. \cite{Jhawar:2015} is symmetric monoidal.
First, recall the definition of source-sink graphs and their
homomorphisms.
\begin{definition}
  \label{def:source-sink-graphs}
  A \textbf{source-sink graph} over $\mathsf{B}$ is a tuple $G = (V ,
  E, s, z)$, where $V$ is the set of vertices, $E$ is a multiset of
  labeled edges with support $E^* \subseteq V \times \mathsf{B} \times
  V$, $s \in V$ is the unique start, $z \in V$ is the unique sink, and
  $s \neq z$.

  \ \\
  \noindent
  Suppose $G = (V , E, s, z)$ and $G' = (V' , E', s', z')$.  Then a
  \textbf{morphism between source-sink graphs}, $f : G \to G'$, is a
  graph homomorphism such that $f(s) = s'$ and $f(z) = z'$.
\end{definition}

Suppose $G = (V , E, s, z)$ and $G' = (V' , E', s', z')$ are two
source-sink graphs. Then given the above definition it is possible to
define sequential and non-communicating parallel composition of
source-sink graphs where I denote disjoint union of sets by $+$ (p
7. \cite{Jhawar:2015}):
\begin{center}
  \begin{math}
    \begin{array}{rll}
      \text{(Sequential Composition)} & G \rhd G'
      = ((V \mathop{\backslash} \{z \}) + V', E^{[s'/z]} + E', s , z')\\
      \text{(Parallel Composition)}   & G \odot G'
      = ((V \mathop{\backslash} \{s,z\}) + V', E^{[s'/s,z'/z]} + E', s' , z')\\
    \end{array}
  \end{math}
\end{center}

It is easy to see that we can define a category of source-sink graphs
and their homomorphisms.  Furthermore, it is a symmetric monoidal
category were parallel composition is the symmetric tensor product.
It is well-known that any category with co-products is symmetric
monoidal where the co-product is the tensor product.

I show here that parallel composition defines a co-product.  This
requires the definition of the following morphisms:
\begin{center}
  \begin{math}
    \begin{array}{lll}
      \mathsf{inj_1} : G_1 \to G_1 \odot G_2\\
      \mathsf{inj_2} : G_2 \to G_1 \odot G_2\\
      \langle f , g \rangle : G_1 \odot G_2 \to G\\
    \end{array}
  \end{math}
\end{center}
In the above $f : G_1 \to G$ and $g : G_2 \to G$ are two source-sink
graph homomorphisms.  Furthermore, the following diagram must commute:
\begin{center}
  \begin{math}
    \bfig
    \Atrianglepair/->`<-`->`->`<-/<700,500>[G`G_1`G_1 \odot G_2`G_2;f`\langle f , g \rangle`g`\mathsf{inj_1}`\mathsf{inj_2}]
    \efig
  \end{math}
\end{center}

Suppose $G_1 = (V_1 , E_1, s_1, z_1)$, $G_2 = (V_2 , E_2, s_2, z_2)$,
and $G = (V , E, s, z)$ are source-sink graphs, and $f : G_1 \to G$
and $g : G_2 \to G$ are source-sink graph morphisms -- note that
$f(s_1) = g(s_2) = s$ and $f(z_1) = g(z_2) = z$ by definition.  Then
we define the required co-product morphisms as follows:
\begin{center}
  \begin{tabular}{llllllllllllllllll}
    \begin{math}
    \begin{array}{lll}
      \mathsf{inj_1} : V_1 \to (V_1 \mathop{\backslash} \{s_1,z_1\}) + V_2\\
      \mathsf{inj_1}(s_1) = s_2\\
      \mathsf{inj_1}(z_1) = z_2\\
      \mathsf{inj_1}(v) = v \text{, otherwise}\\
    \end{array}
    \end{math}
    & \quad & \quad & \quad & \quad &\quad &\quad &\quad &\quad &\quad &\quad &
    \begin{math}
    \begin{array}{lll}
      \mathsf{inj_2} : V_2 \to (V_1 \mathop{\backslash} \{s_1,z_1\}) + V_2\\
      \mathsf{inj_2}(v) = v\\
      \\
      \\
    \end{array}
  \end{math}
  \end{tabular}

  \vspace{10px}
  \begin{math}
    \begin{array}{lll}
      \langle f , g \rangle : (V_1 \mathop{\backslash} \{s_1,z_1\}) + V_2 \to V\\      
      \langle f , g \rangle(v) = f(v), \text{ where } v \in V_1\\
      \langle f , g \rangle(v) = g(v), \text{ where } v \in V_2\\
    \end{array}
  \end{math}
\end{center}
It is easy to see that these define graph homomorphisms.  All that is
left to show is that the diagram from above commutes:
\begin{center}
  \begin{tabular}{lll}
    \begin{math}
    \begin{array}{lll}
      (\mathsf{inj_1};\langle f , g \rangle)(s_1)
      & = & \langle f , g \rangle(\mathsf{inj_1}(s_1))\\
      & = & g(s_2)\\     
      & = & s \\
      & = & f(s_1)\\
    \end{array}
    \end{math}
    & \quad & 
    \begin{math}
    \begin{array}{lll}
      (\mathsf{inj_1};\langle f , g \rangle)(z_1)
      & = & \langle f , g \rangle(\mathsf{inj_1}(z_1))\\
      & = & g(z_2)\\     
      & = & z\\
      & = & f(z_1)\\
    \end{array}
  \end{math}
  \end{tabular}
\end{center}
Now for any $v \in V_1$ we have the following:
\begin{center}
  \begin{math}
    \begin{array}{lll}
      (\mathsf{inj_1};\langle f , g \rangle)(v)
      & = & \langle f , g \rangle(\mathsf{inj_1}(v))\\
      & = & f(v)\\
    \end{array}
  \end{math}
\end{center}
The equation for $\mathsf{inj_2}$ is trivial, because $\mathsf{inj_2}$
is the identity. 

% section source_sink_graphs_are_symmetric_monoidal (end)
% section appendix (end)



\end{document}

%%% Local Variables: 
%%% mode: latex
%%% TeX-master: t
%%% End: 

