\documentclass{llncs}

\usepackage{wrapfig}
\usepackage{amssymb,amsmath}
\usepackage{cmll}
\usepackage{stmaryrd}
\usepackage{todonotes}
\usepackage{mathpartir}
\usepackage{hyperref}
\usepackage[barr]{xy}
\usepackage{mdframed}
\usepackage{thmtools, thm-restate}

%% This renames Barr's \to to \mto.  This allows us to use \to for imp
%% and \mto for a inline morphism.
\let\mto\to
\let\to\relax
\newcommand{\to}{\rightarrow}

% Commands that are useful for writing about type theory and programming language design.
%% \newcommand{\case}[4]{\text{case}\ #1\ \text{of}\ #2\text{.}#3\text{,}#2\text{.}#4}
\newcommand{\interp}[1]{\llbracket #1 \rrbracket}
\newcommand{\normto}[0]{\rightsquigarrow^{!}}
\newcommand{\join}[0]{\downarrow}
\newcommand{\redto}[0]{\rightsquigarrow}
\newcommand{\nat}[0]{\mathbb{N}}
\newcommand{\fun}[2]{\lambda #1.#2}
\newcommand{\CRI}[0]{\text{CR-Norm}}
\newcommand{\CRII}[0]{\text{CR-Pres}}
\newcommand{\CRIII}[0]{\text{CR-Prog}}
\newcommand{\subexp}[0]{\sqsubseteq}
%% Must include \usepackage{mathrsfs} for this to work.
\newcommand{\powerset}[0]{\mathscr{P}}

\date{}

% Cat commands.
\newcommand{\cat}[1]{\mathcal{#1}}
\newcommand{\catop}[1]{\cat{#1}^{\mathsf{op}}}
\newcommand{\Hom}[3]{\mathsf{Hom}_{\cat{#1}}(#2,#3)}
\newcommand{\limp}[0]{\multimap}
\newcommand{\dial}[0]{\mathsf{Dial_3}(\mathsf{Sets})}
\newcommand{\dialSets}[1]{\mathsf{Dial_{#1}}(\mathsf{Sets})}
\newcommand{\dcSets}[1]{\mathsf{DC_{#1}}(\mathsf{Sets})}
\newcommand{\sets}[0]{\mathsf{Sets}}
\newcommand{\obj}[1]{\mathsf{Obj}(#1)}
\newcommand{\mor}[1]{\mathsf{Mor(#1)}}
\newcommand{\id}[0]{\mathsf{id}}
\newcommand{\lett}[0]{\mathsf{let}\,}
\newcommand{\inn}[0]{\,\mathsf{in}\,}
\newcommand{\cur}[1]{\mathsf{cur}(#1)}
\newcommand{\curi}[1]{\mathsf{cur}^{-1}(#1)}

\newenvironment{changemargin}[2]{%
  \begin{list}{}{%
    \setlength{\topsep}{0pt}%
    \setlength{\leftmargin}{#1}%
    \setlength{\rightmargin}{#2}%
    \setlength{\listparindent}{\parindent}%
    \setlength{\itemindent}{\parindent}%
    \setlength{\parsep}{\parskip}%
  }%
  \item[]}{\end{list}}

%% % Theorems
%% \newtheorem{theorem}{Theorem}
%% \newtheorem{lemma}[theorem]{Lemma}
%% \newtheorem{fact}[theorem]{Fact}
%% \newtheorem{corollary}[theorem]{Corollary}
%% \newtheorem{definition}[theorem]{Definition}
%% \newtheorem{remark}[theorem]{Remark}
%% \newtheorem{proposition}[theorem]{Proposition}
%% \newtheorem{notn}[theorem]{Notation}
%% \newtheorem{observation}[theorem]{Observation}

%% Ott
\input{indexed-atll-ott}
\input{atll-ott}
\input{oll-ott}
\renewcommand{\IATLLdrule}[4][]{{\displaystyle\frac{\begin{array}{l}#2\end{array}}{#3}\,\IATLLdrulename{#4}}}
\renewcommand{\IATLLdrulename}[1]{#1}
\renewcommand{\IATLLdruleTXXvarName}{\text{id}_{\odot}}
\renewcommand{\IATLLdruleTXXparaName}{\odot}
\renewcommand{\IATLLdruleTXXseqName}{\rhd}
\renewcommand{\IATLLdruleTXXchoiceName}{\sqcup}

\renewcommand{\IATLLdruleEXXvarName}{\text{id}_{\odot}}
\renewcommand{\IATLLdruleEXXvarCName}{\text{id}_{\rhd}}
\renewcommand{\IATLLdruleEXXvarCCName}{\text{id}_{\sqcup}}
\renewcommand{\IATLLdruleEXXdistParaOneName}{\text{dis}_{\odot_1}}
\renewcommand{\IATLLdruleEXXdistSeqOneName}{\text{dis}_{\rhd_1}}
\renewcommand{\IATLLdruleEXXdistParaTwoName}{\text{dis}_{\odot_2}}
\renewcommand{\IATLLdruleEXXdistSeqTwoName}{\text{dis}_{\rhd_2}}
\renewcommand{\IATLLdruleEXXparaIName}{\odot_I}
\renewcommand{\IATLLdruleEXXparaEName}{\odot_E}
\renewcommand{\IATLLdruleEXXseqIName}{\rhd_I}
\renewcommand{\IATLLdruleEXXseqEName}{\rhd_E}
\renewcommand{\IATLLdruleEXXexPName}{\text{sym}_\odot}
\renewcommand{\IATLLdruleEXXexCName}{\text{sym}_\sqcup}
\renewcommand{\IATLLdruleEXXimpIName}{\multimap_I}
\renewcommand{\IATLLdruleEXXimpEName}{\multimap_E}
\renewcommand{\ATLLdruleEXXcompName}{\text{comp}_\multimap}

\renewcommand{\ATLLdrule}[4][]{{\displaystyle\frac{\begin{array}{l}#2\end{array}}{#3}\,\ATLLdrulename{#4}}}
\renewcommand{\ATLLdrulename}[1]{#1}
\renewcommand{\ATLLdruleTXXvarName}{\text{id}_{\odot}}
\renewcommand{\ATLLdruleTXXparaName}{\odot}
\renewcommand{\ATLLdruleTXXseqName}{\rhd}
\renewcommand{\ATLLdruleTXXchoiceName}{\sqcup}

\renewcommand{\ATLLdruleEXXvarName}{\text{id}_{\odot}}
\renewcommand{\ATLLdruleEXXvarCName}{\text{id}_{\rhd}}
\renewcommand{\ATLLdruleEXXvarCCName}{\text{id}_{\sqcup}}
\renewcommand{\ATLLdruleEXXdistParaOneName}{\text{dis}_{\odot_1}}
\renewcommand{\ATLLdruleEXXdistSeqOneName}{\text{dis}_{\rhd_1}}
\renewcommand{\ATLLdruleEXXdistParaTwoName}{\text{dis}_{\odot_2}}
\renewcommand{\ATLLdruleEXXdistSeqTwoName}{\text{dis}_{\rhd_2}}
\renewcommand{\ATLLdruleEXXparaIName}{\odot_I}
\renewcommand{\ATLLdruleEXXparaEName}{\odot_E}
\renewcommand{\ATLLdruleEXXseqIName}{\rhd_I}
\renewcommand{\ATLLdruleEXXseqEName}{\rhd_E}
\renewcommand{\ATLLdruleEXXexPName}{\text{sym}_\odot}
\renewcommand{\ATLLdruleEXXexCName}{\text{sym}_\sqcup}
\renewcommand{\ATLLdruleEXXimpIName}{\multimap_I}
\renewcommand{\ATLLdruleEXXimpEName}{\multimap_E}
\renewcommand{\ATLLdruleEXXweakSName}{\text{wk}_\odot}
\renewcommand{\ATLLdruleEXXweakPName}{\text{wk}_\rhd}
\renewcommand{\ATLLdruleEXXchoiIName}{\sqcup_I}
\renewcommand{\ATLLdruleEXXchoiEName}{\sqcup_E}
\renewcommand{\IATLLdruleEXXdupName}{\text{dup}_\sqcup}
\renewcommand{\IATLLdruleEXXcontName}{\text{cont}_\sqcup}

\renewcommand{\OLLdrule}[4][]{{\displaystyle\frac{\begin{array}{l}#2\end{array}}{#3}\,\OLLdrulename{#4}}}
\renewcommand{\OLLdrulename}[1]{#1}
\renewcommand{\OLLdrulevarName}{\text{id}_{\odot}}
\renewcommand{\OLLdrulevarCName}{\text{id}_{\rhd}}
\renewcommand{\OLLdruleparaIName}{\odot_I}
\renewcommand{\OLLdruleparaEName}{\odot_E}
\renewcommand{\OLLdruleseqIName}{\rhd_I}
\renewcommand{\OLLdruleseqEName}{\rhd_E}
\renewcommand{\OLLdruleexName}{\text{sym}_\odot}
\renewcommand{\OLLdruleimpIName}{\multimap_I}
\renewcommand{\OLLdruleimpEName}{\multimap_E}
\renewcommand{\OLLdruleTopName}{\top}
\renewcommand{\OLLdruleconjIName}{\times_I}
\renewcommand{\OLLdruleconjEOneName}{\times_{E_1}}
\renewcommand{\OLLdruleconjETwoName}{\times_{E_2}}
  
\begin{document}

\title{A New Foundation of Attack Trees in Linear Logic}

\author{Harley Eades III}
\institute{Computer Science\\Augusta University \\ \href{mailto:heades@augusta.edu}{harley.eades@gmail.com}}

\maketitle 

\begin{abstract}
  In this paper I provide an update on an ongoing research project
  investigating founding attack trees in the resource conscious theory
  called linear logic.  I introduce a new linear logic called the
  Attack Tree Linear Logic (ATLL) which models attack trees as
  formulas and uses implication to reason about attack trees.  I then
  introduce a new generalization of Bucciarelli and Ehrhard’s indexed
  linear logic to obtain Indexed ATLL which adds costs to attack
  trees.  The most interesting aspect of Indexed ATLL is that
  implication is endowed with a relation for taking into consideration
  the costs while reasoning about attack trees.
\end{abstract}

\section{Introduction}
\label{sec:introduction}
\input{intro-output}

\section{A Brief Introduction to Ordered Linear Logic}
\label{sec:a_brief_introduction_to_ordered_linear_logic}
\input{oll-output}
% section a_brief_introduction_to_ordered_linear_logic (end)


\section{Attack Trees without Costs}
\label{sec:attack_trees_without_costs}
\input{attack-trees-output}
% section attack_trees_without_costs (end)

\section{Attack Trees with Costs}
\label{sec:attack_trees_with_costs}
\input{indexed-atll-output}
% section attack_trees_with_costs (end)

\section{Related Work}
\label{sec:related_work}
TODO
% section related_work (end)


\section{Conclusion and Future Work}
\label{sec:conclusion}
I introduced a new linear logic for reasoning about attack trees
called the Attack Tree Linear Logic (ATLL) with and without costs.
Attack trees are modeled as logical formulas and proving relationships
between attack trees corresponds to proving implications.  Adding
costs to ATLL results in a new indexed linear logic called Indexed
ATLL.  Then I showed how endowing implication with a relation on costs
can be used to include the costs when reasoning about attack trees.

ATLL is just one part of this project, the following are some next
steps with respect to (Indexed) ATLL and other parts of the project:
\begin{itemize}
\item Define a term assignment for (Indexed) ATLL that can be used to
  define attack trees using a functional programming language.
\item Determine the applicability of automating reasoning on attack
  trees in (Indexed) ATLL using proof search.  
\item Finish studying the categorical model to which (Indexed) ATLL is
  based.
\item Finishing the development of a modular and stand alone IDE for
  attack tree in Javascript that can be used with various
  formalizations of attack trees.  Hopefully, the community will
  be able to make use of this interface.
\item Develop the concrete implementation, called Lina, of (Indexed)
  ATLL with automation.
\end{itemize}
% section conclusion (end)


\bibliographystyle{plain}
\bibliography{ref}

\appendix

\section*{Appendix}
\label{sec:appendix}

\section{The Full Specification of Indexed ATLL}
\label{subsec:the_full_specification_of_indexed_atll}
\input{full-indexed-atll-output}
% section the_full_specification_of_indexed_atll (end)

% section appendix (end)



\end{document}

%%% Local Variables: 
%%% mode: latex
%%% TeX-master: t
%%% End: 

