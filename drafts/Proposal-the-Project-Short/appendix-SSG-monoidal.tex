In this appendix I show that the category of source-sink graphs
defined by Jhawar et al. \cite{Jhawar:2015} is symmetric monoidal.
First, recall the definition of source-sink graphs and their
homomorphisms.
\begin{definition}
  \label{def:source-sink-graphs}
  A \textbf{source-sink graph} over $\mathsf{B}$ is a tuple $G = (V ,
  E, s, z)$, where $V$ is the set of vertices, $E$ is a multiset of
  labeled edges with support $E^* \subseteq V \times \mathsf{B} \times
  V$, $s \in V$ is the unique start, $z \in V$ is the unique sink, and
  $s \neq z$.

  \ \\
  \noindent
  Suppose $G = (V , E, s, z)$ and $G' = (V' , E', s', z')$.  Then a
  \textbf{morphism between source-sink graphs}, $f : G \to G'$, is a
  graph homomorphism such that $f(s) = s'$ and $f(z) = z'$.
\end{definition}

Suppose $G = (V , E, s, z)$ and $G' = (V' , E', s', z')$ are two
source-sink graphs. Then given the above definition it is possible to
define sequential and non-communicating parallel composition of
source-sink graphs where I denote disjoint union of sets by $+$ (p
7. \cite{Jhawar:2015}):
\begin{center}
  \small
  \begin{math}
    \begin{array}{lll}
      \text{Sequential Composition}:\\
      G \rhd G' = ((V \mathop{\backslash} \{z \}) + V', E^{[s'/z]} + E', s , z')\\\\
      \text{Parallel Composition}:\\
      G \odot G'
      = ((V \mathop{\backslash} \{s,z\}) + V', E^{[s'/s,z'/z]} + E', s' , z')\\
    \end{array}
  \end{math}
\end{center}

It is easy to see that we can define a category of source-sink graphs
and their homomorphisms.  Furthermore, it is a symmetric monoidal
category were parallel composition is the symmetric tensor product.
It is well-known that any category with co-products is symmetric
monoidal where the co-product is the tensor product.

I show here that parallel composition defines a co-product.  This
requires the definition of the following morphisms:
\begin{center}
  \begin{math}
    \begin{array}{lll}
      \mathsf{inj_1} : G_1 \to G_1 \odot G_2\\
      \mathsf{inj_2} : G_2 \to G_1 \odot G_2\\
      \langle f , g \rangle : G_1 \odot G_2 \to G\\
    \end{array}
  \end{math}
\end{center}
In the above $f : G_1 \to G$ and $g : G_2 \to G$ are two source-sink
graph homomorphisms.  Furthermore, the following diagram must commute:
\begin{center}
  \begin{math}
    \bfig
    \Atrianglepair/->`<-`->`->`<-/<700,500>[G`G_1`G_1 \odot G_2`G_2;f`\langle f , g \rangle`g`\mathsf{inj_1}`\mathsf{inj_2}]
    \efig
  \end{math}
\end{center}

Suppose $G_1 = (V_1 , E_1, s_1, z_1)$, $G_2 = (V_2 , E_2, s_2, z_2)$,
and $G = (V , E, s, z)$ are source-sink graphs, and $f : G_1 \to G$
and $g : G_2 \to G$ are source-sink graph morphisms -- note that
$f(s_1) = g(s_2) = s$ and $f(z_1) = g(z_2) = z$ by definition.  Then
we define the required co-product morphisms as follows:
\begin{center}
  \begin{tabular}{llllllllllllllllll}
    \begin{math}
    \begin{array}{lll}
      \mathsf{inj_1} : V_1 \to (V_1 \mathop{\backslash} \{s_1,z_1\}) + V_2\\
      \mathsf{inj_1}(s_1) = s_2\\
      \mathsf{inj_1}(z_1) = z_2\\
      \mathsf{inj_1}(v) = v \text{, otherwise}\\
    \end{array}
    \end{math}\\
    \\
    \begin{math}
    \begin{array}{lll}
      \mathsf{inj_2} : V_2 \to (V_1 \mathop{\backslash} \{s_1,z_1\}) + V_2\\
      \mathsf{inj_2}(v) = v\\      \\
    \end{array}
  \end{math}
  \end{tabular}

  \begin{math}
    \begin{array}{lll}
      \langle f , g \rangle : (V_1 \mathop{\backslash} \{s_1,z_1\}) + V_2 \to V\\      
      \langle f , g \rangle(v) = f(v), \text{ where } v \in V_1\\
      \langle f , g \rangle(v) = g(v), \text{ where } v \in V_2\\
    \end{array}
  \end{math}
\end{center}
It is easy to see that these define graph homomorphisms.  All that is
left to show is that the diagram from above commutes:
\begin{center}
  \begin{tabular}{lll}
    \begin{math}
    \begin{array}{lll}
      (\mathsf{inj_1};\langle f , g \rangle)(s_1)
      & = & \langle f , g \rangle(\mathsf{inj_1}(s_1))\\
      & = & g(s_2)\\     
      & = & s \\
      & = & f(s_1)\\
    \end{array}
    \end{math}
    \\\\
    \begin{math}
    \begin{array}{lll}
      (\mathsf{inj_1};\langle f , g \rangle)(z_1)
      & = & \langle f , g \rangle(\mathsf{inj_1}(z_1))\\
      & = & g(z_2)\\     
      & = & z\\
      & = & f(z_1)\\
    \end{array}
  \end{math}
  \end{tabular}
\end{center}
Now for any $v \in V_1$ we have the following:
\begin{center}
  \begin{math}
    \begin{array}{lll}
      (\mathsf{inj_1};\langle f , g \rangle)(v)
      & = & \langle f , g \rangle(\mathsf{inj_1}(v))\\
      & = & f(v)\\
    \end{array}
  \end{math}
\end{center}
The equation for $\mathsf{inj_2}$ is trivial, because $\mathsf{inj_2}$
is the identity. 
