In this section I introduce attack trees with sequential conjunction
-- sometimes referred to as SAND attack trees -- without cost
annotations, and then the Attack Tree Linear Logic (ATLL).  This
formulation of attack trees was first proposed by Jhawar et
al. \cite{Jhawar:2015}.
\begin{definition}
  \label{def:atrees}
  Suppose $\mathsf{B}$ is a set of base attacks whose elements are
  denoted by $[[b]]$.  Then an \textbf{attack tree} is defined by
  the following grammar:
  \[
  \begin{array}{lll}
    [[T]] ::= [[b]] \mid [[T1 (.) T2]] \mid [[T1 + T2]] \mid [[T1 > T2]]\\
  \end{array}
  \]
  I denote unsynchronized parallel conjunction of attacks by $[[T1 (.) T2]]$,
  choice between attacks by $[[T1 + T2]]$, and sequential
  conjunction of attacks by $[[T1 > T2]]$.  
\end{definition}
Next I define ATLL and discuss how it can be used for reasoning about
attack trees.

In ATLL attack trees are modeled as formulas, and thus, reasoning
about attack trees should correspond to proving implications between
them.  ATLL is an extension of OLL
(Section~\ref{sec:a_brief_introduction_to_ordered_linear_logic}) with
a new operator for modeling choice between attack trees.  The ATLL
syntax is defined by the following definition.
\begin{definition}
  \label{def:ATLL-syntax}
  Suppose $\mathsf{B}$ is a set of base attacks whose elements are
  denoted by $[[b]]$. The syntax of ATLL formulas and contexts are
  defined as follows:
  \[
  \begin{array}{cll}
    \text{(formulas)}         & [[E]] ::= [[b]] \mid [[E1 (.) E1]] \mid [[E1 + E2]] \mid [[E1 > E2]] \mid [[E1 -o E2]]\\
    \text{(base contexts)}    & [[G]],[[D]] ::= [[.]] \mid [[b]] \mid [[G,D]]\\
    \text{(general contexts)} & [[I]],[[H]],[[P]] ::= [[.]] \mid [[E]] \mid [[G,D]]\\\\
  \end{array}
  \]
\end{definition}
\begin{figure}\footnotesize
  \begin{mdframed}
    \begin{mathpar}
      \ATLLdruleTXXvar{} \and
      \ATLLdruleTXXpara{} \and
      \ATLLdruleTXXseq{} \and
      \ATLLdruleTXXchoice{}
    \end{mathpar}
  \end{mdframed}
  \caption{Well Formed Attack Trees in ATLL}
  \label{fig:wf-attack-trees}
\end{figure}
\vspace{-10px}

One can also view ATLL formulas as being defined as an extension of
the syntax for attack trees with linear implication denoted by $[[A -o
    B]]$.  Suppose $[[E]]$ is an ATLL formula such that every base
attack in $[[E]]$ is in the base context $[[G]]$. Then $[[E]]$ is an
attack tree if and only if $[[G |-T E]]$ holds with respect to the
rules in Figure~\ref{fig:wf-attack-trees}.  These inference rules
allow one to restrict themselves to attack trees when using ATLL.

The inference rules for in Figure~\ref{fig:wf-attack-trees} enforce
several properties.  First, they enforce that every base attack is
used exactly once, and that an attack tree does not mention linear
implication.  Furthermore, they allow us to restrict our reasoning to
attack trees in ATLL which I enforce in the hope that it will result
in our reasoning being more amenable to automation, but this is left
to future work.

\begin{figure}\scriptsize
  \begin{mdframed}
    \begin{mathpar}
      \ATLLdruleEXXvar{} \and
      \ATLLdruleEXXvarCC{} \and
      \ATLLdruleEXXvarC{} \and      
      \ATLLdruleEXXparaI{} \and
      \ATLLdruleEXXparaE{} \and
      \ATLLdruleEXXchoiI{} \and
      \ATLLdruleEXXchoiE{} \and
      \ATLLdruleEXXseqI{} \and
      \ATLLdruleEXXseqE{} \and
      \ATLLdruleEXXexP{} \and
      \ATLLdruleEXXchoiI{} \and
      \ATLLdruleEXXchoiE{} \and
      \ATLLdruleEXXexC{} \and
      \IATLLdruleEXXdup{} \and
      \IATLLdruleEXXcont{} \and
      \ATLLdruleEXXimpI{} \and
      \ATLLdruleEXXimpE{} \and
      \ATLLdruleEXXcomp{} 
    \end{mathpar}
  \end{mdframed}
  \caption{ATLL Logical Inference Rules}
  \label{fig:atll-rules}
\end{figure}
\begin{figure}
  \begin{mdframed}
    \begin{mathpar}
      \ATLLdruleEXXdistParaOne{} \and      
      \ATLLdruleEXXdistParaTwo{} \and
      \ATLLdruleEXXdistSeqOne{} \and
      \ATLLdruleEXXdistSeqTwo{}
    \end{mathpar}
  \end{mdframed}
  \caption{ATLL Attack Tree Axioms}
  \label{fig:atll-attack-tree-axioms}
\end{figure}


The ATLL logical inference rules are defined in
Figure~\ref{fig:atll-rules}.  The sequent of this system is denoted by
$[[I;H;P |- E]]$ and consists of the general context $[[I]]$
associated with sequential conjunction, $[[H]]$ associated with
choice, and $[[P]]$ associated with parallel conjunction.  This is
similar to OLL; see
Section~\ref{sec:a_brief_introduction_to_ordered_linear_logic}.  It is
required for hypotheses used with choice to be separated, because the
attack tree equations for choice require contraction and a weak form
of weakening; see the rules $\ATLLdruleEXXdupName{}$ and
$\IATLLdruleEXXcontName{}$.

These rules make up a basic linear logic for attack trees.  The reader
should note that the rule $\ATLLdruleEXXcompName{}$ is very useful
when proving properties of attack trees, but is actually a derivable
rule. Furthermore, one can see quite clearly that these rules are an
extension of the rules for OLL, but instead of non-linear conjunction
it has non-linear choice.  In ATLL one can prove that choice,
sequential conjunction, and parallel conjunction are associative, and
choice and parallel conjunction are symmetric. However, one cannot
prove the usual distributive laws associated with attack trees with
sequential conjunction \cite{Jhawar:2015}.

To overcome this limitation we add the attack tree axioms in
Figure~\ref{fig:atll-attack-tree-axioms} to ATLL.  The logical
connective $[[o-o]]$ is linear biimplication which is defined in the
usual way as classical implication.  The reader should note that these
axioms are restricted to attack trees only and not applicable to full
ATLL formulas.  In addition, the rule $\ATLLdruleEXXdistParaTwoName{}$
is derivable, because parallel conjunction is symmetric.

ATLL allows one to prove every equivalence for attack trees as linear
implications that one might expect, in fact, the following are all
provable in ATLL:
\[\footnotesize
\begin{array}{lll}  
  [[h(T + T) o-o T]] & \quad & [[h(T1 (.) (T2 + T3)) o-o h((T1 (.) T2) + (T1 (.) T3))]] \\
  [[h(T1 (.) T2) o-o h(T2 (.) T1)]] & & [[h((T2 + T3) (.) T1) o-o h((T2 (.) T1) + (T3 (.) T1))]]\\
  [[h(T1 + T2) o-o h(T2 + T1)]] & & [[h(T1 > (T2 + T3)) o-o h((T1 > T2) + (T1 > T3))]]\\
  [[h((T1 + T2) + T3) o-o h(T1 + (T2 + T3))]] & & [[h((T2 + T3) > T1) o-o h((T2 > T1) + (T2 > T1))]]\\
  [[h((T1 (.) T2) (.) T3) o-o h(T1 (.) (T2 (.) T3))]] && [[h((T1 > T2) > T3) o-o h(T1 > (T2 > T3))]]\\
\end{array}
\]
These are all of the equations given by Jhawar et
al.~\cite{Jhawar:2015}.  I conjecture that it should be possible to
define a proof search algorithm for ATLL to automate equational
reasoning on attack trees, but this is future work.  Using implication
as a means to reason about attack trees opens the door for different
types of reasoning.

Suppose $[[T1]]$ is a larger attack tree, and we wish to know if the
attack tree $[[T2]]$ is contained within $[[T1]]$, that is, $[[T2]]$
is a subattack tree of $[[T1]]$.  Currently, ATLL will not allow us to
prove this, but it can if we add the following rules:
\vspace{-5px}
\mdfdefinestyle{mdfexample}{innerbottommargin=1px, innertopmargin=-8.5px}
\begin{mdframed}[style=mdfexample]
  \begin{mathpar}\footnotesize
    \ATLLdruleEXXweakS{} \and
    \ATLLdruleEXXweakC{} \and
    \ATLLdruleEXXweakP{}
  \end{mathpar}
\end{mdframed}
These rules allow us to discard parts of a larger attack tree while
constructing proof derivations, but this only works for attack trees
not general ATLL formulas.  Thus, with these rules ATLL becomes
slightly affine.  We can now use these rules to determine if
$[[T1 -o T2]]$ holds, and if it does, then $[[T2]]$ is a
subattack tree of $[[T1]]$.

I now give an example to illustrate using the system to
prove properties of attack trees.  Consider the two example trees from
the introduction:
\[
\begin{array}{lll}
  [[T1]] := [[(B1 (.) (B2 + B3)) > B4]]\\
  [[T2]] := [[((B1 (.) B2) > B4) + ((B1 (.) B3) > B4)]]\\
\end{array}
\]
The attack trees $[[T1]]$ and $[[T2]]$ are equivalent if and only if
$[[T1 o-o T2]]$ holds.  First, ATLL attack tree axioms are
biimplications, and hence, we may treat these axioms as equivalences.
Thus, using the ATLL attack tree axioms we can prove that $[[T1]]$ is
equivalent to $[[T2]]$ informally as follows:
\[
\begin{array}{lllll}
  [[T1]]
  & := & [[(B1 (.) (B2 + B3)) > B4]]                       & \text{(definition)}\\
  & [[o-o]] & [[((B1 (.) B2) + (B1 (.) B3)) > B4]]         & (\text{rule }\ATLLdruleEXXdistParaOneName{})\\
  & [[o-o]] & [[((B1 (.) B2) > B4) + ((B1 (.) B3) > B4)]]  & (\text{rule }\ATLLdruleEXXdistSeqTwoName{})\\
  & := & [[T2]]                                            & \text{(definition)}\\
\end{array}
\]
The corresponding formal derivation is quite large, but is easily
constructible using the previous reasoning.

%%% Local Variables: 
%%% mode: latex
%%% TeX-master: main.tex
%%% End: 
