In this section I introduce attack trees with sequential conjunction
-- sometimes referred to SAND attack trees -- without cost annotations
in the hope that doing so will make the main ideas of this work more
clear to the reader.  This formulation of attack trees was first
proposed by Jhawar et al. \cite{Jhawar:2015}. First, I define the
syntax of attack trees then a linear logic for reasoning about them.
\begin{definition}
  \label{def:atrees}
  Suppose $\mathsf{B}$ is a set of base attacks whose elements are
  denoted by $[[b]]$.  Then an \textbf{attack tree} is defined by
  the following grammar:
  \[
  \begin{array}{lll}
    [[T]] ::= [[b]] \mid [[T1 (.) T2]] \mid [[T1 + T2]] \mid [[T1 > T2]]\\
  \end{array}
  \]
  I denote unsynchronized parallel conjunction of attacks by $[[T1 (.) T2]]$,
  choice between attacks by $[[T1 + T2]]$, and sequential
  conjunction of attacks by $[[T1 > T2]]$.  
\end{definition}

Now we define the Attack Tree Linear Logic (ATLL) for reasoning about
attack trees. In ATLL attack trees are modeled as formulas, and thus,
reasoning about attack trees should correspond to proving implications
between attack trees.  Furthermore, some care must be taken so that
choice and parallel conjunction are symmetric, but sequential
conjunction is not.  We enforce this property in ATLL by separating
hypotheses as is usually done in ordered linear logic
\cite{Polakow:2001}.
\begin{definition}
  \label{def:ATLL-syntax}
  Suppose $\mathsf{B}$ is a set of base attacks whose elements are
  denoted by $[[b]]$. The syntax of ATLL formulas and contexts are
  defined as follows:
  \[
  \begin{array}{rll}
    \text{(formulas)}         & [[E]] ::= [[b]] \mid [[E1 (.) E1]] \mid [[E1 + E2]] \mid [[E1 > E2]] \mid [[E1 -o E2]]\\
    \text{(base contexts)}    & [[G]],[[D]] ::= [[.]] \mid [[b]] \mid [[G,D]]\\
    \text{(general contexts)} & [[I]],[[P]] ::= [[.]] \mid [[E]] \mid [[G,D]]\\\\
  \end{array}
  \]
\end{definition}
\begin{figure}
  \begin{mdframed}
    \begin{mathpar}
      \ATLLdruleTXXvar{} \and
      \ATLLdruleTXXpara{} \and
      \ATLLdruleTXXseq{} \and
      \ATLLdruleTXXchoice{}
    \end{mathpar}
  \end{mdframed}
  \caption{Well Formed Attack Trees in ATLL}
  \label{fig:wf-attack-trees}
\end{figure}

ATLL formulas are defined as an extension of the syntax for attack
trees with linear implication.  An attack tree then corresponds to an
ATLL formula without implication.  Suppose $[[E]]$ is an ATLL formula
such that every base attack in $[[E]]$ which is a leaf of sequential
conjunction is in the base context $[[G]]$, and every leaf of parallel
conjunction is in the base context $[[D]]$. Then $[[E]]$ is an attack tree
if and only if $[[D |-T E]]$ holds with respect to the rules in
Figure~\ref{fig:wf-attack-trees}.

The inference rules in Figure~\ref{fig:wf-attack-trees} enforce
several properties.  First, they enforce that every base attack is
used exactly once, and that an attack tree does not mention linear
implication.  Furthermore, they allow us to restrict our reasoning to
attack trees which I conjecture will result in our reasoning being
more amenable to automation.

\begin{figure}
  \begin{mdframed}
    \begin{mathpar}
      \ATLLdruleEXXvar{} \and
      \ATLLdruleEXXvarCC{} \and
      \ATLLdruleEXXvarC{} \and      
      \ATLLdruleEXXparaI{} \and
      \ATLLdruleEXXparaE{} \and
      \ATLLdruleEXXchoiI{} \and
      \ATLLdruleEXXchoiE{} \and
      \ATLLdruleEXXseqI{} \and
      \ATLLdruleEXXseqE{} \and
      \ATLLdruleEXXexP{} \and
      \ATLLdruleEXXchoiI{} \and
      \ATLLdruleEXXchoiE{} \and
      \ATLLdruleEXXexC{} \and
      \IATLLdruleEXXdup{} \and
      \IATLLdruleEXXcont{} \and
      \ATLLdruleEXXimpI{} \and
      \ATLLdruleEXXimpE{} \and
      \ATLLdruleEXXcomp{} 
    \end{mathpar}
  \end{mdframed}
  \caption{ATLL Logical Inference Rules}
  \label{fig:atll-rules}
\end{figure}
\begin{figure}
  \begin{mdframed}
    \begin{mathpar}
      \ATLLdruleEXXdistParaOne{} \and      
      \ATLLdruleEXXdistParaTwo{} \and
      \ATLLdruleEXXdistSeqOne{} \and
      \ATLLdruleEXXdistSeqTwo{}
    \end{mathpar}
  \end{mdframed}
  \caption{ATLL Attack Tree Axioms}
  \label{fig:atll-attack-tree-axioms}
\end{figure}

The ATLL logical inference rules are defined in
Figure~\ref{fig:atll-rules}.  These rules make up a basic linear logic
for attack trees.  For example, we can prove that sequential and
parallel conjunction are associative, and the latter is symmetric, but
there are a number of facts that these rules do not prove.  They
cannot prove that choice is associative and symmetric, and they cannot
prove the usual distributive laws associated with attack trees with
sequential conjunction \cite{Jhawar:2015}.  To overcome this
limitation we add the attack tree axioms in
Figure~\ref{fig:atll-attack-tree-axioms} to ATLL.  The logical
connective $[[o-o]]$ is linear biimplication which is defined in the
usual way as classical implication.  The reader should note that these
axioms are restricted to attack trees only and not applicable to full
ATLL formulas.  Furthermore, the reader should note that the rule
$\ATLLdruleEXXcompName{}$ is very useful when proving properties of
attack trees, but is actually an admissible rule, meaning, it can be
proven to hold in terms of the other rules in ATLL, but to simplify
the presentation given here it is taken as part of the system.  In
addition, the rule $\ATLLdruleEXXdistParaTwoName{}$ is admissible as
well, because parallel conjunction is symmetric.

The ATLL logical inference rules also make it clear why we separate
hypotheses into two contexts.  The exchange rule, $\text{sym}_\odot$,
allows one to exchange hypotheses in the context $[[P]]$, but notice
that there is no exchange rule for the context $[[I]]$.  Thus, one
should think of the hypotheses in $[[P]]$ as running in parallel and
the hypotheses in $[[I]]$ as running sequentially.  In fact, this
interpretation is reinforced by the elimination rules, $\odot_E$ and
$\rhd_E$, for parallel and sequential conjunction respectively.

Finally, ATLL allows one to prove every equivalence for attack trees
with sequential conjunction given by Jhawar et al.~\cite{Jhawar:2015}
as linear implications.  I conjecture that it should be possible to
define a proof search algorithm for ATLL to automate equational
reasoning on attack trees, but this is future work.  However, using
implication as a means to reason about attack trees opens the door for
different types of reasoning.

Suppose $[[T1]]$ is a larger attack tree, and we wish to know if the
attack tree $[[T2]]$ is contained within $[[T1]]$, that is, $[[T2]]$
is a subattack tree of $[[T1]]$.  Currently, ATLL will not allow us to
prove this, but it can if we add the following rules:
\begin{center}
  \begin{math}
    \begin{array}{lll}
      \ATLLdruleEXXweakS{} & \quad & \ATLLdruleEXXweakP{}
    \end{array}
  \end{math}
\end{center}
These rules allow us to discard parts of a larger attack tree while
constructing proof derivations, but this only works for attack trees
not general ATLL formulas.  Thus, with these rules ATLL becomes
slightly affine.  We can now use these rules to determine if
$[[T1 -o T2]]$ holds, and if it does, then $[[T2]]$ is a
subattack tree of $[[T1]]$.

I now give a couple of examples to illustrate using the system to
prove properties of attack trees.  Consider the two example trees used
to introduce the SPTool for checking equivalence of attack trees by
Krody et al.~\cite{?}:
\[
\begin{array}{lll}
  [[T1]] := [[(B1 (.) (B2 + B3)) > B4]]\\
  [[T2]] := [[((B1 (.) B2) > B4) + ((B1 (.) B3) > B4)]]\\
\end{array}
\]
where $[[Bi]]$ for $1 \leq i \leq 4$ are the base attacks.  The tree
$[[T1]]$ corresponds to the ATM attack tree from Figure~1 of \cite{?},
and $[[T2]]$ corresponds to the canonical tree given in Figure~2 of
\cite{?}. The attack trees $[[T1]]$ and $[[T2]]$ are equivalent if and
only if $[[T1 o-o T2]]$ holds.  First, ATLL attack tree axioms are
biimplications, and hence, we may treat these axioms as equivalences.
For example, suppose $[[.;.;. |- E1 o-o E2]]$ and $[[I;H;P |- E1 > E3]]$
both hold, then we can prove that $[[I;H;P |- E2 > E3]]$ holds:
\[\small
\inferrule* [right=$\ATLLdruleEXXseqEName{}$] {
  [[I;H;P |- E1 > E3]]
  \\
  $$\mprset{flushleft}
  \inferrule* [right=$\ATLLdruleEXXseqEName{}$] {
    $$\mprset{flushleft}
    \inferrule* [right=$\ATLLdruleEXXimpEName{}$] {
      [[.;.; . |- E1 -o E2]]
      \\
      $$\mprset{flushleft}
      \inferrule* [right=] {
        \,
      }{[[E1;.;. |- E1]]}
    }{[[E1;.;. |- E2]]}
    \\
    $$\mprset{flushleft}
    \inferrule* [right=] {
      \,
    }{[[E3;.;. |- E3]]}
  }{[[E1,E3;.; . |- E2 > E3]]}
}{[[I;H;P |- E2 > E3]]}
\]
Thus, using this realization and the ATLL attack tree axioms we can
prove that $[[T1]]$ is equivalent to $[[T2]]$ informally as follows:
\[
\begin{array}{lllll}
  [[T1]]
  & := & [[(B1 (.) (B2 + B3)) > B4]]                       & \text{(definition)}\\
  & [[o-o]] & [[((B1 (.) B2) + (B1 (.) B3)) > B4]]         & (\text{rule }\ATLLdruleEXXdistParaOneName{})\\
  & [[o-o]] & [[((B1 (.) B2) > B4) + ((B1 (.) B3) > B4)]]  & (\text{rule }\ATLLdruleEXXdistSeqTwoName{})\\
  & := & [[T2]]                                            & \text{(definition)}\\
\end{array}
\]
The corresponding formal derivation is quite large, but is easily
constructible using the previous reasoning.
