Now I generalize Bucciarelli and Ehrhard’s indexed linear logic
\cite{BUCCIARELLI:2000} to obtain an extension of ATLL called Indexed ATLL with the
ability to model and reason about attack trees with costs.
\begin{definition}
  \label{def:atrees}
  Suppose $\mathsf{B}$ is a set of base attacks whose elements are
  denoted by $[[b]]$, and $\mathsf{C}$ is a set of costs whose
  elements are denoted by $[[c]]$.  Additionally, let
  $(\mathsf{C},[[(.)op]])$ and $(\mathsf{C},[[+op]])$ be symmetric
  monoids on $\mathsf{C}$, and $(\mathsf{C},[[>op]])$ be a monoid on
  $\mathsf{C}$, where $[[(.)op]]$ and $[[>op]]$ distribute over
  $[[+op]]$.  Then an \textbf{attack tree with costs} is defined by
  the following grammar:
  \[
  \begin{array}{lll}
    [[T]] ::= [[(b,c)]] \mid [[T1 (.) T2]] \mid [[T1 + T2]] \mid [[T1 > T2]]\\
  \end{array}
  \]  
\end{definition}
\begin{figure}\footnotesize
  \begin{mdframed}
    \begin{mathpar}
      \IATLLdruleTXXvar{} \and
      \IATLLdruleTXXpara{} \and
      \IATLLdruleTXXseq{} \and
      \IATLLdruleTXXchoice{}
    \end{mathpar}
  \end{mdframed}
  \caption{Well Formed Attack Trees in Indexed ATLL}
  \label{fig:wf-index-attack-trees}
\end{figure}
Each base attack is annotated with a cost, and every branching node is
associated with a binary operation for computing the cost at that node
using the costs of the left subtree and the right subtree.

I now extend the syntax and rules of ATLL into Indexed ATLL, but due
to space I do not give every rule here, but only the most interesting
ones, because they are similar to ATLL; the entire system is defined
in Appendix~\ref{subsec:the_full_specification_of_indexed_atll}.
\begin{restatable}[]{definition}{iatllSyntax}
  \label{def:ATLL-syntax}
  Suppose $\mathsf{B}$ is a set of base attacks whose elements are
  denoted by $[[b]]$, and $\mathsf{C}$ is a set of costs whose
  elements are denoted by $[[c]]$.  Additionally, let
  $(\mathsf{C},[[(.)op]])$ and $(\mathsf{C},[[+op]])$ be symmetric
  monoids on $\mathsf{C}$, and $(\mathsf{C},[[>op]])$ be a monoid on
  $\mathsf{C}$, where $[[(.)op]]$ and $[[>op]]$ distribute over
  $[[+op]]$.  Furthermore, let $(\mathsf{C}, [[-oop]])$ be a
  preorder on $\mathsf{C}$.  Then the syntax of Indexed ATLL formulas
  and contexts are defined as follows:
  \[
  \begin{array}{rll}
    \text{(formulas)}         & [[E]] ::= [[(b, c)]] \mid [[E1 (.) E1]] \mid [[E1 + E2]] \mid [[E1 > E2]] \mid [[E1 c -o E2]]\\
    \text{(base contexts)}    & [[G]],[[D]] ::= [[.]] \mid [[(b,c)]] \mid [[G,D]]\\
    \text{(general contexts)} & [[I]],[[P]] ::= [[.]] \mid [[(E,c)]] \mid [[G,D]]\\\\
  \end{array}
  \]
\end{restatable}
\vspace{-10px}
As we can see the only major changes to the ATLL syntax is that all
hypotheses are annotated with costs.

The inference rules for well-formed attack trees in Indexed ATLL are
given in Figure~\ref{fig:wf-index-attack-trees}.  The sequent is now
denoted by $[[D |-T c T]]$ where $[[c]]$ is the fully computed cost
of the root of the attack tree $[[T]]$.  Notice that this cost is
computed during the construction of a proof derivation.  This cost
will be used along with the relation on implication to influence the
relationship between attack trees.
\begin{figure}\scriptsize
  \begin{mdframed}
    \begin{mathpar}
      \IATLLdruleEXXvar{} \and
      \IATLLdruleEXXvarCC{} \and
      \IATLLdruleEXXvarC{} \and
      \IATLLdruleEXXseqI{} \and
      \IATLLdruleEXXseqE{} \and      
      \IATLLdruleEXXimpI{} \and
      \IATLLdruleEXXimpE{} 
    \end{mathpar}
  \end{mdframed}
  \caption{Select Logical Inference Rules for Indexed ATLL}
  \label{fig:atl}
\end{figure}

The logical inference rules for Indexed ATLL are given in
Figure~\ref{fig:atl}.  As I mentioned above I only give a select few
of the rules here, but the complete system can be found in
Appendix~\ref{subsec:the_full_specification_of_indexed_atll}.  The
  most interesting rule is the introduction rule for implication,
  $[[-o]]_I$, because the second premise is $[[-oop]]([[c1]],[[c2]])$.

Suppose $[[(T1,c1)]]$ and $[[(T2,c2)]]$ are two attack trees with the
respective costs, and the relation $[[-oop]]([[c1]],[[c2]])$ holds if
and only if $[[c1]] \leq [[c2]]$. If we wanted to show that the attack
tree $[[T1]]$ is equivalent to the attack tree $[[T2]]$, then we would
have to show $[[T1 o-o T2]]$ where $[[o-oop]]([[c1]],[[c2]])$
holds if and only if $[[-oop]]([[c1]],[[c2]])$ and
$[[-oop]]([[c2]],[[c1]])$ hold.  This would require one to prove
$[[o-oop]]([[c1]],[[c2]])$ which would require $[[c1]] = [[c2]]$.
Thus, in this example, attack trees would be equivalent if and only if
they have the same logical structure and the same overall cost.  Thus,
several different types of reasoning on attack trees can be modeled in
Index ATLL depending on the relation chosen for implication.
