The attack tree for assessing the risk of becoming root on a Unix
machine from the introduction is actually written in Lina's definition
language.  However, there is one simplification that was made.  The
data on branching nodes actually are binary functions, but when those
functions are constant we omit their arguments.

We can see from the example that we treat the nodes of the attack tree
as combinators, either $[[leaf l q]]$ or $[[c l q]]$, where $c$ is a
branching node symbol, $[[l]]$ is a label, for example, a string, and
$[[q]]$ is a quantitative expression.  The types of $[[l]]$ and
$[[q]]$ depend on the type of the tree itself, for example, the type
of the tree above is $[[AttackTree String Double]]$, and thus, labels
are strings, and leafs are constant doubles, but the quantitative data
on branching nodes has type $[[Double -> Double -> Double]]$. Thus,
one novelty of Lina is that data at nodes can be higher order, but the
data at branching nodes is always a binary function whose first
argument is the data from the left tree, and the second argument is
the data from the right tree. Hence, making it easier and more precise
to compute costs across the tree.  In the example above, we actually
used the functions $\lambda x.\lambda y.x + y$ and $\lambda x.\lambda
y.\mathsf{min}\,x\,y$ on the sequential composition and choice
branching nodes.  Finally, branching nodes have two additional
arguments, $[[t1]]$ and $[[t2]]$, which are the left and right trees.

Throughout the remainder of this section we give a brief overview of
the preliminary design of Lina's language for defining attack trees.
The following defines the syntax ($[[d]]$ ranges over any double, and
$[[s]]$ ranges over any string):\\\\
\begin{math}
  \small
  \begin{array}{cll}
  \text{(Quantitative Types)} & [[Q]] := \mathsf{Double} \mid [[Q -> Q]]\\
  \text{(Numeric Operators)}  & [[nop]] := [[+]] \mid [[-]] \mid [[*]] \mid [[/]]\\
  \text{(Quantitative Expressions)} & [[q]] := [[x]] \mid [[d]] \mid [[\x.q]] \mid [[q1 q2]] \mid [[q1 nop q2]] \mid \\
  & \,\,\,\,\,\,\,\,\,\,\,\,\,[[min q1 q2]] \mid [[rec q0 of q1 | q2]]\\
  \text{(Label Types)} & [[L]] := \mathsf{String} \mid \mathsf{Double}\\
  \text{(Labels)} & [[l]] := [[s]] \mid [[d]]\\
  \text{(Kinds)} & [[k]] := [[AttackTree L Q]] \mid [[k1 -> k2]]\\
  \text{(Attack Tree Combinators)} & [[c]] := \otimes \mid [[(.)]] \mid [[;]] \mid [[|_|]]\\
  \text{(Attack Trees)} & [[t]] := [[x]] \mid [[\x.t]] \mid [[t1 t2]] \mid [[leaf l q]] \mid [[c l q]]
  \end{array}
\end{math}
\ \newline\newline Typing for this language is straightforward, and to
save space we do not give every rule.  The typing rules for the
quantitative language corresponds to the simply typed
$\lambda$-calculus with doubles, numeric operators, and a recursor in
the spirit of G\"odel's system T, and thus, we omit their rules here,
but denote the typing judgment by $[[D |- q : Q]]$, where $[[D]]$ is a
typing context consisting of pairs $[[x : Q]]$; for the complete set
of rules see
Appendix~\ref{sec:lina's_attack_tree_definition_language}.  Typing
labels is trivial, and we denote the typing judgment by $[[|- l :
    L]]$.  Finally, the following rules defines the kinding rules for
attack trees:
\begin{mathpar}
  \small
  \TLLdruleKXXVar{} \and
  \TLLdruleKXXFun{} \and
  \TLLdruleKXXApp{} \and
  \TLLdruleKXXLeaf{} \and
  \TLLdruleKXXComb{}
\end{mathpar}
Functions over attack trees will allow for the definition of attack
tree schemas that one could use to build up a library of attack
trees. At this point one could speak of evaluating attack trees which
would correspond to normalizing the tree to its combinator form, but
we could also speak about evaluating the data of the tree, but how
this is done is left for future work.

Lina's attack tree definition language technically lives at the type
level.  We can see that the logical layer of each attack tree
corresponds to a linear type.  When attack trees get large it makes
sense to want to restructure the tree to gain new insights, but
existing tools do not support this in such a way that one knows that
the tree obtained after restructuring is semantically equivalent to
the original tree.  Due to the Curry-Howard-Lambek correspondence Lina
will come equipped with a linear type system that can be used to
register semantically valid transformations of attack trees as
programs between linear types, but there is a big hurdle that first
must be crossed.
