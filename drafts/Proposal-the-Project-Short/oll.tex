In order to aid the reader unfamiliar with linear logic and how it and
other logics can be used as a modeling framework I first give a brief
introduction to (ordered) linear logic and how it can be used to model
various structures in computer science.  I assume very little of the
reader, and start with the basics of the specification of logics.
Ordered linear logic (OLL) \cite{Polakow:2001} will be the ongoing
example throughout this section, because it is the logic ATLL is based
on.  If the reader is familiar with linear logic, then they can safely
skip this section.

Every logic presented in this paper is in the form of sequent-style
natural deduction.  This type of formalization begins with the
specifying the syntax of formulas and sequents.  The latter is defined
by a set of inference rules.  The syntax for OLL is as follows:
\[
\begin{array}{crcl}
\text{(formulas)} & [[A]],[[B]],[[C]] & ::= & [[b]] \mid [[Top]] \mid [[A (.) B]]
  \mid [[A > B]] \mid [[A x B]] \mid [[A -o B]]\\
\text{(ordered contexts)} & [[D]] & ::= & [[.]] \mid [[A]] \mid [[G,D]]\\
\text{(unordered contexts)} & [[G]] & ::= & [[.]] \mid [[A]] \mid [[I,P]]\\
\end{array}
\]
Formulas of OLL consist of atomic formulas denoted by $[[b]]$, a
symmetric tensor product $[[A (.) B]]$, its unit $[[Top]]$, a
non-symmetric tensor product $[[A > B]]$, non-linear conjunction $[[A
    x B]]$, and linear implication $[[A -o B]]$.  Note that the
symmetric tensor product is to linear implication as conjunction is to
implication in classical logic.  The other forms of conjunction are
additional.  Contexts are lists of hypothesis where $[[D]]$ is a list
of ordered hypothesis, and $[[G]]$ is a list of unordered hypothesis.
The former are associated with the non-symmetric tensor product, and
the latter with the symmetric tensor product.  We denote the empty
context by $[[.]]$, and appending of contexts $[[G1]]$ and $[[G2]]$ by
$[[G1,G2]]$.

Linear logic has a resource interpretation, and the locus of this
interpretation is linear implication.  In linear logic hypotheses
cannot be freely duplicated or contracted.  That is, $[[A -o (A (.)
    A)]]$ does not hold, nor does $[[(A (.) A) -o A]]$.  In addition,
hypotheses cannot be freely created, that is, $[[A -o Top]]$ does not
hold in general.  Each of these examples either uses an hypotheses
more than once, or does not use an hypotheses at all.  This implies
that every hypotheses must be used exactly once, and this property is
called the linearity property.

The linearity property produces a resource interpretation of formulas.
Linear implication $[[A -o B]]$ can be interpreted as, consume the
resource $[[A]]$, and then produce $[[B]]$.  Under this interpretation
the linearity property can be read as, every resource must be consumed
exactly once.  The symmetric tensor product $[[A (.) B]]$ can be read
as, consume both the resources $[[A]]$ and $[[B]]$.  Similarly, the
non-symmetric tensor product $[[A > B]]$ can be read as, consume the
resource $[[A]]$, and then consume the resource $[[B]]$. Now
non-linear conjunction $[[A x B]]$ is ordinary conjunction from
classical and intuitionistic logic, and hence, formulas like $[[(A x
    A) -o A]]$ and $[[A -o (A x A)]]$ both hold.  Similarly, $[[(A x
    B) -o A]]$ and $[[(A x B) -o B]]$ hold.  Thus, one should
interpret this as a way of pairing resources up.  I give an example of
how to use the resources interpretation to model a simple state-based
system at the end of this section.

Sequents of OLL are denoted by $[[D;G |- A]]$.  We say that the
sequent $[[D;G |- A]]$ holds if and only if given the hypothesis in
$[[D]]$ and $[[G]]$ one can construct a proof of the formula $[[A]]$
using the following inference rules:
\begin{mdframed}\scriptsize
  \begin{mathpar}
    \OLLdrulevar{} \and
    \OLLdrulevarC{} \and
    \OLLdruleTop{} \and
    \OLLdruleconjI{} \and
    \OLLdruleconjEOne{} \and
    \OLLdruleconjETwo{} \and
    \OLLdruleparaI{} \and
    \OLLdruleparaE{} \and
    \OLLdruleseqI{} \and
    \OLLdruleseqE{} \and
    \OLLdruleex{} \and
    \OLLdruleimpI{} \and
    \OLLdruleimpE{}
  \end{mathpar}
\end{mdframed}
An inference rule should be read from top to bottom as an implication.
The sequents on top of the line are the premises, and the sequent
below the line is the conclusion.  For example, one should read the
rule $\OLLdruleparaIName{}$ as if $[[D1;G1 |- A]]$ and $[[D2;G2 |-
    B]]$ both hold, then $[[D1,D2;G1,G2 |- A (.) B]]$ holds.  An
inference rule with no premises are called axioms.  For example, the
rules $\OLLdrulevarName{}$ and $\OLLdrulevarCName{}$ are both axioms.

Inference rules are read from top to bottom, but applied from bottom
to top.  A sequent, $[[D;G |- A]]$, holds if and only if there is a
series of inference rules that can be composed into a derivation.
Every derivation is a tree where the root is the sequent we wish to
prove, and then rules are composed to form a tree.  A rule can be
composed with another if the formers conclusion matches a premise of
the latter.  If every branch of this tree reaches an axiom, then it is
a valid proof, but if any branch gets stuck, then the sequent does not
hold. The following is a proof that the symmetric tensor is indeed
symmetric:
\vspace{-10px}
\begin{center}\footnotesize
  \begin{math}
    \inferrule* [right=$\OLLdruleimpIName{}$] {
      $$\mprset{flushleft}
      \inferrule* [right=$\OLLdruleparaEName{}$] {
        $$\mprset{flushleft}
        \inferrule* [right=$\OLLdrulevarName{}$] {
          \,
        }{[[. ; A (.) B |- A (.) B]]}
        \\
        $$\mprset{flushleft}
        \inferrule* [right=$\OLLdruleexName{}$] {
          $$\mprset{flushleft}
          \inferrule* [right=$\OLLdruleparaIName{}$] {
            $$\mprset{flushleft}
            \inferrule* [right=$\OLLdrulevarName{}$] {
              \,
            }{[[. ; B |- B]]}
            \\
            $$\mprset{flushleft}
            \inferrule* [right=$\OLLdrulevarName{}$] {
              \,
            }{[[. ; A |- A]]}
          }{[[. ; B, A |- B (.) A]]}
        }{[[. ; A, B |- B (.) A]]}
      }{[[. ; A (.) B |- B (.) A]]}
    }{[[. ; . |- (A (.) B) -o (B (.) A)]]}
  \end{math}
\end{center}
We call the root of the tree the goal of the proof, and as we
construct a derivation this goal is refined into potentially several
new subgoals.  This is called goal directed proof.  As we can see the
previous derivation is a valid proof of the goal $[[. ; . |- (A (.) B)
    -o (B (.) A)]]$.

The following derivation is an invalid proof that the non-symmetric
tensor product is symmetric:
\vspace{-10px}
\begin{center} \footnotesize
  \begin{math}    
    \inferrule* [right=$\OLLdruleimpIName{}$] {
      $$\mprset{flushleft}
      \inferrule* [right=$\OLLdruleseqEName{}$] {
        $$\mprset{flushleft}
        \inferrule* [right=$\OLLdrulevarName{}$] {
          \,
        }{[[. ; A > B |- A > B]]}
        \\
          [[A, B ; . |- B > A]]
      }{[[. ; A > B |- B > A]]}
    }{[[. ; . |- (A > B) -o (B > A)]]}
  \end{math}
\end{center}
At this point we have refined our goal to the subgoal $[[A, B ; . |- B
    > A]]$, but this is impossible to prove, because the exchange rule
$\OLLdruleexName{}$ cannot be be applied to the ordered context to
commute $[[A]]$ and $[[B]]$.

Building proofs from the inference rules given above can often be
tedious and long.  To make this process easier it is often necessary
to introduce new rules that can be proven to be valid in terms of the
inference rules already given.  Rules introduced in this way are
called derivable inference rules.  One derivable rule we will make use
of is the following:
\[ \footnotesize
\OLLdrulecomp{}
\]
Proving this rule is derivable amounts to creating a derivation whose
branches terminate at either an axiom or the premises
$[[D2;G2 |- A -o B]]$ or $[[D1;G1 |- B -o C]]$.  The proof is as follows:
\vspace{-20px}
\[
\inferrule* [right=$\OLLdruleimpIName{}$] {
  $$\mprset{flushleft}
  \inferrule* [right=$\OLLdruleimpEName{}$] {
    [[D1;G1 |- B -o C]]
    \\
    $$\mprset{flushleft}
    \inferrule* [right=$\OLLdruleimpEName{}$] {
      [[D2;G2 |- A -o B]]
      \\
        $$\mprset{flushleft}
      \inferrule* [right=$\OLLdrulevarName{}$] {
        \,
      }{[[.;A |- A]]}
    }{[[D2;G2,A |- B]]}          
  }{[[D1,D2;G1,G2, A |- C]]}
}{[[D1,D2;G1,G2 |- A -o C]]}
\]

At this point I have introduced the basics of sequent-style natural
deduction, but nothing I have said up to now has been specific to
ordered linear logic.  We call the this logic ordered, because it
contains the non-symmetric tensor product.  Its non-symmetry is
enforced by the separation of hypotheses.  What makes a logic linear?

Linear logic was first introduced by Girard~\cite{Girard:1987} in the
late eighties.  A logic is linear if every hypothesis is used exactly
once.  Thus, no hypothesis can be duplicated or removed at will, that
is, the structural rules for contraction (duplication) and weakening
(removal) must be banned from the logic.  This property makes linear
logic particularly suited for reasoning about state-based systems.

We enforce this property by restricting the inference rules in two
ways. First, consider the identity axioms for OLL:
\[
\begin{array}{lll}
  \OLLdrulevar{} & \quad\quad\quad\quad & \OLLdrulevarC{}
\end{array}
\]
The contexts are restricted to containing at most the hypothesis the
axiom is proving holds.  The other restriction is when applying the
inference rules during the construction of derivations one is not
allowed to copy or introduce new hypothesis across premises.  This is
why we take great care at separating the contexts in the definition of
the inference rules.

Modeling structures in logic corresponds to interpreting the
structures as formulas, and then reasoning about the structures
corresponds to proving implications between the interpretations.  As
we will see, attack trees will be modeled as formulas of linear logic,
and then we will prove several properties of attack trees by proving
implications between them, but before moving on to ATLL I give an
example showing how to model a simple state-based system in linear
logic.

There is a common example of problem solving used in artificial
intelligence called the block world scenario; I learned of this
example from Power and Webster's work on embedding linear logic in Coq
\cite{nuimeprn6461}.  Suppose there is a finite set of blocks and a
robot arm that can be used to move blocks around.  For example, the
arm might break a single stack of blocks into two stacks.  This setup
can be modeled in linear logic using a few predicates on blocks.  The
state of the system can be modeled using the following predicates:
\begin{center}
  \begin{tabular}{|rll|}
    \hline
    $[[on x y]]$  & : & block $x$ is on top of block $y$\\
    $[[table x]]$ & : & block $x$ is on the table (i.e. no block underneath of $x$)\\
    $[[clear x]]$ & : & there is no block on top of $x$\\
    $[[holds x]]$ & : & the robot arm is holding block $x$\\
    $[[empty]]$  & : & an atomic formula indicating the robot arm is not holding a block.\\
    \hline
  \end{tabular}
\end{center}

Next we must characterize the pre- and post-states of the valid
actions one can perform during the game.  These are defined as a set
of linear logic axioms, where the hypotheses are the pre-state and the
conclusion is the post-state.  In this setting linear implication,
$[[A -o B]]$, is being interpreted as, if before the action executes,
$[[A]]$ holds, then after the action executes, $[[B]]$ holds. The
following set of inference rules axiomatize the actions:
\begin{mdframed}[innertopmargin=-8px]\small
  \begin{mathpar}
    \OLLdruleget{} \and \OLLdruleput{}
  \end{mathpar}
\end{mdframed}
The $\OLLdrulegetName{}$ axiom states that if the robot arm was empty
and there was no block on top of $[[x]]$, then after the action
executes the robot arm will be holding the block $[[x]]$, and either
the block $[[x]]$ had been on the table, or it was on another block
$[[y]]$ which we now declare has no block on top of it.  The
$\OLLdruleputName{}$ axiom states that if the robot arm was holding
the block $[[x]]$, then after the action executes the robot arm is
empty, the block $[[x]]$ has no block on top of it, and either the
block $[[x]]$ is sitting on the table, or it is sitting on the block
$[[y]]$ which was clear, and now we declare that $[[x]]$ is on top of
$[[y]]$.

We can see that the definition of the two axioms use the tensor
product to indicate when two processes run in parallel.  For example,
in $\OLLdrulegetName{}$ the robot arm is holding the block $[[x]]$ in
parallel to checking the configuration of the table and other blocks.
Then we use non-linear conjunction to model choices.  Furthermore, the
identity to the tensor product can be used to model processes that may
consume a resource, but does not produce any new resources.  For
example, the formula $[[(table x) -o Top]]$ consumes the resource
corresponding to the block $[[x]]$ being on the table, but then does
not update the state.

At this point we can use our logic to prove properties about the block
world scenario, and we can even derive new inference rules.  For
example, we could derive a new inference rule that swaps the two
blocks on top of a stack of at least two, or that performs a get when
there is a block underneath the block the arm will pick up.  We will
use these insights throughout the remainder of this paper to model
attack trees.
