This appendix provides the definitions of both categories in general,
and, in particular, symmetric monoidal closed categories.  We begin
with the definition of a category:
\begin{definition}
  \label{def:category}
  A \textbf{category}, $\cat{C}$, consists of the following data:
  \begin{itemize}
  \item A set of objects $\cat{C}_0$, each denoted by $A$, $B$, $C$, etc.
  \item A set of morphisms $\cat{C}_1$, each denoted by $f$, $g$, $h$, etc.
  \item Two functions $\mathsf{src}$, the source of a morphism, and
    $\mathsf{tar}$, the target of a morphism, from morphisms to
    objects.  If $\mathsf{src}(f) = A$ and $\mathsf{tar}(f) = B$, then
    we write $f : A \to B$.
  \item Given two morphisms $f : A \to B$ and $g : B \to C$, then the
    morphism $f;g : A \to C$, called the composition of $f$ and $g$,
    must exist.
  \item For every object $A \in \cat{C}_0$, the there must exist a
    morphism $\id_A : A \to A$ called the identity morphism on $A$.

  \item The following axioms must hold:
    \begin{itemize}
    \item (Identities) For any $f : A \to B$, $f;\id_B = f = \id_A;f$.
    \item (Associativity) For any $f : A \to B$, $g : B \to C$, and $h
      : C \to D$, $(f;g);h = f;(g;h)$.
    \end{itemize}
  \end{itemize}
\end{definition}

Categories are by definition very abstract, and it is due to this that
makes them so applicable.  The usual example of a category is the
category whose objects are all sets, and whose morphisms are
set-theoretic functions.  Clearly, composition and identities exist,
and satisfy the axioms of a category.  A second example is preordered
sets, $(A , \leq)$, where the objects are elements of $A$ and a
morphism $f : a \to b$ for elements $a, b \in A$ exists iff $a \leq
b$. Reflexivity yields identities, and transitivity yields
composition.  

Symmetric monoidal categories pair categories with a commutative
monoid like structure called the tensor product.
\begin{definition}
  \label{def:monoidal-category}
  A \textbf{symmetric monoidal category (SMC)} is a category, $\cat{M}$,
  with the following data:
  \begin{itemize}
  \item An object $I$ of $\cat{M}$,
  \item A bi-functor $\otimes : \cat{M} \times \cat{M} \to \cat{M}$,
  \item The following natural isomorphisms:
    \[
    \begin{array}{lll}
      \lambda_A : I \otimes A \to A\\
      \rho_A : A \otimes I \to A\\      
      \alpha_{A,B,C} : (A \otimes B) \otimes C \to A \otimes (B \otimes C)\\
    \end{array}
    \]
  \item A symmetry natural transformation:
    \[
    \beta_{A,B} : A \otimes B \to B \otimes A
    \]
  \item Subject to the following coherence diagrams:
    \begin{mathpar}
      \bfig
      \vSquares|ammmmma|/->`->```->``<-/[
        ((A \otimes B) \otimes C) \otimes D`
        (A \otimes (B \otimes C)) \otimes D`
        (A \otimes B) \otimes (C \otimes D)``
        A \otimes (B \otimes (C \otimes D))`
        A \otimes ((B \otimes C) \otimes D);
        \alpha_{A,B,C} \otimes \id_D`
        \alpha_{A \otimes B,C,D}```
        \alpha_{A,B,C \otimes D}``
        \id_A \otimes \alpha_{B,C,D}]      
      
      \morphism(1433,1000)|m|<0,-1000>[
        (A \otimes (B \otimes C)) \otimes D`
        A \otimes ((B \otimes C) \otimes D);
        \alpha_{A,B \otimes C,D}]
      \efig
      \and
      \bfig
      \hSquares|aammmaa|/->`->`->``->`->`->/[
        (A \otimes B) \otimes C`
        A \otimes (B \otimes C)`
        (B \otimes C) \otimes A`
        (B \otimes A) \otimes C`
        B \otimes (A \otimes C)`
        B \otimes (C \otimes A);
        \alpha_{A,B,C}`
        \beta_{A,B \otimes C}`
        \beta_{A,B} \otimes \id_C``
        \alpha_{B,C,A}`
        \alpha_{B,A,C}`
        \id_B \otimes \beta_{A,C}]
      \efig      
    \end{mathpar}
    \begin{mathpar}
      \bfig
      \Vtriangle[
        (A \otimes I) \otimes B`
        A \otimes (I \otimes B)`
        A \otimes B;
        \alpha_{A,I,B}`
        \rho_{A}`
        \lambda_{B}]
      \efig
      \and
      \bfig
      \btriangle[
        A \otimes B`
        B \otimes A`
        A \otimes B;
        \beta_{A,B}`
        \id_{A \otimes B}`
        \beta_{B,A}]
      \efig
      \and
      \bfig
      \Vtriangle[
        I \otimes A`
        A \otimes I`
        A;
        \beta_{I,A}`
        \lambda_A`
        \rho_A]
      \efig
    \end{mathpar}    
  \end{itemize}
\end{definition}


%% \begin{definition}
%%   \label{def:SMCC}
%%   A \textbf{symmetric monoidal closed category (SMCC)} is a symmetric
%%   monoidal category, $(\cat{M},I,\otimes)$, such that, for any object
%%   $B$ of $\cat{M}$, the functor $- \otimes B : \cat{M} \to \cat{M}$
%%   has a specified right adjoint.  Hence, for any objects $A$ and $C$
%%   of $\cat{M}$ there is an object $A \limp B$ of $\cat{M}$ and a
%%   natural bijection:
%%   \[
%%   \Hom{\cat{M}}{A \otimes B}{C} \cong \Hom{\cat{M}}{A}{B \limp C}
%%   \]
%% \end{definition}
