\documentclass{llncs}

\usepackage{amssymb,amsmath}
\usepackage{cmll}
\usepackage{textcomp}
\usepackage{stmaryrd}
\usepackage{todonotes}
\usepackage{mathpartir}
\usepackage{hyperref}
\usepackage[barr]{xy}

% Commands that are useful for writing about type theory and programming language design.
%% \newcommand{\case}[4]{\text{case}\ #1\ \text{of}\ #2\text{.}#3\text{,}#2\text{.}#4}
\newcommand{\interp}[1]{\llbracket #1 \rrbracket}
\newcommand{\normto}[0]{\rightsquigarrow^{!}}
\newcommand{\join}[0]{\downarrow}
\newcommand{\redto}[0]{\rightsquigarrow}
\newcommand{\nat}[0]{\mathbb{N}}
\newcommand{\fun}[2]{\lambda #1.#2}
\newcommand{\CRI}[0]{\text{CR-Norm}}
\newcommand{\CRII}[0]{\text{CR-Pres}}
\newcommand{\CRIII}[0]{\text{CR-Prog}}
\newcommand{\subexp}[0]{\sqsubseteq}
%% Must include \usepackage{mathrsfs} for this to work.
\newcommand{\powerset}[0]{\mathscr{P}}

\date{}

% Cat commands.
\newcommand{\cat}[1]{\mathcal{#1}}
\newcommand{\catop}[1]{\cat{#1}^{\mathsf{op}}}
\newcommand{\Hom}[3]{\mathsf{Hom}_{\cat{#1}}(#2,#3)}
\newcommand{\limp}[0]{\multimap}
\newcommand{\dial}[0]{\mathsf{Dial_2}(\mathsf{Sets})}
\newcommand{\dialSets}[1]{\mathsf{Dial_{#1}}(\mathsf{Sets})}
\newcommand{\dcSets}[1]{\mathsf{DC_{#1}}(\mathsf{Sets})}
\newcommand{\sets}[0]{\mathsf{Sets}}
\newcommand{\obj}[1]{\mathsf{Obj}(#1)}
\newcommand{\mor}[1]{\mathsf{Mor(#1)}}
\newcommand{\id}[0]{\mathsf{id}}
\newcommand{\lett}[0]{\mathsf{let}\,}
\newcommand{\inn}[0]{\,\mathsf{in}\,}
\newcommand{\cur}[1]{\mathsf{cur}(#1)}
\newcommand{\curi}[1]{\mathsf{cur}^{-1}(#1)}

\begin{document}

\title{Proposing a New Foundation of Attack Trees in Monoidal Categories}

\author{Harley Eades III}
\institute{Computer and Information Sciences, Augusta University, Augusta, GA, \email{heades@augusta.edu}}

\maketitle 

\begin{abstract}
  TODO
\end{abstract}

\section{Introduction}
\label{sec:introduction}

What do propositional logic, multisets, directed acyclic graphs,
source sink graphs, Petri nets, and Markov processes all have in
common?  They are all mathematical models of attack trees \cite{??},
but even more than that, they can all be modeled in some form of a
symmetric monoidal category\footnote{We provide a proof that the
  category of source sink graphs is monoidal in
  Appendix~\ref{sec:source_sink_graphs_are_symmetric_monoidal}.}
\cite{?} -- for the definition of a symmetric monoidal category see
Appendix~\ref{sec:symmetric_monoidal_categories}.  Taking things a
little bit further, monoidal categories have a tight correspondence
with linear logic through the beautiful Curry-Howard-Lambek
correspondence \cite{?}.  This correspondence states that objects of a
monoidal category correspond to the formulas of linear logic and the
morphisms correspond to proofs of valid sequents of the logic.  I
propose that attack trees -- in many different flavors -- be modeled
as objects in monoidal categories, and hence, as formulas of linear
logic.

The Curry-Howard-Lambek correspondence is a three way relationship:
\begin{center}
  \setlength{\tabcolsep}{7pt}
  \begin{tabular}{ccccc}
    Categories & $\iff$ & Logic    & $\iff$   & Functional Programming\\
    Objects    & $\iff$ & Formulas & $\iff$   & Types    \\
    Morphisms  & $\iff$ & Proofs   & $\iff$   & Programs 
  \end{tabular}
\end{center}
By modeling attack trees in monoidal categories we obtain a sound
mathematical model, a logic for reasoning about attack trees, and the
means of constructing a functional programming language for defining
attack trees (as types), and constructing semantically valid
transformations (as programs) of attack trees.

Linear logic was first proposed by Girard \cite{?} and was quickly
realized to be a theory of resources.  In linear logic, every
hypothesis must be used exactly once.  Thus, formulas like $A \otimes
A$ and $A$ are not logically equivalent -- here $\otimes$ is linear
conjunction.  This resource perspective of linear logic has been very
fruitful in computer science.  It has lead to linear logic as being a
logical foundation of concurrency \cite{?} where formulas may be
considered as processes.  This perspective fits modeling attack trees
perfectly, because they essentially correspond to concurrent
processes.

Girard's genius behind linear logic was that he isolated the
structural rules -- weakening and contraction -- by treating them as
an effect and putting them inside a comonad called the of-course
exponential denoted $!A$.  In fact, $!A \otimes !A$ is logically
equivalent to $!A$, and thus, by staying in the comonad we become
propositional.  This implies that a modal of attack trees in linear
logic also provides a model of attack trees in propositional logic,
and a combination of the two.  It is possible to have the best of both
worlds.

In this short paper I introduce a newly funded research
project\footnote{This material is based upon work supported by the
  National Science Foundation CRII CISE Research Initiation grant,
  ``CRII:SHF: A New Foundation for Attack Trees Based on Monoidal
  Categories``, under Grant No. 1565557.}  investigating founding
attack trees in monoidal categories, and through the
Curry-Howard-Lambek correspondence deriving a new domain-specific
functional programming language called Lina for Linear Threat
Analysis.  We begin by defining the style of attack trees we will
study here in Section~\ref{sec:attack_trees}, then we give a semantics
of attack trees in a model of full intuitionistic linear logic in
Section~\ref{sec:concrete_semantics_of_attack_trees_in_dialectica_spaces},
we discuss how the semantics may be further abstracted in
Section~\ref{sec:abstract_semantics_of_attack_trees_in_monoidal_categories},
and finally, we discuss the design of Lina in
Section~\ref{sec:lina:_a_domain_specific_pl_for_threat_analysis}.
% section introduction (end)

\section{Attack Trees}
\label{sec:attack_trees}

% section attack_trees (end)

\section{Concrete Semantics of Attack Trees in Dialectica Spaces}
\label{sec:concrete_semantics_of_attack_trees_in_dialectica_spaces}

% section concrete_semantics_of_attack_trees_in_dialectica_spaces (end)

\section{Abstract Semantics of Attack Trees in Monoidal Categories}
\label{sec:abstract_semantics_of_attack_trees_in_monoidal_categories}

% section abstract_semantics_of_attack_trees_in_monoidal_categories (end)

\section{Lina: A Domain Specific PL for Threat Analysis}
\label{sec:lina:_a_domain_specific_pl_for_threat_analysis}

% section lina:_a_domain_specific_pl_for_threat_analysis (end)

\section{Conclusion and Future Work}
\label{sec:conclusion_and_future_work}

% section conclusion_and_future_work (end)


\bibliographystyle{plain}
\bibliography{ref}

\appendix

\section{Symmetric Monoidal Categories}
\label{sec:symmetric_monoidal_categories}
This appendix provides the definitions of both categories in general,
and, in particular, symmetric monoidal closed categories.  We begin
with the definition of a category:
\begin{definition}
  \label{def:category}
  A \textbf{category}, $\cat{C}$, consists of the following data:
  \begin{itemize}
  \item A set of objects $\cat{C}_0$, each denoted by $A$, $B$, $C$, etc.
  \item A set of morphisms $\cat{C}_1$, each denoted by $f$, $g$, $h$, etc.
  \item Two functions $\mathsf{src}$, the source of a morphism, and
    $\mathsf{tar}$, the target of a morphism, from morphisms to
    objects.  If $\mathsf{src}(f) = A$ and $\mathsf{tar}(f) = B$, then
    we write $f : A \to B$.
  \item Given two morphisms $f : A \to B$ and $g : B \to C$, then the
    morphism $f;g : A \to C$, called the composition of $f$ and $g$,
    must exist.
  \item For every object $A \in \cat{C}_0$, the there must exist a
    morphism $\id_A : A \to A$ called the identity morphism on $A$.

  \item The following axioms must hold:
    \begin{itemize}
    \item (Identities) For any $f : A \to B$, $f;\id_B = f = \id_A;f$.
    \item (Associativity) For any $f : A \to B$, $g : B \to C$, and $h
      : C \to D$, $(f;g);h = f;(g;h)$.
    \end{itemize}
  \end{itemize}
\end{definition}

Categories are by definition very abstract, and it is due to this that
makes them so applicable.  The usual example of a category is the
category whose objects are all sets, and whose morphisms are
set-theoretic functions.  Clearly, composition and identities exist,
and satisfy the axioms of a category.  A second example is preordered
sets, $(A , \leq)$, where the objects are elements of $A$ and a
morphism $f : a \to b$ for elements $a, b \in A$ exists iff $a \leq
b$. Reflexivity yields identities, and transitivity yields
composition.  

Symmetric monoidal categories pair categories with a commutative
monoid like structure called the tensor product.
\begin{definition}
  \label{def:monoidal-category}
  A \textbf{symmetric monoidal category (SMC)} is a category, $\cat{M}$,
  with the following data:
  \begin{itemize}
  \item An object $I$ of $\cat{M}$,
  \item A bi-functor $\otimes : \cat{M} \times \cat{M} \to \cat{M}$,
  \item The following natural isomorphisms:
    \[
    \begin{array}{lll}
      \lambda_A : I \otimes A \to A\\
      \rho_A : A \otimes I \to A\\      
      \alpha_{A,B,C} : (A \otimes B) \otimes C \to A \otimes (B \otimes C)\\
    \end{array}
    \]
  \item A symmetry natural transformation:
    \[
    \beta_{A,B} : A \otimes B \to B \otimes A
    \]
  \item Subject to the following coherence diagrams:\\
    \begin{math}
      \begin{array}{l}
        \bfig
        \square|amma|/->`->``/<1200,500>[
          ((A \otimes B) \otimes C) \otimes D`
          (A \otimes (B \otimes C)) \otimes D`
          (A \otimes B) \otimes (C \otimes D)`;
          \alpha_{A,B,C} \otimes \id_D`
          \alpha_{A \otimes B,C,D}g``]

        \square(0,-500)|amma|/`->``<-/<1200,500>[
          (A \otimes B) \otimes (C \otimes D)``
          A \otimes (B \otimes (C \otimes D))`
          A \otimes ((B \otimes C) \otimes D);`
          \alpha_{A,B,C \otimes D}``
          \id_A \otimes \alpha_{B,C,D}]       
      
      \morphism(1200,500)|m|<0,-1000>[
         (A \otimes (B \otimes C)) \otimes D`
         A \otimes ((B \otimes C) \otimes D);
         \alpha_{A,B \otimes C,D}]
      \efig
      \\
      \bfig
      \vSquares|ammmmma|/->`->`->``->`->`->/[
        (A \otimes B) \otimes C`
        (B \otimes A) \otimes C`
        A \otimes (B \otimes C)`
        B \otimes (A \otimes C)`
        (B \otimes C) \otimes A`
        B \otimes (C \otimes A);
        \beta_{A,B} \otimes \id_C`
        \alpha_{A,B,C}`
        \alpha_{B,A,C}``
        \beta_{A,B \otimes C}`
        \id_B \otimes \beta_{A,C}`
        \alpha_{B,C,A}]
      \efig\\      
        \end{array}
      \end{math}

    \begin{mathpar}
      \bfig
      \Vtriangle[
        (A \otimes I) \otimes B`
        A \otimes (I \otimes B)`
        A \otimes B;
        \alpha_{A,I,B}`
        \rho_{A}`
        \lambda_{B}]
      \efig
      \and
      \bfig
      \btriangle[
        A \otimes B`
        B \otimes A`
        A \otimes B;
        \beta_{A,B}`
        \id_{A \otimes B}`
        \beta_{B,A}]
      \efig
      \and
      \bfig
      \Vtriangle[
        I \otimes A`
        A \otimes I`
        A;
        \beta_{I,A}`
        \lambda_A`
        \rho_A]
      \efig
    \end{mathpar}    
  \end{itemize}
\end{definition}

Monoidal categories posses additional structure, and hence, ordinary
functors are not enough, thus, the notion must also be extended.
\begin{definition}
  \label{def:MCFUN}
  Suppose we are given two monoidal categories
  $(\cat{M}_1,\top_1,\otimes_1,\alpha_1,\lambda_1,\rho_1)$ and
  $(\cat{M}_2,\top_2,\otimes_2,\alpha_2,\lambda_2,\rho_2)$.  Then a
  \textbf{monoidal functor} is a functor $F : \cat{M}_1 \mto
  \cat{M}_2$, a map $m_{\top_1} : \top_2 \mto F\top_1$ and a natural transformation
  $m_{A,B} : FA \otimes_2 FB \mto F(A \otimes_1 B)$ subject to the
  following coherence conditions:
  \begin{mathpar}
    \footnotesize
    \bfig
    \vSquares|ammmmma|/->`->`->``->`->`->/[
      (FA \otimes_2 FB) \otimes_2 FC`
      FA \otimes_2 (FB \otimes_2 FC)`
      F(A \otimes_1 B) \otimes_2 FC`
      FA \otimes_2 F(B \otimes_1 C)`
      F((A \otimes_1 B) \otimes_1 C)`
      F(A \otimes_1 (B \otimes_1 C));
      {\alpha_2}_{FA,FB,FC}`
      m_{A,B} \otimes \id_{FC}`
      \id_{FA} \otimes m_{B,C}``
      m_{A \otimes_1 B,C}`
      m_{A,B \otimes_1 C}`
      F{\alpha_1}_{A,B,C}]
    \efig
    \end{mathpar}
  \begin{mathpar}
    \bfig
    \square|amma|/->`->`<-`->/<1000,500>[
      \top_2 \otimes_2 FA`
      FA`
      F\top_1 \otimes_2 FA`
      F(\top_1 \otimes_1 A);
      {\lambda_2}_{FA}`
      m_{\top_1} \otimes \id_{FA}`
      F{\lambda_1}_{A}`
      m_{\top_1,A}]
    \efig
    \and
    \bfig
    \square|amma|/->`->`<-`->/<1000,500>[
      FA \otimes_2 \top_2`
      FA`
      FA \otimes_2 F\top_1`
      F(A \otimes_1 \top_1);
      {\rho_2}_{FA}`
      \id_{FA} \otimes m_{\top_1}`
      F{\rho_1}_{A}`
      m_{A,\top_1}]
    \efig
    \end{mathpar}
\end{definition}

% section symmetric_monoidal_categories (end)


\section{Source Sink Graphs are Symmetric Monoidal}
\label{sec:source_sink_graphs_are_symmetric_monoidal}

% section source_sink_graphs_are_symmetric_monoidal (end)


\end{document}

%%% Local Variables: 
%%% mode: latex
%%% TeX-master: t
%%% End: 
