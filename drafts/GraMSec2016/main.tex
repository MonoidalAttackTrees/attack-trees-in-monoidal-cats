\documentclass{llncs}

\usepackage{amssymb,amsmath}
\usepackage{cmll}
\usepackage{stmaryrd}
\usepackage{todonotes}
\usepackage{mathpartir}
\usepackage{hyperref}
\usepackage[barr]{xy}

% Commands that are useful for writing about type theory and programming language design.
%% \newcommand{\case}[4]{\text{case}\ #1\ \text{of}\ #2\text{.}#3\text{,}#2\text{.}#4}
\newcommand{\interp}[1]{\llbracket #1 \rrbracket}
\newcommand{\normto}[0]{\rightsquigarrow^{!}}
\newcommand{\join}[0]{\downarrow}
\newcommand{\redto}[0]{\rightsquigarrow}
\newcommand{\nat}[0]{\mathbb{N}}
\newcommand{\fun}[2]{\lambda #1.#2}
\newcommand{\CRI}[0]{\text{CR-Norm}}
\newcommand{\CRII}[0]{\text{CR-Pres}}
\newcommand{\CRIII}[0]{\text{CR-Prog}}
\newcommand{\subexp}[0]{\sqsubseteq}
%% Must include \usepackage{mathrsfs} for this to work.
\newcommand{\powerset}[0]{\mathscr{P}}

\date{}

% Cat commands.
\newcommand{\cat}[1]{\mathcal{#1}}
\newcommand{\catop}[1]{\cat{#1}^{\mathsf{op}}}
\newcommand{\homs}[3]{\mathsf{Hom}_{\cat{#1}}(#2,#3)}
\newcommand{\limp}[0]{\multimap}
\newcommand{\dial}[0]{\mathsf{Dial_2}(\mathsf{Sets})}
\newcommand{\dialSets}[1]{\mathsf{Dial_{#1}}(\mathsf{Sets})}
\newcommand{\dcSets}[1]{\mathsf{DC_{#1}}(\mathsf{Sets})}
\newcommand{\sets}[0]{\mathsf{Sets}}
\newcommand{\obj}[1]{\mathsf{Obj}(#1)}
\newcommand{\mor}[1]{\mathsf{Mor(#1)}}
\newcommand{\id}[0]{\mathsf{id}}
\newcommand{\lett}[0]{\mathsf{let}\,}
\newcommand{\inn}[0]{\,\mathsf{in}\,}
\newcommand{\cur}[1]{\mathsf{cur}(#1)}
\newcommand{\curi}[1]{\mathsf{cur}^{-1}(#1)}

\begin{document}

\title{Proposing a New Foundation of Attack Trees in Monoidal Categories}

\author{Harley Eades III}
\institute{Computer and Information Sciences, Augusta University, Augusta, GA, \email{heades@augusta.edu}}

\maketitle 

\begin{abstract}
  TODO
\end{abstract}

\section{Introduction}
\label{sec:introduction}

What do propositional logic, multisets, directed acyclic graphs,
source sink graphs, Petri nets, and Markov processes all have in
common?  They are all mathematical models of attack trees \cite{??},
but even more than that, they can all be modeled in some form of a
symmetric monoidal category\footnote{We provide a proof that the
  category of source sink graphs is monoidal in
  Appendix~\ref{sec:source_sink_graphs_are_symmetric_monoidal}.}
\cite{?} -- for the definition of a symmetric monoidal category see
Appendix~\ref{sec:symmetric_monoidal_categories}.  Taking things a
little bit further, monoidal categories have a tight correspondence
with linear logic through the beautiful Curry-Howard-Lambek
correspondence \cite{?}.  This correspondence states that objects of a
monoidal category correspond to the formulas of linear logic and the
morphisms correspond to proofs of valid sequents of the logic.  I
propose that attack trees -- in many different flavors -- be modeled
as objects in monoidal categories, and hence, as formulas of linear
logic.

The Curry-Howard-Lambek correspondence is a three way relationship:
\begin{center}
  \setlength{\tabcolsep}{7pt}
  \begin{tabular}{ccccc}
    Categories & $\iff$ & Logic    & $\iff$   & Functional Programming\\
    Objects    & $\iff$ & Formulas & $\iff$   & Types    \\
    Morphisms  & $\iff$ & Proofs   & $\iff$   & Programs 
  \end{tabular}
\end{center}

% section introduction (end)

\section{Attack Trees}
\label{sec:attack_trees}

% section attack_trees (end)

\section{Concrete Semantics of Attack Trees in Dialectica Spaces}
\label{sec:concrete_semantics_of_attack_trees_in_dialectica_spaces}

% section concrete_semantics_of_attack_trees_in_dialectica_spaces (end)

\section{Abstract Semantics of Attack Trees in Monoidal Categories}
\label{sec:abstract_semantics_of_attack_trees_in_monoidal_categories}

% section abstract_semantics_of_attack_trees_in_monoidal_categories (end)

\section{Lina: A Domain Specific PL for Threat Analysis}
\label{sec:lina:_a_domain_specific_pl_for_threat_analysis}

% section lina:_a_domain_specific_pl_for_threat_analysis (end)

\bibliographystyle{plain}
\bibliography{ref}

\appendix

\section{Symmetric Monoidal Categories}
\label{sec:symmetric_monoidal_categories}

% section symmetric_monoidal_categories (end)


\section{Source Sink Graphs are Symmetric Monoidal}
\label{sec:source_sink_graphs_are_symmetric_monoidal}

% section source_sink_graphs_are_symmetric_monoidal (end)


\end{document}

%%% Local Variables: 
%%% mode: latex
%%% TeX-master: t
%%% End: 
