\documentclass{llncs}

\usepackage{amssymb,amsmath}
\usepackage{cmll}
\usepackage{stmaryrd}
\usepackage{todonotes}
\usepackage{mathpartir}
\usepackage{hyperref}
\usepackage[barr]{xy}

% Commands that are useful for writing about type theory and programming language design.
%% \newcommand{\case}[4]{\text{case}\ #1\ \text{of}\ #2\text{.}#3\text{,}#2\text{.}#4}
\newcommand{\interp}[1]{\llbracket #1 \rrbracket}
\newcommand{\normto}[0]{\rightsquigarrow^{!}}
\newcommand{\join}[0]{\downarrow}
\newcommand{\redto}[0]{\rightsquigarrow}
\newcommand{\nat}[0]{\mathbb{N}}
\newcommand{\fun}[2]{\lambda #1.#2}
\newcommand{\CRI}[0]{\text{CR-Norm}}
\newcommand{\CRII}[0]{\text{CR-Pres}}
\newcommand{\CRIII}[0]{\text{CR-Prog}}
\newcommand{\subexp}[0]{\sqsubseteq}
%% Must include \usepackage{mathrsfs} for this to work.
\newcommand{\powerset}[0]{\mathscr{P}}

\date{}

% Cat commands.
\newcommand{\cat}[1]{\mathcal{#1}}
\newcommand{\catop}[1]{\cat{#1}^{\mathsf{op}}}
\newcommand{\homs}[3]{\mathsf{Hom}_{\cat{#1}}(#2,#3)}
\newcommand{\limp}[0]{\multimap}
\newcommand{\dial}[0]{\mathsf{Dial_2}(\mathsf{Sets})}
\newcommand{\dialSets}[1]{\mathsf{Dial_{#1}}(\mathsf{Sets})}
\newcommand{\dcSets}[1]{\mathsf{DC_{#1}}(\mathsf{Sets})}
\newcommand{\sets}[0]{\mathsf{Sets}}
\newcommand{\obj}[1]{\mathsf{Obj}(#1)}
\newcommand{\mor}[1]{\mathsf{Mor(#1)}}
\newcommand{\id}[0]{\mathsf{id}}
\newcommand{\lett}[0]{\mathsf{let}\,}
\newcommand{\inn}[0]{\,\mathsf{in}\,}
\newcommand{\cur}[1]{\mathsf{cur}(#1)}
\newcommand{\curi}[1]{\mathsf{cur}^{-1}(#1)}

\begin{document}

\title{Proposing a New Foundation of Attack Trees in Monoidal Categories}

\author{Harley Eades III}
\institute{Computer and Information Sciences, Augusta University, Augusta, GA, \email{heades@augusta.edu}}

\maketitle 

\begin{abstract}
  TODO
\end{abstract}

\section{Introduction}
\label{sec:introduction}
\textbf{Attack trees} are a modeling tool, originally proposed by
Bruce Schneier \cite{Schneier:1999}, which are used to assess the
threat potential of a security critical system.  Attack trees have
since been used to analyze the threat potential of many types of
security critical systems, for example, cybersecurity of power grids
\cite{Ten:2007}, wireless networks \cite{Reinhardt:2012}, and many
others.  Attack trees consists of several goals, usually specified in
English prose, for example, ``compromise safe'' or ``obtain
administrative privileges'', where the root is the ultimate goal of
the attack and each node coming off of the root is a refinement of the
main goal into a subgoal.  Then each subgoal can be further refined.
The leaves of an attack tree make up the set of base attacks.  Subgoals
can be either disjunctively or conjunctively combined.

%% Extensions Of ATREES:
%%   - Attack nets
%%   - Sequential conjunction
\textbf{Extensions of attack trees.}  There have been a number of
extensions of attack trees to include new operators on goals.  One
such extension recasts attack trees into attack nets which have all of
the benefits of attack trees with the additional benefit of being able
to include the flaw hypothesis model for penetration testing
\cite{McDermott:2001}.  A second extension adds sequential conjunction
of attacks, that is, suppose $A_1$ and $A_2$ are attacks, then
$A_1;A_2$ is the attack obtained by performing $A_1$, and then
executing attack $A_2$ directly after $A_1$ completes
\cite{Jhawar:2015}.

\textbf{The need for a foundation.}  Attack trees for real-world
security scenarios can grow to be quite complex.  The attack tree
presented in \cite{Ten:2007} to access the security of power grids has
twenty-nine nodes with sixty counter measures attached to the nodes
throughout the tree.  The details of the tree spans several pages of
appendix.  The attack tree developed for the border gateway protocol
has over a hundred nodes \cite{Convey:2003}, and the details of the
tree spans ten pages.  Manipulating such large trees without a formal
semantics can be dangerous.

%% Semantics of ATREES:
%%   - Boolean logic
%%   - Attack nets (2000)
%%   - Multisets (2006)
%%   - Series-parallel graphs (extension of multisets) (2015)
\textbf{The formal semantics of attack trees.} The leading question
the field is seeking to answer by giving a mathematical foundation to
attack trees is ``what is an attack tree?''  There have been numerous
attempts at answering this question.  For example, attack trees have
been based on propositional logic and De Morgan Algebras
\cite{Kordy:2014,Kordy:2012,Pietre-Cambacedes:2010}, multisets
\cite{Mauw:2006}, Petri nets \cite{McDermott:2001}, tree automata
\cite{Camtepe:2007}, and series parallel graphs
\cite{Jhawar:2015}. \textbf{There is currently no known semantics of
  attack trees based in category theory}.

By far the most intuitive foundation of attack trees is propositional
logic or De Morgan algebras, however, neither of these properly
distinguish between attack trees with repeated subgoals.  If we
consider each subgoal as a \textbf{resource} then the attack tree
using a particular resource twice is different than an attack tree
where it is used only once.  The multiset semantics of attack trees
was developed precisely to provide a resource conscious foundation
\cite{Mauw:2006}. The same can be said for the Petri nets semantics
\cite{McDermott:2001}.  A second benefit of a semantics based in
multisets, Petri nets, and even tree automata is that operators on
goals in attack trees are associated with concurrency operators from
process algebra.  That is, the goals of an attack tree should be
thought of as being run concurrently -- it seems this connection to
process algebra has been overlooked.  Furthermore, when moving to
these alternate foundations \textbf{the intuitiveness and elegance of
  the propositional logic semantics is lost}.  Lastly, existing work
has focused on specifically what an attack tree is, and has not sought
to understand what the theory of attack trees is.

\textbf{Category Theory.}  Each of the various existing mathematical
foundations attempt to answer the question ``what is an attack
tree?'', but they are all seemingly very different mathematical
structures.  Is there a unifying core foundation common to all of
these existing foundations? A categorical foundation will answer this
question in the positive. The powers of category theory -- an abstract
branch of mathematics first proposed by Samuel Eilenberg and Saunders
Mac Lane \cite{MacLane:1971} -- are its ability to abstract away
unneeded details from a mathematical structure, and its ability to
form relationships between seemingly unrelated mathematical
structures.  For example, intuitionistic logic and the
$\lambda$-calculus both correspond to cartesian closed categories, and
hence, are different perspectives of the same theory
\cite{Lambek:1980}.  A categorical foundation of attack trees will
result in the most basic mathematical foundation of attack trees,
furthermore, it will reveal a relationship between all of the existing
mathematical foundations, and lastly, it will reconnect attack trees
with logic, but also forge new connections with functional programming
languages and formal verification through the Curry-Howard-Lambek
correspondence.  This implies that by giving attack trees a semantics
in category theory one obtains a more powerful semantic analysis and
the ability to \textbf{derive a programming language that can not only
  define attack trees, but also reason about them using formal
  verification} (Section~\ref{sec:prop-lina}).  In addition, the
various graphical languages used in category theory,
\cite{Selinger:2009}, may lead to new graphical tools for threat
analysis.

\textbf{This Proposal.} I propose to found attack trees in linear
logic rather than propositional logic by giving attack trees a
semantics in symmetric monoidal categories.  A semantics in symmetric
monoidal categories is a generalization over the previous forms of
models of attack trees.  Multisets and Petri nets are examples of
symmetric monoidal categories \cite{Brown:1991,Tzouvaras:1998}.
Furthermore, symmetric monoidal categories are a categorical model of
linear logic \cite{dePaiva:2014}, thus regaining the elegant
connection between attack trees and logical formulas.  The connection
to linear logic also opens the door to a connection between threat
analysis using attack trees and process algebra such as Petri nets,
but also Chu spaces \cite{Pratt:1999}, dialectica spaces
\cite{dePaiva:2006b}, and session types
\cite{Wadler:2012,Caires:2010}.  I and my trainees will fully develop
the semantics of attack trees in symmetric monoidal categories and
show that not only can attack trees be modeled by these types of
categories, but so can a range of their extensions like attack trees
with sequential conjunction.  We will also investigate new attack tree
operators based on the connection to process algebra.  Then we will
develop a new formal system called Lina for \underline{Lin}ear Threat
\underline{A}nalysis (Section~\ref{sec:prop-lina}).  Lina will be a
statically typed linear polymorphic functional domain-specific
programming language designed to construct, manipulate, and prove
properties of attack trees.  Lina will represent attack trees as
linear types, and thus, programs between these types will be
considered transformations of attack trees.  The semantics of Lina
will also be in symmetric monoidal categories forming a tight
correspondence with the semantics of attack trees.  This system will
be the first threat analysis tool to support proving properties of
attack trees, thus \textbf{connecting software verification to threat
  analysis.}  Finally, new threat analysis tools can be built on top
of Lina.
% section introduction (end)

\bibliographystyle{plain}
\bibliography{ref}

\end{document}

%%% Local Variables: 
%%% mode: latex
%%% TeX-master: t
%%% End: 
