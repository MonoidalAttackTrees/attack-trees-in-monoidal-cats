\newcommand{\forth}{\frac{1}{4}}
\newcommand{\half}{\frac{1}{2}}

Kordy et al.~\cite{Kordy:2012} gave a very elegant and simple
semantics of attack-defense trees in boolean algebras.  Unfortunately,
while their semantics is elegant it does not capture the resource
aspect of attack trees, it allows contraction, and it does not provide
a means to model sequential composition.  In this section we give a
semantics of attack trees in the spirit of Kordy et al.'s using a four
valued logic.  This section was formally verified in the Agda Proof
Assistant~\cite{Norell:2009}\footnote{The formalization can be found at \url{https://github.com/MonoidalAttackTrees/ATLL-Formalization}}.

The propositional variables of our quaternary logic, denoted by $A$, $B$,
$C$, and $D$, range over the set $\mathsf{4} = \{0, \forth, \half,
1\}$.  We think of $0$ and $1$ as we usually do in boolean algebras,
but we think of $\forth$ and $\half$ as intermediate values that can
be used to break various structural rules.  In particular we will use
these values to prevent exchange for sequential composition from
holding, and contraction from holding for parallel and sequential
composition.
\begin{definition}
  \label{def:logical-connectives}
  The logical connectives of our four valued logic are defined as
  follows:
  \begin{itemize}
  \item[] Parallel and Sequential Conjunction:\vspace{-13px}
    \begin{center}
      \begin{math}
        \setlength{\arraycolsep}{10px}
        \begin{array}{lll}
          \begin{array}{lll}
            A \odot_4 B = 1,\\
            \,\,\,\,\,\,\,\text{where neither $A$ nor $B$ are $0$}\\
          A \odot_4 B = 0, \text{otherwise}\\
          \\
        \end{array}
        &
        \begin{array}{lll}          
          A \rhd_4 B = 1,\\
          \,\,\,\,\,\,\,\text{where } A \in \{\half, 1\} \text{ and } B \neq 0\\
          \forth \rhd_4 B = \forth, \text{where $B \neq 0$}\\[2px]         
          A \rhd_4 B = 0, \text{otherwise}
        \end{array}
        \end{array}
      \end{math}
    \end{center}
    \vspace{-5px}
  \item[] Choice: $A \sqcup_4 B = \mathsf{max}(A,B)$    
  \end{itemize}
\end{definition}
These definitions are carefully crafted to satisfy the necessary
properties to model attack trees.  Comparing these definitions with
Kordy et al.'s~\cite{Kordy:2012} work we can see that choice is
defined similarly, but parallel composition is not a product --
ordinary conjunction -- but rather a linear tensor product, and
sequential composition is not actually definable in a boolean algebra,
and hence, makes heavy use of the intermediate values to insure that
neither exchange nor contraction hold.  

We use the usual notion of equivalence between propositions, that is,
propositions $\phi$ and $\psi$ are considered equivalent, denoted by
$\phi \equiv \psi$, if and only if they have the same truth tables. In
order to model attack trees the previously defined logical connectives
must satisfy the appropriate equivalences corresponding to the
equations between attack trees.  These equivalences are all proven by
the following result.
\begin{lemma}[Properties of the Attack Tree Operators in the Quaternary Semantics]
  \label{lemma:props_atree_ops_quaternary-semantics}
  \begin{itemize}
  \item[] (Symmetry) For any $A$ and $B$, $A \bullet B \equiv B \bullet A$, for $\bullet \in \{\odot_4, \sqcup_4\}$.\\[-5px]
  \item[] (Symmetry for Sequential Conjunction) It is not the case that, for any $A$ and $B$, $A \rhd_4 B \equiv B \rhd_4 A$.\\[-5px]
  \item[] (Associativity) For any $A$, $B$, and $C$, $(A \bullet B) \bullet C \equiv A \bullet (B \bullet C)$, for $\bullet \in \{\odot_4, \rhd_4, \sqcup_4\}$.\\[-5px]
  \item[] (Contraction for Parallel and Sequential Conjunction) It is not the case that for any $A$, $A \bullet A \equiv A$, for $\bullet \in \{\odot_4, \rhd_4\}$.\\[-5px]
  \item[] (Distributive Law) For any $A$, $B$, and $C$, $A \bullet (B \sqcup_4 C) \equiv (A \bullet B) \sqcup_4 (A \bullet C)$, for $\bullet \in \{\odot_4, \rhd_4\}$.\\[-5px]
  \end{itemize}
\end{lemma}
\begin{proof}
  Symmetry, associativity, contraction for choice, and the
  distributive law for each operator hold by simply comparing truth
  tables.  As for contraction for parallel composition, suppose $A =
  \forth$.  Then by definition $A \odot_4 A = 1$, but $\forth$ is not
  $1$.  Contraction for sequential composition also fails, suppose $A
  = \half$.  Then by definition $A \rhd_4 A = 1$, but $\half$ is not
  $1$.  Similarly, symmetry fails for sequential composition. Suppose
  $A = \forth$ and $B = \half$.  Then $A \rhd_4 B = \forth$, but $B
  \rhd_4 A = 1$.
\end{proof}

At this point it is quite easy to model attack trees as formulas.  The
following defines their interpretation.
\begin{definition}
  \label{def:interp-aterms-quaternary}
  Suppose $\mathbb{B}$ is some set of base attacks, and $\nu :
  \mathbb{B} \mto \mathsf{PVar}$ is an assignment of base attacks to
  propositional variables.  Then we define the interpretation of
  attack trees to propositions as follows:
  \begin{center}
    \begin{math}
      \setlength{\arraycolsep}{5px}
      \begin{array}{lll}
        \begin{array}{lll}
          \interp{[[b]] \in \mathbb{B}} & = & \nu([[b]])\\
          \interp{[[AND(A, B)]]} & = & \interp{[[A]]} \odot_4 \interp{[[B]]}\\
        \end{array}
        &
        \begin{array}{lll}
          \interp{[[SEQ(A,B)]]} & = & \interp{[[A]]} \rhd_4 \interp{[[B]]}\\
          \interp{[[OR(A,B)]]} & = & \interp{[[A]]} \sqcup_4 \interp{[[B]]}\\
        \end{array}
      \end{array}
    \end{math}
  \end{center}
\end{definition}
We can use this semantics to prove equivalences between attack trees.
\begin{lemma}[Equivalence of Attack Trees in the Quaternary Semantics]
  \label{lemma:equivalence_of_attack_trees}
  Suppose $\mathbb{B}$ is some set of base attacks, and $\nu :
  \mathbb{B} \mto \mathsf{PVar}$ is an assignment of base attacks to
  propositional variables.  Then for any attack trees $[[A]]$ and
  $[[B]]$, if $[[A ~ B]]$, then $\interp{[[A]]} \equiv
  \interp{[[B]]}$.
\end{lemma}
\begin{proof}
  This proof holds by induction on the form of $[[A ~ B]]$.
\end{proof}
This is a very simple and elegant semantics, but it also leads to a
more substantial theory.

\section{Specialization in the Quaternary Semantics}
\label{sec:specialization_in_the_quaternary_semantics}
The quaternary semantics introduced in the previous section does
indeed capture all of the equivalences of attack trees, but it also
supports proving specializations.  Consider the example attack trees
in Fig.~\ref{fig:2}.  
\begin{figure}
  \begin{tabular}{|l|l|}
    \hline
    \begin{tabular}{l}
      \textbf{A.}\\
      \begin{minipage}{\textwidth/2}
        \vspace{3px}
        \begin{minted}[fontsize=\scriptsize]{haskell}
 and_node "obtain secret"
  (or_node "obtain encryped file"
     (base_na "bribe sysadmin")
     (base_na "steal backup"))
  (seq_node "obtain password"
     (base_na "break into system")
     (base_na "install keylogger"))
        \end{minted}
        \vspace{-3px}
      \end{minipage}      
    \end{tabular}
    &
    \begin{tabular}{l}
      \textbf{B.}\\
      \begin{minipage}{\textwidth/2-5.6px}
        \vspace{3px}
        \begin{minted}[fontsize=\scriptsize]{haskell}
 seq_node "break in, obtain secret"
  (base_na "break into system")
  (and_node "obtain secret inside"
    (base_na "install keylogger")
    (base_na "steal backup"))
        \end{minted}
        \vspace{20px}
      \end{minipage}      
    \end{tabular}    
    \\    
    \hline
  \end{tabular}\vspace{-2px}
  \begin{tabular}{|l|}
    \begin{tabular}{l}
      \\[-9.8px]
    \textbf{C.}\\
      \begin{minipage}{\textwidth}
        \begin{minted}[fontsize=\scriptsize]{haskell}
 or_node "obtain secret"
  (and_node "obtain secret via sysadmin"
   (base_na "bribe sysadmin")
   (seq_node "obtain password"
     (base_na "break into system")
     (base_na "install keylogger")))
   (seq_node "break in, obtain secrect"
     (base_na "break into system")
     (and_node "obtain secret inside"
       (base_na "install keylogger")
       (base_na "steal backup")))        
        \end{minted}
        \vspace{2px}
      \end{minipage}
    \end{tabular}
    \\
    \hline
  \end{tabular}
    \caption{Encrypted Data Attack from Figure~1 (A), Figure~3 (B), and Figure~2 (C) of Horne et al.~\cite{horne2017semantics}}
    \label{fig:2}
\end{figure}
Attack tree C is a sound specialization of attack tree A, and attack
tree B is a sound specialization of attack tree A.  Attack tree C
requires the attacker to break into the system before they can steal
the backup, but attack tree A does not require this.  Then attack tree
B has dropped bribing the sysadmin and simply requires the attacker to
just steal the backups.  Notice that none of the attack trees in
Fig.~\ref{fig:2} are equivalent.  So how do we prove these
specializations are sound?

We simply define a notion of entailment in the quaternary semantics.
Denote by $A \leq_4 B$ the obvious ordering on $\mathsf{4}$.  Then we
have the following result immediately.
\begin{lemma}[Entailment in the Quaternary Semantics]
  \label{lemma:entailment_in_the_quaternary_semantics}
  $A \equiv B$ if and only if $A \leq_4 B$ and $B \leq_4 A$
\end{lemma}
This result shows that we can now break up the equivalence of attack
trees into directional properties captured here by entailments, and
hence, every equivalence proved in the previous section can also be
used directionally.

We may now formally define when a attack tree is a sound
specialization of another attack tree.
\begin{definition}
  \label{def:specialization}
  An attack tree $A$ is a sound specialization of an attack $B$ if and
  only if $\interp{A} \leq_4 \interp{B}$.
\end{definition}
The next result proves some additional properties in the quaternary
semantics that can be used to reason about attack trees.
\begin{lemma}[Properties of Entailment in the Quaternary Semantics]
  \label{lemma:properties_of_entailment_in_the_quaternary_semantics}
  \begin{tabular}{lll}
    \begin{tabular}{lll}
    Ideal Entailments:\\
    \begin{math}
      \footnotesize
      \begin{array}{lll}
        ((a \odot_4 b) \rhd_4 (c \odot_4 d)) \leq_4 ((a \rhd_4 c) \odot_4 (b \rhd_4 d))\\
        ((a \odot_4 b) \rhd_4 c) \leq_4 (a \odot_4 (b \rhd_4 c))\\                    
        (a \rhd_4 (b \odot_4 c)) \leq_4 (b \odot_4 (a \rhd_4 c))\\
        (a \rhd_4 b) \leq_4 (a \odot_4 b)\\          
      \end{array}
    \end{math}      
  \end{tabular}
  \\\\
  \begin{tabular}{lllll}
    \begin{tabular}{lll}
      Filter Entailments:\\
      \begin{math}
        \footnotesize
        \begin{array}{lll}
          ((a \rhd_4 c) \odot_4 (b \rhd_4 d)) \leq_4 ((a \odot_4 b) \rhd_4 (c \odot_4 d))\\
          (a \odot_4 (b \rhd_4 c)) \leq_4 ((a \odot_4 b) \rhd_4 c)\\
          \\
          \\
        \end{array}
      \end{math}           
    \end{tabular}
    & \quad & 
    \begin{tabular}{lll}
        Choice Entailments:\\
        \begin{math}
          \footnotesize
          \begin{array}{lll}
            a \leq_4 (a \sqcup_4 b)\\
            b \leq_4 (a \sqcup_4 b)\\
          \end{array}
        \end{math}
      \end{tabular}
  \end{tabular}
  \end{tabular}
\end{lemma}
Each of the above entailments are due to Horne et
al.~\cite{horne2017semantics}.  They introduce two types of semantics
called the \emph{ideal semantics} and the \emph{filter semantics}.
The former satisfies all of the entailments in
Lemma~\ref{lemma:props_atree_ops_quaternary-semantics} and the left
side of
Lemma~\ref{lemma:properties_of_entailment_in_the_quaternary_semantics},
but the latter is similar, however, satisfying the right side of
Lemma~\ref{lemma:properties_of_entailment_in_the_quaternary_semantics}.
They were able to show that the ideal entailments allow one to prove
properties of attack trees with different attribute domains than the
filter entailments.

In comparison with their work the semantics presented here is a blend
of the ideal and filter semantics.  It primarily consists of the ideal
semantics, but as we can see from
Lemma~\ref{lemma:properties_of_entailment_in_the_quaternary_semantics}
the first two axioms are actually logical equivalences. This implies
that this semantics can prove properties that neither the ideal nor the
filter semantics can capture.  However, we do not yet know which
attribute domains correspond to this semantics.  We can isolate
ourselves to just the ideal or the filter semantics, but what
attribute domains can we reason about in this semantics when we blend
them?  This is left to future work.

We can now formally prove that the attack tree C is a specialization
of attack tree A, and that attack tree B is a specialization of attack
tree A from Fig.~\ref{fig:2}.
\begin{example}
  \label{ex:ex1}
  First, consider the following assignment:
\begin{center}
  \begin{math}
    \begin{array}{lllllll}
      a := \verb!"bribe sysadmin"!
      & \quad &
      b := \verb!"break into system"!
      \\
      c := \verb!"install keylogger"!
      & \quad &
      d := \verb!"steal backup"!
    \end{array}
  \end{math}
\end{center}
Then we have the following interpretations:
\begin{center}
  \footnotesize
  \begin{math}
    \begin{array}{c}
      \begin{array}{ccc}
        \begin{array}{rclcl}
        \interp{A} & = & \interp{[[AND(OR(a,dd),SEQ(b,c))]]} \\
                   & = & (a \sqcup_4 d) \odot_4 (b \rhd_4 c)\\
      \end{array}
      & \quad &
      \begin{array}{rclcl}
        \interp{B} & = & \interp{[[SEQ(b,AND(c,dd))]]} \\
                   & = & b \rhd_4 (c \odot_4 d)\\      
      \end{array}
      \end{array}
      \\ \\
      \begin{array}{rclcl}
        \interp{C} & = & \interp{[[OR(AND(a,SEQ(b,c)),SEQ(b,AND(c,dd)))]]}\\
                   & = & (a \odot_4 (b \rhd_4 c)) \sqcup_4 (b \rhd_4 (c \odot_4 d))\\
      \end{array}
    \end{array}        
  \end{math}
\end{center}
We reuse the same names for base attacks across the interpretations
above.  Finally, we have the following two entailments:
\begin{center}
  \begin{math}
    \footnotesize
    \begin{array}{lll}
      \begin{array}{lll}
        \underline{\interp{C} \leq_4 \interp{A}}:\\[4px]
        \,\,\,\,\begin{array}{rl}
               & (a \odot_4 (b \rhd_4 c)) \sqcup_4 (b \rhd_4 (c \odot_4 d))\\
        \leq_4 &  (a \odot_4 (b \rhd_4 c)) \sqcup_4 (b \rhd_4 (d \odot_4 c))\\
        \leq_4 &  (a \odot_4 (b \rhd_4 c)) \sqcup_4 (d \odot_4 (b \rhd_4 c))\\
        \leq_4 &  (a \sqcup_4 d) \odot_4 (b \rhd_4 c)\\
        \end{array}
      \end{array}
      & \quad &
      \begin{array}{lll}
        \underline{\interp{B} \leq_4 \interp{A}}:\\[4px]
        \,\,\,\,\,\begin{array}{lll}
               & b \rhd_4 (c \odot_4 d)\\
        \leq_4 &  b \rhd_4 (c \odot_4 (a \sqcup_4 d))\\
        \leq_4 &  b \rhd_4 ((a \sqcup_4 d) \odot_4 c)\\
        \leq_4 & (a \sqcup_4 d) \odot_4 (b \rhd_4 c)\\
        \end{array}
      \end{array}
    \end{array}
  \end{math}
\end{center}
Notice that neither $[[A]] \leq_4 [[C]]$ nor $[[A]] \leq_4 [[B]]$
hold, and thus, equivalences cannot prove the previous properties.
\end{example}

% section specialization_in_the_quaternary_semantics (end)


%%% Local Variables: 
%%% mode: latex
%%% TeX-master: main.tex
%%% End: 
