\documentclass{llncs}

\usepackage{wrapfig}
\usepackage{amssymb,amsmath,mathtools}
\usepackage{cmll}
\usepackage{stmaryrd}
\usepackage{todonotes}
\usepackage{mathpartir}
\usepackage{hyperref}
\usepackage{mdframed}
\usepackage{thmtools, thm-restate}
\usepackage{enumitem}
\usepackage{minted}

%% This renames Barr's \to to \mto.  This allows us to use \to for imp
%% and \mto for a inline morphism.
\let\mto\to
\let\to\relax
\newcommand{\to}{\rightarrow}

% Commands that are useful for writing about type theory and programming language design.
%% \newcommand{\case}[4]{\text{case}\ #1\ \text{of}\ #2\text{.}#3\text{,}#2\text{.}#4}
\newcommand{\interp}[1]{\llbracket #1 \rrbracket}
\newcommand{\normto}[0]{\rightsquigarrow^{!}}
\newcommand{\join}[0]{\downarrow}
\newcommand{\redto}[0]{\rightsquigarrow}
\newcommand{\nat}[0]{\mathbb{N}}
\newcommand{\fun}[2]{\lambda #1.#2}
\newcommand{\CRI}[0]{\text{CR-Norm}}
\newcommand{\CRII}[0]{\text{CR-Pres}}
\newcommand{\CRIII}[0]{\text{CR-Prog}}
\newcommand{\subexp}[0]{\sqsubseteq}
%% Must include \usepackage{mathrsfs} for this to work.
\newcommand{\powerset}[0]{\mathscr{P}}

\date{}

% Cat commands.
\newcommand{\Four}[0]{\mathsf{4}}
\newcommand{\cat}[1]{\mathcal{#1}}
\newcommand{\catop}[1]{\cat{#1}^{\mathsf{op}}}
\newcommand{\Hom}[3]{\mathsf{Hom}_{\cat{#1}}(#2,#3)}
\newcommand{\limp}[0]{\multimap}
\newcommand{\dial}[1]{\mathsf{Dial_{#1}}(\mathsf{Sets})}
\newcommand{\dcSets}[1]{\mathsf{DC_{#1}}(\mathsf{Sets})}
\newcommand{\sets}[0]{\mathsf{Sets}}
\newcommand{\obj}[1]{\mathsf{Obj}(#1)}
\newcommand{\mor}[1]{\mathsf{Mor(#1)}}
\newcommand{\id}[0]{\mathsf{id}}
\newcommand{\lett}[0]{\mathsf{let}\,}
\newcommand{\inn}[0]{\,\mathsf{in}\,}
\newcommand{\cur}[1]{\mathsf{cur}(#1)}
\newcommand{\curi}[1]{\mathsf{cur}^{-1}(#1)}

\newenvironment{changemargin}[2]{%
  \begin{list}{}{%
    \setlength{\topsep}{0pt}%
    \setlength{\leftmargin}{#1}%
    \setlength{\rightmargin}{#2}%
    \setlength{\listparindent}{\parindent}%
    \setlength{\itemindent}{\parindent}%
    \setlength{\parsep}{\parskip}%
  }%
  \item[]}{\end{list}}

%% % Theorems
%% \newtheorem{theorem}{Theorem}
%% \newtheorem{lemma}[theorem]{Lemma}
%% \newtheorem{fact}[theorem]{Fact}
%% \newtheorem{corollary}[theorem]{Corollary}
%% \newtheorem{definition}[theorem]{Definition}
%% \newtheorem{remark}[theorem]{Remark}
%% \newtheorem{proposition}[theorem]{Proposition}
%% \newtheorem{notn}[theorem]{Notation}
%% \newtheorem{observation}[theorem]{Observation}

%% Ott
\input{aterms-ott}
\input{atll-ott}

\begin{document}

\title{On Linear Logic, Functional Programming, and Attack Trees}

\author{Harley Eades III\inst{1} \and Jiaming Jiang\inst{2} \and Aubrey Bryant\inst{1}}

\institute{Computer Science, Augusta University, \href{mailto:harley.eades@gmail.com}{harley.eades@gmail.com}
\and
Computer Science, North Carolina State University}

\maketitle

\begin{abstract}
  This paper has two main contributions, the first is a new simple
  linear logical semantics of causal attack trees in four-valued truth
  tables.  Our semantics is very expressive, supporting
  specializations, and combines in an interesting way the \emph{ideal}
  and \emph{filter} semantics of causal attack trees. Our second
  contribution is Lina, a new embedded, in Haskell, domain specific
  functional programming language for conducting threat analysis using
  attack trees.  Lina has many benefits over existing tools, for
  example, Lina allows one to specify attack trees very abstractly
  increasing reuse, and is highly compositional allowing one to break
  down complex attack trees into smaller ones that can be reasoned
  about and analyzed incrementally.  Furthermore, Lina supports
  automatically proving properties about attack trees using Maude, and
  is the first tool to support proving specializations about attack
  trees by using the truth table semantics first proposed in this
  paper.
\end{abstract}

\section{Introduction}
\label{sec:introduction}
\input{intro-output}

\vspace{-7px}
\section{Causal Attack Trees}
\label{sec:causal_attack_trees}
\input{attack-trees-output}
% section causal_attack_trees (end)

\vspace{-7px}
\section{A Quaternary Semantics for Causal Attack Trees}
\label{sec:a_quaternary_semantics_for_causal_attack_trees}
\input{quaternary-semantics-output}
% section a_ternary_semantics_for_causal_attack_trees (end)

%% \section{The Attack Tree Linear Logic (ATLL)}
%% \label{sec:the_attack_tree_linear_logic_(atll)}
%% \input{atll-output}
%% % section the_attack_tree_linear_logic_(atll) (end)

\vspace{-7px}
\section{Lina: An EDSL for Conducting Threat Analysis using Causal Attack Trees}
\label{sec:lina:_an_edsl_for_conducting_threat_analysis_using_causal_attack_trees}
%% - Different types of attack trees
%%    - Process
%%    - Attributed
%%    - Full
%%    - Configurations (change this to attribute domains?)
%%      - We don't yet support or_node operators
%% - The AT data types
%%   - Mention their generality (polymorphism)
%% - Syntax
%%   - Should this be a grammar? yes
%% - Backend support
%%   - Flexibility in an id labeled tree where attributes and labels are isolated in tables
%%   - Maude  
%%     - Causal AT Eq
%%     - Quaternary Semantics (specializations)
%%       - We can automatically prove the example from the last section.
%% - Attribute domains
%%    - Generating attacks
%%    - Ordering attacks based on type classes
All of the models mentioned in this paper have been incorporated into
a new embedded domain specific language (EDSL) for conducting threat
analysis called Lina\footnote{Lina is under active development and its
  implementation can be found online at
  \url{https://github.com/MonoidalAttackTrees/Lina}} which means
small, young palm tree, but we constructed the name by combining the
words linear and attack.
 
Lina is embedded inside of Haskell, a statically-typed functional
programming language.  The most important property of any EDSL is that
they subsume the entirety of their host language, and can be
prototyped quite rapidly.  Haskell contributes several advantages,
such as cutting edge verification tools, and a strong type system for
catching bugs quickly.  

Lina currently supports three types of causal attack trees:
\begin{itemize}
\item \underline{Process Attack Trees}: these are attack trees with no attributes
  at all,
  
\item \underline{Attributed Process Attack Trees}: these are attack trees with
  attributes on the base attacks only.  This is an intermediate
  representation used to build full attack trees.
  
\item \underline{Full Attack Trees}: these are attributed process attack trees
  with an associated attribute domain.
\end{itemize}
\newcommand{\mh}[1]{\mintinline{haskell}{#1}} Internally, we represent
causal attack trees by a simple data type, called \mh{IAT}, whose
nodes are labeled with an integer identifier we call \mh{ID}.  We then
define each type of attack tree as a record (labeled tuple):

\noindent \begin{tabular}{|l|ll|}
  \hline & & \\[-7px]
    \begin{minipage}{2in}        
    \begin{minted}[fontsize=\scriptsize]{haskell}
-- Attributed Process Attack Tree
data APAttackTree attribute label =
 APAttackTree {
  process_tree :: IAT,
  labels :: B.Bimap label ID,
  attributes :: M.Map ID attribute
 } 
    \end{minted}
    \vspace{-2px}
  \end{minipage}  
  & \quad &
    \begin{minipage}{2.4in}        
    \begin{minted}[fontsize=\scriptsize]{haskell}
-- Process Attack Tree
type PAttackTree label = APAttackTree () label

-- Full Attack Tree
data AttackTree attribute label = AttackTree {            
      ap_tree :: APAttackTree attribute label,
      configuration :: Conf attribute
}
    \end{minted}
    \end{minipage}\\
    & &\\    
    \hline
\end{tabular}

\noindent
A \mh{B.Bimap} is a dictionary where we can efficiently look up
\mh{ID}s given a \verb!label! or efficiently look up \verb!label!s
given an \mh{ID}. A \mh{M.Map} is a typical dictionary, and \mh{()} is
the unit type.

This design has several benefits. Internal attack trees are very easy
to translate to various backends, especially formulas because we can
use the \mh{ID}s on base attacks as atomic formulas -- which has its
own benefits discussed below -- and modifying labels and attributes is
more efficient than having them labeled on the trees themselves.  The
previous data types reveal that actually all attack trees are
attributed process attack trees, and a process attack tree simply does
not use the attributes.  This allows Lina to offer a uniform syntax
for specifying all types of attack tree.

One important aspect of the definition of the various forms of attack
trees is that the types \verb!label! and \verb!attribute! are actually
type variables, and thus, our definition of attack trees is very
general; in fact, \verb!label! and \verb!attribute! can be
instantiated with any type whose elements are comparable.  This
property is captured by ad-hoc polymorphism using type classes in
Haskell, and is checked during type checking.

Conducting threat analysis using attack trees requires them to be
associated with an attribute domain.  Typically, an attribute domain
is a set, together with operations for computing the attribute of the
branching nodes of an attack tree given attributes on the base
attacks.  In Lina attribute domains are defined by a type, here called
\verb!attribute!, and a configuration:
\begin{center}
  \begin{minted}[fontsize=\scriptsize]{haskell}
data Conf attribute = (Ord attribute) => Conf {
  orOp  :: attribute -> attribute -> attribute,    
  andOp :: attribute -> attribute -> attribute,
  seqOp :: attribute -> attribute -> attribute
}
\end{minted}
\end{center}
Utilizing higher-order functions we can define configurations easily
and generically.  For example, here is the configuration that computes
the minimum attribute for choice nodes, the maximum attribute for
parallel nodes, and takes the sum of the children nodes as the
attribute for sequential nodes:
\begin{center}
  \begin{minted}[fontsize=\scriptsize]{haskell}
minMaxAddConf :: (Ord attribute,Semiring attribute) => Conf attribute
minMaxAddConf = Conf min max (.+.)
\end{minted}
\end{center}
Notice here that this configuration will work with any type at all
whose elements are comparable and form a semiring, thus making
configurations generic and reusable.  This includes types like
\mh{Integer} and \mh{Double}.

The definitional language for attributed process attack trees of type \\
\mh{APAttackTree attribute label} is described by the following
grammar:
\begin{center}
  \footnotesize
  \begin{math}
    \begin{array}{lrlll}
      at & ::=  & \mh{base_na label} \mid \mh{base_wa attribute label} \mid \mh{or_node label at1 at2} \\
      & \mid & \mh{and_node label at1 at2} \mid \mh{seq_node label at1 at2}\\
    \end{array}
  \end{math}
\end{center}
A full example of the definition of an attributed process attack tree
for attacking an autonomous vehicle can be found in
Fig.~\ref{fig:vehicle_attack}. 
\begin{figure}
      \begin{tabular}{|l|}
        \hline\\[-7px]
      \begin{minipage}{\textwidth}
        \begin{minted}[fontsize=\scriptsize]{haskell}
import Lina.AttackTree

vehicle_attack :: APAttackTree Double String
vehicle_attack = start_PAT $
  or_node "Autonomous Vehicle Attack"
    (seq_node "External Sensor Attack"
       (base_wa 0.2 "Modify Street Signs to Cause Wreck")
       (and_node "Social Engineering Attack"
          (base_wa 0.6 "Pose as Mechanic")
          (base_wa 0.1 "Install Malware")))
    (seq_node "Over Night Attack"
       (base_wa 0.05 "Find Address where Car is Stored")
       (seq_node "Compromise Vehicle"
          (or_node "Break In"
             (base_wa 0.8 "Break Window")
             (base_wa 0.5 "Disable Door Alarm/Locks"))
          (base_wa 0.1 "Install Malware")))
        \end{minted}
        \vspace{2px}
      \end{minipage} \\
      \hline
    \end{tabular}
  \caption{Lina Script for an Autonomous Vehicle Attack.}
  \label{fig:vehicle_attack}
\end{figure}
The definition of \mh{vehicle_attack} begins with a call to
\mh{start_PAT}.  Behind the scenes, all of the \mh{ID}'s within the
internal attack tree are managed implicitly, which requires the
internals of Lina to work within a special state-based type.  The
function \mh{start_PAT} initializes this state.

Finally, we can define the vehicle attack tree as follows:
\begin{minted}[fontsize=\scriptsize]{haskell}
vehicle_AT :: AttackTree Double String
vehicle_AT = AttackTree vehicle_attack minMaxMaxConf
\end{minted}
This attack tree associates the vehicle attack attributed process
attack tree with a configuration called \mh{minMaxMaxConf} that simply
takes the minimum as the attribute of choice nodes, and the maximum as
the attribute of every parallel and sequential node.

Lina as two important features that other tools lack.  First, it can
abstract the definitions of attack trees. Second, it is highly
compositional, because it is embedded inside of a functional
programming language.  Consider the following abstraction of
\mh{vehicle_attack}:
\begin{minted}[fontsize=\scriptsize]{haskell}
vehicle_AT' :: Conf Double -> AttackTree Double String
vehicle_AT' conf = AttackTree vehicle_attack conf
\end{minted}
Here the configuration has been abstracted.  This facilitates
experimentation because the security practitioner can run several
different forms of analysis on the same attack tree using different
attribute domains.

Attack trees in Lina can also be composed and decomposed; hence,
complex trees can be broken down into smaller ones, then studied in
isolation.  This helps facilitate correctness, and offers more
flexibility.  As an example, in Fig.~\ref{fig:vehicle_attack_decomp}
we break up \mh{vehicle_attack} into several smaller attack trees.
\begin{figure}  
  \begin{tabular}{|l|l|l|}
    \hline & \\[-8px]
    \begin{tabular}{l}
      \begin{minipage}{2.2in}        
        \begin{minted}[fontsize=\scriptsize]{haskell}
se_attack :: APAttackTree Double String
se_attack = start_PAT $
  and_node "social engineering attack"
     (base_wa 0.6 "pose as mechanic")
     (base_wa 0.1 "install malware")
        \end{minted}
        \vspace{3px}
      \end{minipage}
    \end{tabular}
    &
    \begin{tabular}{l}
      \begin{minipage}{2.5in}        
        \begin{minted}[fontsize=\scriptsize]{haskell}
bi_attack :: APAttackTree Double String
bi_attack = start_PAT $
  or_node "break in"
     (base_wa 0.8 "break window")
     (base_wa 0.5 "disable door alarm/locks")
        \end{minted}
        \vspace{2px}
      \end{minipage}
    \end{tabular}\\
    \hline & \\[-8px]
        \begin{tabular}{l}
      \begin{minipage}{2.2in}        
        \begin{minted}[fontsize=\scriptsize]{haskell}
cv_attack :: APAttackTree Double String
cv_attack = start_PAT $
  seq_node "compromise vehicle"
    (insert bi_attack)
    (base_wa 0.1 "install malware")
        \end{minted}
        \vspace{10px}
      \end{minipage}
    \end{tabular}
    &
    \begin{tabular}{l}
      \begin{minipage}{2.5in}        
        \begin{minted}[fontsize=\scriptsize]{haskell}
es_attack :: APAttackTree Double String
es_attack = start_PAT $
  seq_node "external sensor attack"
       (base_wa 0.2 "modify street signs to cause
                     wreck")
       (insert se_attack)
        \end{minted}
        \vspace{2px}
      \end{minipage}
    \end{tabular}\\
    \hline & \\[-8px]
        \begin{tabular}{l}
      \begin{minipage}{2.2in}        
        \begin{minted}[fontsize=\scriptsize]{haskell}
on_attack :: APAttackTree Double String
on_attack = start_PAT $
  seq_node "overnight attack"
     (base_wa 0.05 "Find address where car
                    is stored")
     (insert cv_attack)
        \end{minted}
        \vspace{3px}
      \end{minipage}
    \end{tabular}
    &
    \begin{tabular}{l}
      \begin{minipage}{2.5in}        
        \begin{minted}[fontsize=\scriptsize]{haskell}
vehicle_attack'' :: APAttackTree Double String
vehicle_attack'' = start_PAT $
  or_node "Autonomous Vehicle Attack"
    (insert es_attack)
    (insert on_attack)
        \end{minted}
        \vspace{11px}
      \end{minipage}
    \end{tabular}\\
    \hline 
\end{tabular}
\caption{The Autonomous Vehicle Attack Decomposed}
\label{fig:vehicle_attack_decomp}
\end{figure}
We can see in the example that if we wish to use an already defined
attack tree in an attack tree we are defining, then we can make use of the
\mh{insert} function.  As we mentioned above, behind the scenes Lina
maintains a special state that tracks the identifiers of each node;
thus, when one wishes to insert an existing attack tree, which will
have its own identifier labeling, into a new tree, then that internal
state must be updated; thus, \mh{insert} carries out this updating.
Lina is designed so that the user never has to encounter that internal
state.

So far we have introduced Lina's basic design and definitional
language for specifying causal attack trees, and we have already begun
seeing improvements over existing tools; however, Lina has so much
more to offer.  We now introduce Lina's support for reasoning about
and performing analysis on causal attack trees.

Kordy et al.~\cite{Kordy2017} introduce the SPTool, an equivalence
checker for causal attack trees that makes use of the rewriting logic
system Maude~\cite{clavel2005maude} which allows one to specify
rewrite systems and systems of equivalences.  Kordy et al. specify the
equivalences for causal attack trees from Jhawar et
al.'s~\cite{Jhawar:2015} work in Maude, and then use Maude's querying
system to automatically prove equivalences between causal attack
trees.  This is a great idea, and we incorporate it into Lina, but we
make several advancements over SPTool.

Lina includes a general Maude interface, and allows the user to easily
define new Maude backends, where a \emph{Maude backend} corresponds to
a Maude specification of a particular rewrite system.  Currently, Lina
has two Maude backends: equivalences for causal attack trees, and the
multiplicative attack tree linear logic (MATLL).  The former is
essentially the exact same specification as the SPTool, but the latter
corresponds to the quaternary semantics defined in
Section~\ref{sec:a_quaternary_semantics_for_causal_attack_trees} and
Section~\ref{sec:specialization_in_the_quaternary_semantics};
specifically, this backend is defined as a rewrite system that
includes all of the rules from
Lemma~\ref{lemma:props_atree_ops_quaternary-semantics} and
Lemma~\ref{lemma:properties_of_entailment_in_the_quaternary_semantics}.

Attributed process attack trees are converted into the following
syntax:
\begin{center}
  \begin{math}
    \begin{array}{lll}
      \text{(Maude Formula)} & F := \mh{ID} \mid F1 ; F2 \mid F1.F2 \mid F1+F2
    \end{array}
  \end{math}
\end{center}
This is done by simply converting the internal attack tree into the
above syntactic form.  For example, the Maude formula for the
autonomous vehicle attack from Fig.~\ref{fig:vehicle_attack} is
$\mh{(0 ; (1 . 2)) || (5 ; ((6 || 7) ; 2))}$, where each integer
corresponds to the identifier of the base attacks.  Note that the base
attack $\mh{2}$ appears twice, this is because this base attack
appears twice in the original attack tree.  This syntax is then used
to write the Maude specification for the various backends.

The full Maude specification for the causal attack tree equivalence
checker can be found in
Appendix~\ref{sec:maude-spec-causal-attack-trees}.  However, Kordey et
al.'s specification only supports proving equivalences, but what about
specializations?  Lina supports proving specializations between attack
trees using the MATLL Maude backend.  Its full Maude specification can
be found in Fig.~\ref{fig:maude-spec-matll}.
\begin{figure}
\begin{mdframed}
\scriptsize
\begin{verbatim}
mod MATLL is

protecting LOOP-MODE .

sorts Formula .
subsort Nat < Formula .

op _||_  : Formula Formula -> Formula [ctor assoc comm] .
op _._   : Formula Formula -> Formula [ctor assoc comm prec 41] .
op _;_   : Formula Formula -> Formula [ctor assoc prec 40] .

var a b c d : Formula .

rl [a1]          : a . (b || c)       => (a . b) || (a . c) .
rl [a1Inv]       : (a . b) || (a . c) => a . (b || c) .
rl [a2]          : a ; (b || c)       => (a ; b) || (a ; c) .
rl [a2Inv]       : (a ; b) || (a ; c) => a ; (b || c) .
rl [a3]          : (b || c) ; a       => (b ; a) || (c ; a) .
rl [a3Inv]       : (b ; a) || (c ; a) => (b || c) ; a .
rl [a4]          : (a . b) ; c        => a . (b ; c) .
rl [a4Inv]       : a . (b ; c)        => (a . b) ; c .
rl [a5]          : (a ; b) . (c ; d)  => (a . c) ; (b . d) .
rl [a5Inv]       : (a . c) ; (b . d)  => (a ; b) . (c ; d) .
rl [switch]      : a ; (b . c)        => b . (a ; c) .
rl [seq-to-para] : a ; b              => a . b .
endm      
\end{verbatim}
\end{mdframed}
  \caption{Maude Specification for MATLL.}
  \label{fig:maude-spec-matll}
\end{figure}
The axioms \verb!a1! through \verb!a5! are actually equivalences, but
the last two rules are not.  At this point we can use these backends
to reason about attack trees.

The programmer can make queries to Lina by first importing one or more
Lina modules, and then making a query using Haskell's REPL -- read,
evaluate, print, loop -- called GHCi.  Consider the example Lina
program in Fig.~\ref{fig:lina-script-encrypt}. These are the attack
trees from Fig.~\ref{fig:2}.
\begin{figure}
  \begin{mdframed}
\begin{minted}[fontsize=\scriptsize]{haskell}
import Lina.AttackTree
import Lina.Maude.MATLL

-- A
enc_data1 :: PAttackTree String
enc_data1 = start_PAT $
  and_node "obtain secret"
    (or_node "obtain encryped file"
       (base_na "bribe sysadmin")
       (base_na "steal backup"))
    (seq_node "obtain password"
       (base_na "break into system")
       (base_na "install keylogger"))

-- C
enc_data2 :: PAttackTree String
enc_data2 = start_PAT $
  or_node "obtain secret"
    (and_node "obtain secret via sysadmin"
       (base_na "bribe sysadmin")
          (seq_node "obtain password"
             (base_na "break into system")
             (base_na "install keylogger")))
    (seq_node "break in, then obtain secrect"
       (base_na "break into system")
       (and_node "obtain secret from inside"
         (base_na "install keylogger")
         (base_na "steal backup")))
    \end{minted}
  \end{mdframed}
  \caption{Full Lina Script for the Attack Trees A and C from Fig.~\ref{fig:2}.}
  \label{fig:lina-script-encrypt}
\end{figure}
Then an example Lina session is as follows:
\begin{mdframed}\scriptsize
  \begin{verbatim}
> :load source/Lina/Examples/Specializations.hs
...
Ok, modules loaded
> is_specialization enc_data2 enc_data1
True
>     
  \end{verbatim}
\end{mdframed}
In this session we first load the Lina script from
Fig.~\ref{fig:lina-script-encrypt} which is stored in the file
\verb!Specializations.hs!.  Then we ask Lina if \verb!enc_data2! is a
specialization of \verb!enc_data1!, and Lina responds \verb!True!,
thus automating the proof given in Example~\ref{ex:ex1}.

In addition to reasoning about attack trees, Lina also support
analysis of attack trees.  Currently, Lina supports several types of
analysis: evaluating attack trees, querying the attack tree for the
attribute value of a node, projecting out the set of attacks from an
attack tree, and computing the maximal and minimal attack.

When one defines an attack tree that tree is left unevaluated; that
is, the attribute dictionary associated with the attack tree only has
attributes recorded for the base attacks.  If one wishes to know the
attribute values at the branching nodes, then one must evaluate the
attack tree, which populates the attribute dictionary with the missing
attributes.  For example, we may evaluate the attack tree for the
autonomous vehicle attack from Fig.~\ref{fig:vehicle_attack}, and
query the tree for the attributes at various nodes:
\begin{mdframed}
\scriptsize
\begin{verbatim}
> let (Right e_vat) = eval vehicle_AT
> e_vat <@> "social engineering attack"
0.6
> 
\end{verbatim}  
\end{mdframed}
Here we first evaluate the attack tree \mh{vehicle_AT} giving it the
name \mh{e_vat}, and then we use the attributed query combinator
\verb!<@>! to ask for the attribute at the parallel node labeled with
\mh{"social engineering attack"}. Note that the evaluator, \mh{eval},
uses the configuration associated with the attack tree to compute the
values at each branching node.

It is also possible to project out various attacks from an attack
tree.  In Lina an \emph{attack} corresponds to essentially an attack
tree with no choice nodes.  We call its data type \mh{Attack attribute
  label}.  An attack does not have any choice nodes, because they are
all split into multiple attacks; one for each child node.  For
example, the set of possible attacks for the autonomous vehicle attack
from Fig.~\ref{fig:vehicle_attack} can be found in
Fig.~\ref{fig:vehicle_attack_attacks}.
\begin{figure}  
  \begin{tabular}{|l|l|l|}
    \hline \\[-7px]    
    \begin{tabular}{l}
    \\[-9.8px]
    \begin{minipage}{\textwidth}        
      \begin{minted}[fontsize=\scriptsize]{haskell}
SEQ("external sensor attack",0.6)
	("modify street signs to cause wreck",0.2)
	(AND("social engineering attack",0.6)
		("pose as mechanic",0.6)
		("install malware",0.1))
      \end{minted}
      \vspace{2px}
    \end{minipage}
    \end{tabular}
    \\
    \hline\\[-7px]    
    \begin{tabular}{l}
    \\[-9.8px]
    \begin{minipage}{\textwidth}        
      \begin{minted}[fontsize=\scriptsize]{haskell}
SEQ("over night attack",0.8)
	("Find address where car is stored",0.05)
	(SEQ("compromise vehicle",0.8)
		("break window",0.8)
		("install malware",0.1))
      \end{minted}
      \vspace{2px}
    \end{minipage}
    \end{tabular}
    \\
    \hline\\[-7px]    
    \begin{tabular}{l}
    \\[-9.8px]
    \begin{minipage}{\textwidth}        
      \begin{minted}[fontsize=\scriptsize]{haskell}
SEQ("over night attack",0.5)
	("Find address where car is stored",0.05)
	(SEQ("compromise vehicle",0.5)
		("disable door alarm/locks",0.5)
		("install malware",0.1))
      \end{minted}
      \vspace{2px}
    \end{minipage}
    \end{tabular}\\
    \hline
  \end{tabular}
  \caption{Set of Possible Attacks for an Autonomous Vehicle Attack.}
  \label{fig:vehicle_attack_attacks}
\end{figure}
Lina can compute these automatically using the \mh{get_attacks}
command.  Finally, given the set of attacks for the autonomous vehicle
attack we can also compute the set of minimal and maximal attacks.
For example, consider the following session:
\begin{mdframed}
  \scriptsize
  \begin{verbatim}
> min_attacks.get_attacks $ vehicle_AT
[SEQ("over night attack",0.5)
	("Find address where car is stored",0.05)
	(SEQ("compromise vehicle",0.5)
		("disable door alarm/locks",0.5)
		("install malware",0.1))]
  \end{verbatim}
\end{mdframed}

In this session we first apply \mh{get_attacks} to \mh{vehicle_AT} to
compute the set of possible attacks, and then we compute the minimal
attack from this set.

%%% Local Variables: 
%%% mode: latex
%%% TeX-master: main.tex
%%% End: 

% section lina:_an_edsl_for_conducting_threat_analysis_using_causal_attack_trees (end)

\vspace{-7px}
\section{Conclusion and Future Work}
\label{sec:conclusion}
We made two main contributions: a new four-valued truth table
semantics of causal attack trees that supports specializations of
attack trees, and a new embedded, in Haskell, domain specific
programming language called Lina for specifying, reasoning, and
analyzing attack trees.

We plan to lift the quaternary semantics into a natural deduction
system based on the logic of bunched implications, and then study
proof search within this new system.  Lina is under active
development, and we have a number of extensions planned, for example,
adding support for attack-defense trees, attack(-defense) graphs,
attack nets, a GUI for viewing the various models, and a SMT backend.
Finally, it is necessary for number of case studies to be carried out
within Lina to be able to support the types of analysis required for
real world applications.  
% section conclusion (end)

\vspace{-7px}
\section{Acknowledgments}
\label{sec:acknowledgments}
This work was supported by NSF award \#1565557.
% section acknowledgments (end)


\bibliographystyle{plain}
\bibliography{ref}

\end{document}

%%% Local Variables: 
%%% mode: latex
%%% TeX-master: t
%%% End: 

