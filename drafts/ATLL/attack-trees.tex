We begin by introducing causal attack trees.  This formulation of
attack trees was first proposed by Jhawar et al. \cite{Jhawar:2015}
where they called them SAND attack trees, but sequential composition
does not always meet the same properties as conjunction, for example,
classically it is a self dual operator, thus, we follow Horne et al.'s
lead \cite{horne2017semantics} and call them causal attack trees.
\begin{definition}
  \label{def:atrees}
  Suppose $\mathbb{B}$ is a set of base attacks whose elements are
  denoted by $[[b]]$.  Then an \textbf{attack tree} is defined by
  the following grammar:
  \[
  \begin{array}{lll}
    [[A]],[[B]],[[C]],[[T]] := [[b]] \mid [[OR(A,B)]] \mid [[AND(A,B)]] \mid [[SEQ(A,B)]]\\
  \end{array}
  \]
  \noindent
  Equivalence of attack trees, denoted by $[[A ~ B]]$, is defined as
  follows:
  \begin{center}
    \begin{math} \footnotesize
      \begin{array}{|l|l|}
        \hline
        \begin{array}{lll}
          [[OR(A,A) ~ A]]\\
          [[OR(A,B) ~ OR(B,A)]]\\
          [[AND(A,B) ~ AND(B,A)]]\\
          \\\\
        \end{array}
        &
        \begin{array}{lll}          
          [[OR(OR(A,B),C) ~ OR(A,OR(B,C))]]\\
          [[AND(AND(A,B),C) ~ AND(A,AND(B,C))]]\\
          [[SEQ(SEQ(A,B),C) ~ SEQ(A,SEQ(B,C))]]\\                
          [[AND(A,OR(B,C)) ~ OR(AND(A,B),AND(A,C))]]\\
          [[SEQ(A,OR(B,C)) ~ OR(SEQ(A,B),SEQ(A,C))]]\\
        \end{array}\\
        \hline
      \end{array}
    \end{math}  
  \end{center}
\end{definition}
Throughout the sequel we will show that the previous rules are sound
with respect to our new model, but just as Horne et
al. \cite{horne2017semantics} we will then show that there are
properties of attack trees that these rules do not support, but our
semantics allows, for example, the rules given in
Lemma~\ref{lemma:properties_of_entailment_in_the_quaternary_semantics}
cannot be modeled using these rules.

%%% Local Variables: 
%%% mode: latex
%%% TeX-master: main.tex
%%% End: 
