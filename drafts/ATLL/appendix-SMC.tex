This appendix provides the definitions of both categories in general,
and, in particular, symmetric monoidal closed categories.  We begin
with the definition of a category:
\begin{definition}
  \label{def:category}
  A \textbf{category}, $\cat{C}$, consists of the following data:
  \begin{itemize}
  \item A set of objects $\cat{C}_0$, each denoted by $A$, $B$, $C$, etc.
  \item A set of morphisms $\cat{C}_1$, each denoted by $f$, $g$, $h$, etc.
  \item Two functions $\mathsf{src}$, the source of a morphism, and
    $\mathsf{tar}$, the target of a morphism, from morphisms to
    objects.  If $\mathsf{src}(f) = A$ and $\mathsf{tar}(f) = B$, then
    we write $f : A \to B$.
  \item Given two morphisms $f : A \to B$ and $g : B \to C$, then the
    morphism $f;g : A \to C$, called the composition of $f$ and $g$,
    must exist.
  \item For every object $A \in \cat{C}_0$, the there must exist a
    morphism $\id_A : A \to A$ called the identity morphism on $A$.

  \item The following axioms must hold:
    \begin{itemize}
    \item (Identities) For any $f : A \to B$, $f;\id_B = f = \id_A;f$.
    \item (Associativity) For any $f : A \to B$, $g : B \to C$, and $h
      : C \to D$, $(f;g);h = f;(g;h)$.
    \end{itemize}
  \end{itemize}
\end{definition}

Categories are by definition very abstract, and it is due to this that
makes them so applicable.  The usual example of a category is the
category whose objects are all sets, and whose morphisms are
set-theoretic functions.  Clearly, composition and identities exist,
and satisfy the axioms of a category.  A second example is preordered
sets, $(A , \leq)$, where the objects are elements of $A$ and a
morphism $f : a \to b$ for elements $a, b \in A$ exists iff $a \leq
b$. Reflexivity yields identities, and transitivity yields
composition.  

Symmetric monoidal categories pair categories with a commutative
monoid like structure called the tensor product.
\begin{definition}
  \label{def:monoidal-category}
  A \textbf{symmetric monoidal category (SMC)} is a category, $\cat{M}$,
  with the following data:
  \begin{itemize}
  \item An object $I$ of $\cat{M}$,
  \item A bi-functor $\otimes : \cat{M} \times \cat{M} \to \cat{M}$,
  \item The following natural isomorphisms:
    \[
    \begin{array}{lll}
      \lambda_A : I \otimes A \to A\\
      \rho_A : A \otimes I \to A\\      
      \alpha_{A,B,C} : (A \otimes B) \otimes C \to A \otimes (B \otimes C)\\
    \end{array}
    \]
  \item A symmetry natural transformation:
    \[
    \beta_{A,B} : A \otimes B \to B \otimes A
    \]
  \item Subject to the following coherence diagrams:\\
    \begin{math}
      \begin{array}{l}
        \bfig
        \square|amma|/->`->``/<1200,500>[
          ((A \otimes B) \otimes C) \otimes D`
          (A \otimes (B \otimes C)) \otimes D`
          (A \otimes B) \otimes (C \otimes D)`;
          \alpha_{A,B,C} \otimes \id_D`
          \alpha_{A \otimes B,C,D}g``]

        \square(0,-500)|amma|/`->``<-/<1200,500>[
          (A \otimes B) \otimes (C \otimes D)``
          A \otimes (B \otimes (C \otimes D))`
          A \otimes ((B \otimes C) \otimes D);`
          \alpha_{A,B,C \otimes D}``
          \id_A \otimes \alpha_{B,C,D}]       
      
      \morphism(1200,500)|m|<0,-1000>[
         (A \otimes (B \otimes C)) \otimes D`
         A \otimes ((B \otimes C) \otimes D);
         \alpha_{A,B \otimes C,D}]
      \efig
      \\
      \bfig
      \vSquares|ammmmma|/->`->`->``->`->`->/[
        (A \otimes B) \otimes C`
        (B \otimes A) \otimes C`
        A \otimes (B \otimes C)`
        B \otimes (A \otimes C)`
        (B \otimes C) \otimes A`
        B \otimes (C \otimes A);
        \beta_{A,B} \otimes \id_C`
        \alpha_{A,B,C}`
        \alpha_{B,A,C}``
        \beta_{A,B \otimes C}`
        \id_B \otimes \beta_{A,C}`
        \alpha_{B,C,A}]
      \efig\\      
        \end{array}
      \end{math}

    \begin{mathpar}
      \bfig
      \Vtriangle[
        (A \otimes I) \otimes B`
        A \otimes (I \otimes B)`
        A \otimes B;
        \alpha_{A,I,B}`
        \rho_{A}`
        \lambda_{B}]
      \efig
      \and
      \bfig
      \btriangle[
        A \otimes B`
        B \otimes A`
        A \otimes B;
        \beta_{A,B}`
        \id_{A \otimes B}`
        \beta_{B,A}]
      \efig
      \and
      \bfig
      \Vtriangle[
        I \otimes A`
        A \otimes I`
        A;
        \beta_{I,A}`
        \lambda_A`
        \rho_A]
      \efig
    \end{mathpar}    
  \end{itemize}
\end{definition}

Monoidal categories posses additional structure, and hence, ordinary
functors are not enough, thus, the notion must also be extended.
\begin{definition}
  \label{def:MCFUN}
  Suppose we are given two monoidal categories
  $(\cat{M}_1,\top_1,\otimes_1,\alpha_1,\lambda_1,\rho_1)$ and
  $(\cat{M}_2,\top_2,\otimes_2,\alpha_2,\lambda_2,\rho_2)$.  Then a
  \textbf{monoidal functor} is a functor $F : \cat{M}_1 \mto
  \cat{M}_2$, a map $m_{\top_1} : \top_2 \mto F\top_1$ and a natural transformation
  $m_{A,B} : FA \otimes_2 FB \mto F(A \otimes_1 B)$ subject to the
  following coherence conditions:
  \begin{mathpar}
    \footnotesize
    \bfig
    \vSquares|ammmmma|/->`->`->``->`->`->/[
      (FA \otimes_2 FB) \otimes_2 FC`
      FA \otimes_2 (FB \otimes_2 FC)`
      F(A \otimes_1 B) \otimes_2 FC`
      FA \otimes_2 F(B \otimes_1 C)`
      F((A \otimes_1 B) \otimes_1 C)`
      F(A \otimes_1 (B \otimes_1 C));
      {\alpha_2}_{FA,FB,FC}`
      m_{A,B} \otimes \id_{FC}`
      \id_{FA} \otimes m_{B,C}``
      m_{A \otimes_1 B,C}`
      m_{A,B \otimes_1 C}`
      F{\alpha_1}_{A,B,C}]
    \efig
    \end{mathpar}
  \begin{mathpar}
    \bfig
    \square|amma|/->`->`<-`->/<1000,500>[
      \top_2 \otimes_2 FA`
      FA`
      F\top_1 \otimes_2 FA`
      F(\top_1 \otimes_1 A);
      {\lambda_2}_{FA}`
      m_{\top_1} \otimes \id_{FA}`
      F{\lambda_1}_{A}`
      m_{\top_1,A}]
    \efig
    \and
    \bfig
    \square|amma|/->`->`<-`->/<1000,500>[
      FA \otimes_2 \top_2`
      FA`
      FA \otimes_2 F\top_1`
      F(A \otimes_1 \top_1);
      {\rho_2}_{FA}`
      \id_{FA} \otimes m_{\top_1}`
      F{\rho_1}_{A}`
      m_{A,\top_1}]
    \efig
    \end{mathpar}
\end{definition}
