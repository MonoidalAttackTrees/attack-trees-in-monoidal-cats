\newcommand{\forth}{\frac{1}{4}}
\newcommand{\half}{\frac{1}{2}}

Kordy et al.~\cite{Kordy:2012} gave a very elegant and simple
semantics of attack-defense trees in boolean algebras.  Unfortunately,
while their semantics is elegant it does not capture the resource
aspect of attack trees, it allows contraction, and it does not provide
a means to model sequential conjunction.  In this section we give a
semantics of attack trees in the spirit of Kordy et al.'s using a four
valued logic.

The propositional variables of our ternary logic, denoted by $A$, $B$,
$C$, and $D$, range over the set $\{0, \forth, \half, 1\}$.  We think
of $0$ and $1$ as we usually do in boolean algebras, but we think of
$\forth$ and $\half$ as intermediate values that can be used to break
various structural rules.  In particular we will use these values to
prevent exchange for sequential conjunction from holding, and
contraction from holding for parallel and sequential conjunction.
\begin{definition}
  \label{def:logical-connectives}
  The logical connectives of our four valued logic are defined as
  follows:
  \begin{itemize}
  \item[] Parallel Conjunction:
    \begin{center}
      \begin{math}
        \begin{array}{lll}
          A \odot_4 B = 1, \text{where neither $A$ nor $B$ are $0$}\\
          A \odot_4 B = 0, \text{otherwise}
        \end{array}
      \end{math}
    \end{center}
  \item[] Sequential Conjunction:
    \begin{center}
      \begin{math}
        \begin{array}{lll}
          \forth \rhd_4 B = \forth, \text{where $B \neq 0$}\\[2px]         
          A \rhd_4 B = \half, \text{where } A \in \{\half, 1\} \text{ and } B \neq 0\\
          A \rhd_4 B = 0, \text{otherwise}
        \end{array}
      \end{math}
    \end{center}
  \item[] Choice: $A \sqcup_4 B = \mathsf{max}(A,B)$    
  \end{itemize}
\end{definition}
These definitions are carefully crafted to satisfy the necessary
properties to model attack trees.  Comparing these definitions with
Kordy et al.'s~\cite{Kordy:2012} work we can see that choice is
defined similarly, but parallel conjunction is not a product --
ordinary conjunction -- but rather a linear tensor product, and
sequential conjunction is not actually definable in a boolean algebra,
and hence, makes heavy use of the intermediate values to insure that
neither exchange nor contraction hold.  The following results solidify
these claims.

We use the usual notion of equivalence between propositions, that is,
propositions $\phi$ and $\psi$ are considered equivalent, denoted by
$\phi \equiv \psi$, if and only if they have the same truth tables. In
order to model attack trees parallel conjunction must be symmetric,
associative, but not satisfy contraction.
\begin{lemma}[Parallel Conjunction is Symmetric]
  \label{lemma:parallel_conjunction_is_symmetric}
  For any $A$ and $B$, $A \odot_4 B \equiv B \odot A$.
\end{lemma}
\begin{proof}
  This proof holds by simply comparing truth tables.
\end{proof}

\begin{lemma}[Parallel Conjunction is Associative]
  \label{lemma:parallel_conjunction_is_associative}
  For any $A$, $B$, and $C$, $(A \odot_4 B) \odot_4 C \equiv A \odot_4 (B \odot_4 C)$.
\end{lemma}
\begin{proof}
  This proof holds by simply comparing truth tables.
\end{proof}

\begin{lemma}[Parallel Conjunction is not Contractive]
  \label{lemma:parallel_conjunction_is_not_contractive}
  It is not the case that for any $A$, $A \odot_4 A \equiv A$.
\end{lemma}
\begin{proof}
  Suppose $A = \forth$.  Then by definition $A \odot_4 A = 1$, but
  $\forth$ is not $1$.
\end{proof}
Similarly, sequential conjunction must be associative, but not
symmetric nor satisfy contraction.
\begin{lemma}[Sequential Conjunction is Associative]
  \label{lemma:sequential_conjunction_is_associative}
  For any $A$, $B$, and $C$, $(A \rhd_4 B) \rhd_4 C \equiv A \rhd_4 (B \rhd_4 C)$.
\end{lemma}
\begin{proof}
  This proof holds by simply comparing truth tables.
\end{proof}

\begin{lemma}[Sequential Conjunction is not Symmetric]
  \label{lemma:sequential_conjunction_is_symmetric}
  It is not the case that for any $A$ and $B$, $A \rhd_4 B \equiv B \rhd_4 A$.
\end{lemma}
\begin{proof}
  Suppose $A = \forth$ and $B = \half$.  Then $A \rhd_4 B = \forth$,
  but $B \rhd_4 A = \half$.
\end{proof}

\begin{lemma}[Sequential Conjunction is not Contractive]
  \label{lemma:sequential_conjunction_is_not_contractive}
  It is not the case that for any $A$, $A \rhd_4 A \equiv A$.
\end{lemma}
\begin{proof}
  Suppose $A = 1$.  Then by definition $A \odot A = \half$, but
  $\half$ is not $1$.
\end{proof}
Now choice satisfies all three properties, that is, it is symmetric,
associative, and does satisfy contraction.
\begin{lemma}[Choice is Symmetric]
  \label{lemma:choice_is_symmetric}
  For any $A$ and $B$, $A \sqcup_4 B \equiv B \sqcup_4 A$.
\end{lemma}
\begin{proof}
  This proof holds by simply comparing truth tables.
\end{proof}

\begin{lemma}[Choice is Associative]
  \label{lemma:choice_is_associative}
  For any $A$, $B$, and $C$, $(A \sqcup_4 B) \sqcup_4 C \equiv A \sqcup_4 (B \sqcup_4 C)$.
\end{lemma}
\begin{proof}
  This proof holds by simply comparing truth tables.
\end{proof}

\begin{lemma}[Choice is Contractive]
  \label{lemma:choice_is_contractive}
  For any $A$, $A \sqcup_4 A \equiv_4 A$.
\end{lemma}
\begin{proof}
  This proof holds by simply comparing truth tables.
\end{proof}
Finally, the necessary distributive laws hold.
\begin{lemma}[Parallel Conjunction Distributes Over Choice]
  \label{lemma:parallel_conjunction_distributes_over_choice}
  \begin{itemize}
  \item[i.] For any $A$, $B$, and $C$, $A \odot_4 (B \sqcup_4 C) \equiv (A \odot_4 B) \sqcup_4 (A \odot_4 C)$.
  \item[ii.] For any $A$, $B$, and $C$, $(A \sqcup_4 B) \odot_4 C \equiv (A \odot_4 C) \sqcup_4 (B \odot_4 C)$.
  \end{itemize}
\end{lemma}
\begin{proof}
  This proof holds by simply comparing truth tables.
\end{proof}

\begin{lemma}[Sequential Conjunction Distributes Over Choice]
  \label{lemma:sequential_conjunction_distributes_over_choice}
  \begin{itemize}
  \item[i.] For any $A$, $B$, and $C$, $A \rhd_4 (B \sqcup_4 C) \equiv (A \rhd_4 B) \sqcup_4 (A \rhd_4 C)$.
  \item[ii.] For any $A$, $B$, and $C$, $(A \sqcup_4 B) \rhd_4 C \equiv (A \rhd_4 C) \sqcup_4 (B \rhd_4 C)$.
  \end{itemize}
\end{lemma}
\begin{proof}
  This proof holds by simply comparing truth tables.
\end{proof}

At this point it is quite easy to model attack trees as formulas.  The
following defines their interpretation.
\begin{definition}
  \label{def:interp-aterms-ternary}
  Suppose $\mathbb{B}$ is some set of base attacks, and $\alpha :
  \mathbb{B} \mto \mathsf{PVar}$ is an assignment of base attacks to
  propositional variables.  Then we define the interpretation of
  $\mathsf{ATerms}$ to propositions as follows:
  \begin{center}
    \begin{math}
      \begin{array}{lll}
        \interp{[[b]] \in \mathbb{B}} & = & \alpha([[b]])\\
        \interp{[[AND T1 T2]]} & = & \interp{[[T1]]} \odot \interp{[[T2]]}\\
        \interp{[[SAND T1 T2]]} & = & \interp{[[T1]]} \rhd \interp{[[T2]]}\\
        \interp{[[OR T1 T2]]} & = & \interp{[[T1]]} \sqcup \interp{[[T2]]}\\
      \end{array}
    \end{math}
  \end{center}
\end{definition}
We can use this semantics to prove equivalences between attack trees.
\begin{lemma}[Equivalence of Attack Trees in the Ternary Semantics]
  \label{lemma:equivalence_of_attack_trees}
  Suppose $\mathbb{B}$ is some set of base attacks, and $\alpha :
  \mathbb{B} \mto \mathsf{PVar}$ is an assignment of base attacks to
  propositional variables.  Then for any attack trees $[[T1]]$ and
  $[[T2]]$, $[[T1 ~ T2]]$ if and only if $\interp{[[T1]]} \equiv \interp{[[T2]]}$.
\end{lemma}
\begin{proof}
  This proof holds by induction on the form of $[[T1 ~ T2]]$.
\end{proof}
This is a very simple and elegant semantics, but it also leads to a
more substantial theory.
