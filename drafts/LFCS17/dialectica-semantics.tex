In her thesis de Paiva \cite{?} gave one of the first sound and
complete categorical models, called dialectica categories, of full
intuitionistic linear logic.  Her models arose from giving a
categorical definition to G\"odel's Dialectica interpretation.  de
Paiva defines a particular class of dialectica categories called $GC$
over a base category $C$, see page 41 of \cite{dePaiva:1988}.  She
later showed that by instantiating $C$ to $\mathsf{Sets}$, the
category of sets and total functions, that one arrives at concrete
instantiation of $GC$ she called $\dial{2}$ whose objects are called
\emph{dialectica spaces}, and then she abstracts $\dial{2}$ into a
family of concrete dialectica spaces, $\dial{\text{$L$}}$, by replacing
$\mathsf{2}$ with an arbitrary lineale $L$.

In this section we construct the dialectica category, $\dial{4}$, and
show that it is a model of attack trees.  This will be done by
essentially lifting each of the attack tree operators defined for the
lineale semantics given in the previous section into the dialectica
category.  Working with dialectica categories can be very complex due
to the nature of how they are constructed.  In fact, they are one of
the few examples of theories that are easier to work with in a proof
assistant than outside of one.  Thus, throughout this section we only
give brief proof sketches, but the interested reader will find the
complete proofs in the formalization.


