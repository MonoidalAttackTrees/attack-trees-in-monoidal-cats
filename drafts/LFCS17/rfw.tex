\textbf{Related Work.}  Horne et al.~\cite{horne2017semantics} also
propose modeling SAND attack trees using linear logic, but they base
their work on pomsets and classical linear logic.  In addition, their
logic cannot derive the distributive law for sequential conjunction up
to an equivalence, but they can derive $[[* |- ((A > B) + (A > C)) -o
    (A > (B + C))]]$.  The full equivalence is derivable in ATLL
however.

The logic of bunched implications \cite{Ohearn:2003} has already been
shown to be able to support non-commutative operators by O'hearn, but
here we show how distributive laws can be controlled using properties
on contexts.

de Paiva~\cite{depaiva1991} shows how to model non-commutative
operators in dialectica categories, but here we show an alternative
way of doing this, and we extend the model to include more operators
like choice and its distributive laws.

\textbf{Future Work.}  We plan to build a term assignment for ATLL
that can be used a scripting language for defining and reasoning about
attack trees.  In addition, we plan to extend equivalence of attack
trees with contraction for choice, and to investigate adding a
modality that adds weakening to ATLL, and then, this modality could be
used to reason about subattack trees.  Finally, we leave the proof
theory of ATLL to future work.
