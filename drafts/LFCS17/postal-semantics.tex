Classical natural deduction has a semantics in boolean algebras, and
so the semantics in the previous section begs the question of whether
there is a natural deduction system that can be used to reason about
attack trees.  We answer this question in the positive, but before
defining the logic we first build up a non-trivial concrete
categorical model of our desired logic in dialectica spaces, but this
first requires the refinement of the quaternary semantics into a
postal semantics we call the lineale semantics of SAND attack trees.
This semantics will live at the base of the dialectica space model
given in the next section.

\newcommand{\Four}[0]{\mathsf{Four}} We denote by $\leq_4 : \Four
\times \Four \to \Four$ the obvious preorder on $\Four$ making
$(\Four, \leq_4)$ a preordered set.  It is well known that every
preordered set induces a category whose objects are the elements of
the carrier set, here $\Four$, and morphisms $\Hom{\Four}{a}{b} = a
\leq_4 b$.  Composition of morphisms hold by transitivity and
identities exists by reflexivity.  Under this setting it is
straightforward to show that for any propositions $\phi$ and $\psi$
over $\Four$ we have $\phi \equiv \psi$ if and only if $\phi \leq_4
\psi$ and $\psi \leq_4 \psi$.  Thus, every result proven for the
logical connectives on $\Four$ in the previous section induce
properties on morphisms in this setting.

In addition to the induced properties just mentioned we also have the
following new ones which are required when lifting this semantics to
dialectica spaces, but are also important when building a
corresponding logic.
\begin{lemma}[Parallel Conjunction is Functorial]
  \label{lemma:parallel_conjunction_is_functorial}
  For any $a,b,c,d \in \Four$, if $a \leq_4 c$ and $b \le_4 d$, then
  $(a \odot_4 b) \leq_4 (c \odot_4 d)$.
\end{lemma}
\begin{proof}
  This proof holds by a straightforward case analysis on $a$, $b$,
  $c$, and $d$.  In any cases where $(a \odot_4 b) \leq_4 (c \odot_4
  d)$ does not hold, then one of the premises will also not hold.
\end{proof}

\begin{lemma}[Sequential Conjunction is Functorial]
  \label{lemma:sequential_conjunction_is_functorial}
  For any $a,b,c,d \in \Four$, if $a \leq_4 c$ and $b \le_4 d$, then
  $(a \rhd_4 b) \leq_4 (c \rhd_4 d)$.
\end{lemma}
\begin{proof}
  This proof is similar to the previous result.
\end{proof}

\begin{lemma}[Choice is Functorial]
  \label{lemma:parallel_conjunction_is_functorial}
  For any $a,b,c,d \in \Four$, if $a \leq_4 c$ and $b \le_4 d$, then
  $(a \sqcup_4 b) \leq_4 (c \sqcup_4 d)$.
\end{lemma}
\begin{proof}
  This proof is similar to the previous result.
\end{proof}

The logic we a building up is indeed intuitionistic, but none of the
operators we have introduced thus far are closed, but we can define
the standard symmetric linear tensor product in $\Four$ that is
closed.
\begin{definition}
  \label{def:tensor-and-implication}
  The following defines the linear tensor product on $\Four$ as well
  as linear implication:
  \begin{itemize}
  \item[] Tensor Product:
    \begin{center}
      \begin{math}
        \begin{array}{lll}
          A \otimes_4 B = \mathsf{max}(A,B), \text{where $A$ nor $B$ are $0$}\\
          A \otimes_4 B = 0, \text{otherwise}
        \end{array}
      \end{math}
    \end{center}
    The unit of the tensor product is $I_4 = \forth$.\\
    
  \item[] Linear Implication:
    \begin{center}
      \begin{math}
        \begin{array}{lll}
          A \limp_4 B = 0, \text{where $B <_4 A$}\\
          A \limp_4 A = A, \text{where $A \in \{\forth,\half\}$}\\
          A \limp_4 B = 1, \text{otherwise}
        \end{array}
      \end{math}
    \end{center}
  \end{itemize}
\end{definition}
The expected monoidal properties hold for the tensor product.
\begin{lemma}[Tensor is Symmetric Monoidal]
  \label{lemma:tensor_is_symmetric_monoidal}
  \begin{itemize}
  \item[] (Symmetry) For any $A$ and $B$, $A \otimes_4 B \equiv B \otimes A$.
  \item[] (Associativity) For any $A$, $B$, and $C$, $(A \otimes_4 B) \otimes_4 C \equiv A \otimes_4 (B \otimes_4 C)$.
  \item[] (Left Unitor) For any $A$, $(A \otimes I_4) \equiv A$.
  \item[] (Right Unitor) For any $A$, $(I_4 \otimes A) \equiv A$.
  \item[] (Functorality) For any $A$, $B$, $C$, and $D$, if $A \leq_4 C$ and $B \le_4 D$, then
    $(A \otimes_4 B) \leq_4 (C \otimes_4 D)$.
  \end{itemize}
\end{lemma}
