In this section we introduce SAND attack trees.  This formulation of
attack trees was first proposed by Jhawar et al. \cite{Jhawar:2015}.
\begin{definition}
  \label{def:atrees}
  Suppose $\mathsf{B}$ is a set of base attacks whose elements are
  denoted by $[[b]]$.  Then an \textbf{attack tree} is defined by
  the following grammar:
  \[
  \begin{array}{lll}
    [[A]],[[B]],[[C]],[[T]] := [[b]] \mid [[OR(A,B)]] \mid [[AND(A,B)]] \mid [[SAND(A,B)]]\\
  \end{array}
  \]
  \noindent
  Equivalence of attack trees, denoted by $[[A ~ B]]$, is defined as
  follows:
  \begin{center}
    \begin{math}
      \setlength{\arraycolsep}{10px}
      \begin{array}{lll}
        (\text{E}_1) & [[OR(OR(A,B),C) ~ OR(A,OR(B,C))]]\\
        (\text{E}_2) & [[AND(AND(A,B),C) ~ AND(A,AND(B,C))]]\\
        (\text{E}_3) & [[SAND(SAND(A,B),C) ~ SAND(A,SAND(B,C))]]\\
        (\text{E}_4) & [[OR(A,B) ~ OR(B,A)]]\\
        (\text{E}_5) & [[AND(A,B) ~ AND(B,A)]]\\
        (\text{E}_6) & [[AND(A,OR(B,C)) ~ OR(AND(A,B),AND(A,C))]]\\
        (\text{E}_7) & [[SAND(A,OR(B,C)) ~ OR(SAND(A,B),SAND(A,C))]]\\
      \end{array}
    \end{math}
  \end{center}
\end{definition}
This definition of SAND attack trees differs slightly from Jhawar
et. al.'s~\cite{Jhawar:2015} definition.  They define $n$-ary
operators, but we only consider the binary case, because it fits better
with the models presented here and it does not loose any generality
because we can model the $n$-ary case using binary operators in the
obvious way.  Finally, they also include the equivalence $[[OR(A,A) ~
    A]]$, but it is not obvious how to include this in the models
presented here and we leave its addition to future work.

%%% Local Variables: 
%%% mode: latex
%%% TeX-master: main.tex
%%% End: 
