\usepackage{amsmath,amssymb,amsthm}

\usepackage{fullpage}
\usepackage{hyperref}
\usepackage{color}
\usepackage[barr]{xy}
\usepackage{todonotes}
\usepackage{tikz}
\usepackage{tikz-qtree}
\usepackage{stmaryrd}
\usepackage{mathpartir}
\usepackage{layout}
\usepackage[margin=1.1in]{geometry}

%% \usepackage{nobibprint}

\newtheorem{thm}{Theorem}
\newtheorem{lemma}[thm]{Lemma}
\newtheorem{corollary}[thm]{Corollary}
\newtheorem{definition}[thm]{Definition}
\newtheorem{remark}[thm]{Remark}
\newtheorem{proposition}[thm]{Proposition}
\newtheorem{notn}[thm]{Notation}
\newtheorem{observation}[thm]{Observation}

\newcommand{\cat}[1]{\mathbb{#1}}
\newcommand{\catobj}[1]{\mathsf{Obj}(\cat{#1})}
\newcommand{\catop}[1]{\mathbb{#1}^{\mathsf{op}}}
\newcommand{\sets}[0]{\mathsf{Sets}}
\newcommand{\homs}[2]{\mathsf{Hom}[#1,#2]}
\newcommand{\cur}[0]{\mathsf{cur}}
\newcommand{\curi}[0]{\mathsf{cur}^{-1}}
\newcommand{\app}[0]{\mathsf{app}}
\newcommand{\id}[0]{\mathsf{id}}
\newcommand{\injl}[0]{\mathsf{inj_l}}
\newcommand{\injr}[0]{\mathsf{inj_r}}
\newcommand{\pow}[1]{\mathcal{P}(#1)}
\newcommand{\cpy}[0]{\textcopyright}
\newcommand{\lett}[3]{\mathsf{let}\,#1 = #2\,\mathsf{in}\,#3}

% Commands that are useful for writing about type theory and programming language design.
\newcommand{\case}[4]{\text{case}\ #1\ \text{of}\ #2\text{.}#3\text{,}#2\text{.}#4}
\newcommand{\interp}[1]{[\negthinspace[#1]\negthinspace]}
\newcommand{\normto}[0]{\rightsquigarrow^{!}}
\newcommand{\join}[0]{\downarrow}
\newcommand{\redto}[0]{\rightsquigarrow}
\newcommand{\nat}[0]{\mathbb{N}}
\newcommand{\terms}[0]{\mathcal{T}}
\newcommand{\fun}[2]{\lambda #1.#2}
\newcommand{\CRI}[0]{\text{CR-Norm}}
\newcommand{\CRII}[0]{\text{CR-Pres}}
\newcommand{\CRIII}[0]{\text{CR-Prog}}
\newcommand{\subexp}[0]{\sqsubseteq}
\newcommand{\napprox}[2]{\lfloor #1 \rfloor_{#2}}
\newcommand{\interpset}{\mathcal{I}}
\newcommand{\powerset}[1]{\mathcal{P}(#1)}
\newcommand{\vinterp}[1]{\mathcal{V}[\negthinspace[#1]\negthinspace]}
\newcommand{\vbinterp}[2]{\bar{\mathcal{V}}_{#1}[\negthinspace[#2]\negthinspace]}
\newcommand{\ginterp}[1]{\mathcal{G}[\negthinspace[#1]\negthinspace]}
\newcommand{\dinterp}[1]{\mathcal{D}[\negthinspace[#1]\negthinspace]}
\newcommand{\tinterp}[1]{\mathcal{T}[\negthinspace[#1]\negthinspace]}


\title{\vspace{-50px}Summer Research Proposal: \\ Marrying Gradual and Linear Types \\ The Attack Tree Linear Logic (ATLL)}
\author{Harley Eades III, Computer Science, Augusta University}
\date{\vspace{-22px}}
\begin{document}
\maketitle  

\begin{full}
%% \layout
\section{Overview}
\label{sec:overview}

This proposal seeks summer faculty salary support from the Hull
College of Business for the months of June and July to support two
cutting edge research projects in computer science and cyber security.
I briefly describe these projects before discussing the summer
activities that the requested funds will support.

\section{Marrying Gradual and Linear Types}
\label{sec:marrying_gradual_and_linear_types}
Gradual typing is a new area of research in the theory of programming
languages that impacts much of the technical industry.  Linear types
provide a means of specifying and broadening computer programs to be
more safe when dealing with input from the outside world.  This
project will be the first to bring these two paradigms together
providing the best of both worlds.

Over the course of the summer my trainee, a Ph.D. student from the
University of Iowa, and I will develop a new programming language that
combines gradual and linear types.  We will then mathematically prove
the correctness of this programming language inside a new cutting edge
system called Agda which is a proof assistant for conducting machine
checked mathematical proofs.  This will certify that our programming
language meets all of the desired properties solidifying its
correctness.
% section marrying_gradual_and_linear_types (end)

\section{The Attack Tree Linear Logic (ATLL)}
\label{sec:the_attack_tree_linear_logic_(atll)}
\textbf{Attack trees} are a modeling tool, originally proposed by
Bruce Schneier \cite{Schneier:1999}, which are used to assess the
threat potential of a security critical system.  Attack trees have
since been used to analyze the threat potential of many types of
security critical systems, for example, cybersecurity of power grids
\cite{Ten:2007}, wireless networks \cite{Reinhardt:2012}, and many
others.  Attack trees consists of several goals, usually specified in
English prose, for example, ``compromise safe'' or ``obtain
administrative privileges'', where the root is the ultimate goal of
the attack and each node coming off of the root is a refinement of the
main goal into a subgoal.  Then each subgoal can be further refined.
The leaves of an attack tree make up the set of base attacks.  Subgoals
can be either disjunctively or conjunctively combined.

\textbf{The need for a foundation.}  Attack trees for real-world
security scenarios can grow to be quite complex.  The attack tree
presented in \cite{Ten:2007} to access the security of power grids has
twenty-nine nodes with sixty counter measures attached to the nodes
throughout the tree.  The details of the tree spans several pages of
appendix.  The attack tree developed for the border gateway protocol
has over a hundred nodes \cite{Convey:2003}, and the details of the
tree spans ten pages.  Manipulating such large trees without a formal
semantics can be dangerous.

%% Semantics of ATREES:
%%   - Boolean logic
%%   - Attack nets (2000)
%%   - Multisets (2006)
%%   - Series-parallel graphs (extension of multisets) (2015)
\textbf{The formal semantics of attack trees.} The leading question
the field is seeking to answer by giving a mathematical foundation to
attack trees is ``what is an attack tree?''  There have been numerous
attempts at answering this question.  However, the research on the
mathematical foundation of attack trees being done here at Augusta
University\footnote{\label{grant}The National Science Foundation CRII CISE Research Initiation grant,
  ``CRII:SHF: A New Foundation for Attack Trees Based on Monoidal
  Categories``, under Grant No. 1565557.} is the first of its kind,
and the first to propose the use of linear logic.

This summer my trainee, a visiting Ph.D. student from North Carolina
State University, will aid me in developing a new system for reasoning
about attack trees and conducting threat analysis using attack trees
called the Attack Tree Linear Logic (ATLL).  This system is the first
of its kind and has the potential to greatly impact the cyber security
research community.

% section the_attack_tree_linear_logic_(atll) (end)

\section{The Proposal}
\label{sec:the_proposal}
I will be hosting two visiting Ph.D. students from June 1 till Augusta
1.  They will be fully funded from my NSF grant, fn.~\ref{grant}, but
this grant does not include PI salary support.

The amount I am requesting is \$10,000. This will cover my salary for
the summer months of June and July.  The amount of effort required for
the proper mentorship and research development necessary for this
summer is at least as much effort as teaching a summer course. Lastly,
this research program benefits, not only myself, but the college and
university as a whole, because I will be mentoring two visiting
Ph.D. students as well as building relationships with other
universities through this mentorship.

In addition, I will be holding a summer research seminar that will be
open to Hull College of Business students as well as three graduate
students and two postdocs from the University of Iowa.  The project
will begin on June 1 and end on July 30.

At the completion of the summer project I will happily provide drafts
of the two papers these projects will produce to the college and/or
present the results and activities to the college during a brown bag.
% section the_proposal (end)


\end{full}
%% \nocite{*}
\bibliographystyle{plain}
\bibliography{refs}



