The ideal semantics for casual attack trees was first proposed by
Horne et al.\cite{horne2017semantics}.  In this section we give a
simple truth table semantics that corresponds to their ideal semantics
within the ideal quaternary logic.

\begin{definition}
  \label{def:ideal-semantics}
  The logical connectives of the \emph{ideal quaternary logic} are
  defined as follows:\vspace{-5px}
  \begin{center}
    \begin{math}
      \setlength{\arraycolsep}{5px}
      \begin{array}{lll}
        \begin{array}{lll}
          \text{Parallel Composition:}\\
          \begin{array}{lll}
            P \odot_I Q = 1,\\
            \,\,\,\,\,\,\,\text{where neither $P$ nor $Q$ are $0$}\\
            P \odot_I Q = 0, \text{otherwise}\\\\\\
          \end{array}
        \end{array}
        &
        \begin{array}{lll}
          \text{Sequential Composition:}\\
          \begin{array}{lll}          
            P \rhd_I Q = \half,\\
            \,\,\,\,\,\,\,\text{where } P \in \{\half, 1\} \text{ and } Q \neq 0\\            
            P \rhd_I Q = \forth,\\
            \,\,\,\,\,\,\,\text{where } P = \forth \text{ and } Q \neq 0\\
            P \rhd_I Q = 0, \text{otherwise}\\
          \end{array}
        \end{array}
        \\[30px]
        \begin{array}{lll}
          \text{Choice:}\\    
          \begin{array}{lll}
            P \sqcup_I Q = \mathsf{max}(P,Q)
          \end{array}
        \end{array}
      \end{array}
    \end{math}
  \end{center}        
\end{definition}
These definitions are carefully crafted to satisfy the necessary
properties to model attack trees on the ideal semantics.  Comparing
these definitions with Kordy et al.'s~\cite{Kordy:2012} work we can
see that choice is defined similarly, but parallel composition is not
a product -- ordinary conjunction -- but rather a linear tensor
product. Sequential composition is not actually definable in a boolean
algebra, and hence makes use of the intermediate values to insure that
neither exchange nor contraction hold.

In order to model attack trees, the previously defined logical
connectives must satisfy the appropriate equivalences corresponding to
the equations between attack trees.  We break these properties up into
the following lemmata.

\begin{lemma}[Basic Properties for Choice]
  \label{lemma:basic_properties_for_choice}
  The following properties hold:
  \begin{enumerate}
  \item $(P \sqcup_I Q) \equiv (Q \sqcup_I P)$\\[-5px]
  \item $((P \sqcup_I Q) \sqcup_I R) \equiv (P \sqcup_I (Q \sqcup_I R))$\\[-5px]
  \item $P \leq_4 (P \sqcup_I Q)$\\[-5px]
  \item $Q \leq_4 (P \sqcup_I Q)$\\[-5px]
  \item $\text{If }P \leq_4 R \text{ and } Q \leq_4 R \text{, then } (P \sqcup_I Q) \leq_4 R$\\[-5px]
  \item $\text{If }P \leq_4 R \text{ and } Q \leq_4 S \text{, then } (P \sqcup_I Q) \leq_4 (R \sqcup_I S)$
  \end{enumerate}
\end{lemma}
\begin{proof}
  Each of the properties hold by comparing truth tables.
\end{proof}
The previous lemma shows that choice has the same properties as
boolean disjunction.  Hence, it is possible to show using these rules
that $P \sqcup_I P \equiv P$ which follows from properties three,
four, and five.
\begin{lemma}[Basic Properties for Parallel Composition]
  \label{lemma:basic_properties_for_parallel}
  The following properties hold:
  \begin{enumerate}
  \item $(P \odot_I P) \not\equiv P$\\[-5px]
  \item $(P \odot_I Q) \equiv (Q \odot_I P)$\\[-5px]
  \item $((P \odot_I Q) \odot_I R) \equiv (P \odot_I (Q \odot_I R))$\\[-5px]
  \item $(P \odot_I (Q \sqcup_I R)) \equiv ((P \odot_I Q) \sqcup_I (P \odot_I R))$\\[-5px]
  \item $\text{If }P \leq_4 R \text{ and } Q \leq_4 S \text{, then } (P \odot_I Q) \leq_4 (R \odot_I S)$
  \end{enumerate}
\end{lemma}
\begin{proof}
  We give the proof of property one.  The other properties hold by
  comparing truth tables.  Suppose $P = \half$, then $P \odot_I P =
  \half \odot_I \half = 1$, but $1$ is not $\half$.
\end{proof}
The previous lemma shows that sequential composition is a linear
tensor product.  In particular, the first property guarantees that
sequential composition does not contract parallel copies of attack
trees into a single attack tree.
\begin{lemma}[Basic Properties for Sequential Composition]
  \label{lemma:basic_properties_for_parallel}
  The following properties hold:
  \begin{enumerate}
  \item $(P \rhd_I P) \not\equiv P$\\[-5px]
  \item $(P \rhd_I Q) \not\equiv (Q \rhd_I P)$\\[-5px]
  \item $(P \rhd_I (Q \rhd_I R)) \equiv ((P \rhd_I Q) \rhd_I R)$\\[-5px]
  \item $(P \rhd_I (Q \sqcup_I R)) \equiv ((P \rhd_I Q) \sqcup_I (P \rhd_I R))$\\[-5px]
  \item $\text{If }P \leq_4 R \text{ and } Q \leq_4 S \text{, then } (P \rhd_I Q) \leq_4 (R \rhd_I S)$
  \end{enumerate}
\end{lemma}
\begin{proof}
  We give proofs for properties one and two, but the others hold by
  comparing truth tables.  As for property one, suppose $P = 1$, then
  $P \rhd_I P = 1 \rhd_I 1 = \half$, but $1$ is not $\half$.  Now for
  property two, suppose $P = 1$ and $Q = \forth$, then $P \rhd_I Q = 1
  \rhd_I \forth = \half$, but $Q \rhd_I P = \forth \rhd_I 1 = \forth$.
\end{proof}
This lemma is similar to the previous.  However, property two
guarantees that sequential composition is not commutative.
\begin{lemma}[The Ideal Properties]
  \label{lemma:the_ideal_properties}
  The following properties hold:
  \begin{enumerate}
  \item $((P \sqcup_I Q) \rhd_I (R \sqcup_I S)) \leq_4 ((P \rhd_I R) \sqcup_I (Q \rhd_I S))$\\[-5px]
  \item $((P \sqcup_I Q) \rhd_I R) \leq_4 (P \sqcup_I (Q \rhd_I R))$\\[-5px]
  \item $(P \rhd_I (Q \sqcup_I R) \leq_4 (Q \sqcup_I (P \rhd_I R))$\\[-5px]
  \item $(P \rhd_I Q) \leq_4 (P \sqcup_I Q)$
  \end{enumerate}
\end{lemma}
\begin{proof}
  Each property holds by comparing truth tables.
\end{proof}

At this point it is quite easy to model attack trees as formulas.  The
following defines their interpretation.
\begin{definition}
  \label{def:interp-aterms-quaternary}
  Suppose $\mathbb{B}$ is some set of base attacks, and $\nu :
  \mathbb{B} \mto \mathsf{PVar}$ is an assignment of base attacks to
  propositional variables.  Then we define the interpretation of
  attack trees to propositions as follows:
  \begin{center}
    \begin{math}
      \setlength{\arraycolsep}{5px}
      \begin{array}{lll}
        \begin{array}{lll}
          \interp{[[b]] \in \mathbb{B}} & = & \nu([[b]])\\
          \interp{[[AND(A, B)]]} & = & \interp{[[A]]} \odot_I \interp{[[B]]}
        \end{array}
        &
        \begin{array}{lll}
          \interp{[[SEQ(A,B)]]} & = & \interp{[[A]]} \rhd_I \interp{[[B]]}\\
          \interp{[[OR(A,B)]]} & = & \interp{[[A]]} \sqcup_I \interp{[[B]]}
        \end{array}
      \end{array}
    \end{math}
  \end{center}
\end{definition}
We can use this semantics to prove equivalences between attack trees.
\begin{lemma}[Equivalence of Attack Trees in the Ideal Quaternary Semantics]
  \label{lemma:equivalence_of_attack_trees}
  Suppose $\mathbb{B}$ is some set of base attacks, and $\nu :
  \mathbb{B} \mto \mathsf{PVar}$ is an assignment of base attacks to
  propositional variables.  Then for any attack trees $[[A]]$ and
  $[[B]]$, if $[[A ~ B]]$, then $\interp{[[A]]} \equiv
  \interp{[[B]]}$.
\end{lemma}
\begin{proof}
  This proof holds by induction on the form of $[[A ~ B]]$.
\end{proof}

%%% Local Variables: 
%%% mode: latex
%%% TeX-master: main.tex
%%% End: 
