\newcommand{\forth}{\frac{1}{4}}
\newcommand{\half}{\frac{1}{2}}

Kordy et al.~\cite{Kordy:2012} gave a very elegant and simple
semantics of attack-defense trees in Boolean algebras.  Unfortunately,
while their semantics is elegant, it does not capture the resource
aspect of attack trees, it allows contraction, and it does not provide
a means to model sequential composition.  In this section we give a
semantics of attack trees in the spirit of Kordy et al.'s using a
four-valued logic.  This section was formally verified in the Agda
Proof Assistant~\cite{Norell:2009}\footnote{The formalization can be
  found at
  \url{https://github.com/MonoidalAttackTrees/ATLL-Formalization}}.

 We now give two types of quaternary semantics for casual attack
 trees.  We do this by defining two four-valued logics we call
 quaternary logics.  The propositional variables, elements of the set
 $\mathsf{PVar}$, of our quaternary logics, denoted by $P$, $Q$, $R$,
 and $S$, range over the set $\mathsf{4} = \{0, \forth, \half, 1\}$.
 We think of $0$ and $1$ as we usually do in boolean algebras, but we
 think of $\forth$ and $\half$ as intermediate values that can be used
 to break various structural rules\footnote{Choosing $\forth$ and
   $\half$ as the symbols for the intermediate values was arbitrary,
   and one can choose any symbols at all for these two values and the
   semantics will still be correct.}.  In particular we will use these
 values to prevent exchange for sequential composition from holding,
 and contraction from holding for parallel and sequential composition.

We use the usual notion of equivalence between propositions; that is,
propositions $\phi$ and $\psi$ are considered equivalent, denoted by
$\phi \equiv \psi$, if and only if they have the same truth tables.
In addition, we define a notion of entailment for the quaternary
logics.  Denote by $P \leq_4 Q$ the usual natural number ordering
restricted to $\mathsf{4}$.  Then we have the following result
immediately.
\begin{lemma}[Entailment in the Quaternary Logics]
  \label{lemma:entailment_in_the_quaternary_semantics}
  $P \equiv Q$ if and only if $P \leq_4 Q$ and $Q \leq_4 P$
\end{lemma}
This result shows that we can break up the equivalence of attack trees
into directional properties captured here by entailments, and hence,
every equivalence proved throughout this section can also be used
directionally.

\subsection{The Ideal Quaternary Logic}
\label{subsec:the_ideal_quaternary_semantics}
\input{ideal-output}
% subsection the_ideal_quaternary_semantics (end)

\subsection{The Filterish Quaternary Logic}
\label{subsec:the_filterish_quaternary_semantics}
\input{filter-output}
% subsection the_filterish_quaternary_semantics (end)

\subsection{An Example Specialization}
\label{subsec:an_example_specialization}
\input{example-output}
% subsection an_example_specialization (end)

%%% Local Variables: 
%%% mode: latex
%%% TeX-master: main.tex
%%% End: 
