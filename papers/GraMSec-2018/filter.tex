We now introduce the filterish semantics for casual attack trees.
This is a restricted notion of the filter semantics of Horne et
al.~\cite{horne2017semantics}.  We were unable to find a quaternary
semantics for the full filter semantics, because we obtained
contractions when attempting to satisfy the corresponding
specialization properties in the filter model.  We are unsure if these
contradictions arise due to the fact that the semantics proposed here
is intuitionistic while Horne et al.~\cite{horne2017semantics} use
classical logic, or if four values just are not enough, or if we just
have not been able to find it.

In this section we do as we did in the previous and define a
quaternary logic called the \emph{filterish quaternary logic}.
\begin{definition}
  \label{def:filterish-semantics}
  The logical connectives of the \emph{filterish quaternary logic} are
  defined as follows:\vspace{-5px}
  \begin{center}
    \begin{math}
      \setlength{\arraycolsep}{5px}
      \begin{array}{lll}
        \begin{array}{lll}
          \text{Parallel Composition:}\\
          \begin{array}{lll}
            P \odot_F Q = \half,\\
            \,\,\,\,\,\,\,\text{where neither $P$ nor $Q$ are $0$}\\
            P \odot_F Q = 0, \text{otherwise}\\\\\\
          \end{array}
        \end{array}
        &
        \begin{array}{lll}
          \text{Sequential Composition:}\\
          \begin{array}{lll}          
            P \rhd_F Q = 1,\\
            \,\,\,\,\,\,\,\text{where } P \in \{\half, 1\} \text{ and } Q \neq 0\\
            P \rhd_F Q = \forth,\\
            \,\,\,\,\,\,\,\text{where } P = \forth  \text{ and } Q \neq 0\\
            P \rhd_F Q = 0, \text{otherwise}
          \end{array}
        \end{array}
        \\[30px]
        \begin{array}{lll}
          \text{Choice:}\\    
          \begin{array}{lll}
            P \sqcup_F Q = \mathsf{max}(P,Q)
          \end{array}
        \end{array}
      \end{array}
    \end{math}
  \end{center}        
\end{definition}
We have the same basic properties as the ideal quaternary logic.  We
omit proofs, because they are similar to the corresponding properties
in the ideal semantics.
\begin{lemma}[Basic Properties for Choice]
  \label{lemma:basic_properties_for_choice}
  The following properties hold:
  \begin{enumerate}
  \item $(P \sqcup_F Q) \equiv (Q \sqcup_F P)$\\[-5px]
  \item $((P \sqcup_F Q) \sqcup_F R) \equiv (P \sqcup_F (Q \sqcup_F R))$\\[-5px]
  \item $P \leq_4 (P \sqcup_F Q)$\\[-5px]
  \item $Q \leq_4 (P \sqcup_F Q)$\\[-5px]
  \item $\text{If }P \leq_4 R \text{ and } Q \leq_4 R \text{, then } (P \sqcup_F Q) \leq_4 R$\\[-5px]
  \item $\text{If }P \leq_4 R \text{ and } Q \leq_4 S \text{, then } (P \sqcup_F Q) \leq_4 (R \sqcup_F S)$
  \end{enumerate}
\end{lemma}

\begin{lemma}[Basic Properties for Parallel Composition]
  \label{lemma:basic_properties_for_parallel}
  The following properties hold:
  \begin{enumerate}
  \item $(P \odot_F P) \not\equiv P$\\[-5px]
  \item $(P \odot_F Q) \equiv (Q \odot_F P)$\\[-5px]
  \item $((P \odot_F Q) \odot_F R) \equiv (P \odot_F (Q \odot_F R))$\\[-5px]
  \item $(P \odot_F (Q \sqcup_F R)) \equiv ((P \odot_F Q) \sqcup_F (P \odot_F R))$\\[-5px]
  \item $\text{If }P \leq_4 R \text{ and } Q \leq_4 S \text{, then } (P \odot_F Q) \leq_4 (R \odot_F S)$
  \end{enumerate}
\end{lemma}

\begin{lemma}[Basic Properties for Sequential Composition]
  \label{lemma:basic_properties_for_parallel}
  The following properties hold:
  \begin{enumerate}
  \item $(P \rhd_F P) \not\equiv P$\\[-5px]
  \item $(P \rhd_F Q) \not\equiv (Q \rhd_F P)$\\[-5px]
  \item $(P \rhd_F (Q \rhd_F R)) \equiv ((P \rhd_F Q) \rhd_F R)$\\[-5px]
  \item $(P \rhd_F (Q \sqcup_F R)) \equiv ((P \rhd_F Q) \sqcup_F (P \rhd_F R))$\\[-5px]
  \item $\text{If }P \leq_4 R \text{ and } Q \leq_4 S \text{, then } (P \rhd_F Q) \leq_4 (R \rhd_F S)$
  \end{enumerate}
\end{lemma}
We now give the filterish properties that correspond to a subset of
the filter properties proposed by Horne et
al.~\cite{horne2017semantics}.
\begin{lemma}[The Filterish Properties]
  \label{lemma:the_filterish_properties}
  The following properties hold:
  \begin{enumerate}
  \item $((P \rhd_F R) \odot_F (Q \rhd_F S)) \leq_4 ((P \odot_F Q) \rhd_F (R \sqcup_F S))$\\[-5px]
  \item $(P \sqcup_F (Q \rhd_F R)) \leq_4 ((P \sqcup_F Q) \rhd_F R)$
  \end{enumerate}
\end{lemma}
The remaining filter properties proposed by Horne et
al.~\cite{horne2017semantics} actually fail in both directions.
\begin{lemma}
  \label{lemma:the_unfilterish_properties}
  There exists an $P$, $Q$, and $R$ that cause the following
  properties to not hold:
  \begin{enumerate}
  \item $(P \rhd_F (Q \odot_F R)) \leq_r (Q \sqcup_F (P \rhd_F R))$\\[-5px]
  \item $(P \rhd_F Q) \leq_4 (P \sqcup_F Q)$
  \end{enumerate}
\end{lemma}
Interestingly, if we change Definition~\ref{def:filterish-semantics}
so that all the basic properties hold and
Lemma~\ref{lemma:the_unfilterish_properties} holds, then the
inequalities in Lemma~\ref{lemma:the_filterish_properties} degenerate
to equalities.  We were unable to find a definition of the logical
connectives that make all of the properties in both of the previous
lemmas hold.

Just as we did for the ideal quaternary semantics we can show that we
can model attack trees as formulas.  The following defines their
interpretation.
\begin{definition}
  \label{def:interp-aterms-quaternary}
  Suppose $\mathbb{B}$ is some set of base attacks, and $\nu :
  \mathbb{B} \mto \mathsf{PVar}$ is an assignment of base attacks to
  propositional variables.  Then we define the interpretation of
  attack trees to propositions as follows:
  \begin{center}
    \begin{math}
      \setlength{\arraycolsep}{5px}
      \begin{array}{lll}
        \begin{array}{lll}
          \interp{[[b]] \in \mathbb{B}} & = & \nu([[b]])\\
          \interp{[[AND(A, B)]]} & = & \interp{[[A]]} \odot_F \interp{[[B]]}
        \end{array}
        &
        \begin{array}{lll}
          \interp{[[SEQ(A,B)]]} & = & \interp{[[A]]} \rhd_F \interp{[[B]]}\\
          \interp{[[OR(A,B)]]} & = & \interp{[[A]]} \sqcup_F \interp{[[B]]}
        \end{array}
      \end{array}
    \end{math}
  \end{center}
\end{definition}
We can use this semantics to prove equivalences between attack trees.
\begin{lemma}[Equivalence of Attack Trees in the Ideal Quaternary Semantics]
  \label{lemma:equivalence_of_attack_trees}
  Suppose $\mathbb{B}$ is some set of base attacks, and $\nu :
  \mathbb{B} \mto \mathsf{PVar}$ is an assignment of base attacks to
  propositional variables.  Then for any attack trees $[[A]]$ and
  $[[B]]$, if $[[A ~ B]]$, then $\interp{[[A]]} \equiv
  \interp{[[B]]}$.
\end{lemma}
\begin{proof}
  This proof holds by induction on the form of $[[A ~ B]]$.
\end{proof}
