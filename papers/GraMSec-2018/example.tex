The quaternary logics introduced in the previous section do indeed
capture all of the equivalences of attack trees, but they also support
proving specializations.  Consider the example attack trees in
Fig.~\ref{fig:2}.
\begin{figure}
  \begin{tabular}{|l|l|}
    \hline
    \begin{tabular}{l}
      \textbf{A.}\\
      \begin{minipage}{\textwidth/2}
        \vspace{3px}
        \begin{minted}[fontsize=\scriptsize]{haskell}
 and_node "obtain secret"
  (or_node "obtain encryped file"
     (base_na "bribe sysadmin")
     (base_na "steal backup"))
  (seq_node "obtain password"
     (base_na "break into system")
     (base_na "install keylogger"))
        \end{minted}
        \vspace{-3px}
      \end{minipage}      
    \end{tabular}
    &
    \begin{tabular}{l}
      \textbf{B.}\\
      \begin{minipage}{\textwidth/2-5.6px}
        \vspace{3px}
        \begin{minted}[fontsize=\scriptsize]{haskell}
 seq_node "break in, obtain secret"
  (base_na "break into system")
  (and_node "obtain secret inside"
    (base_na "install keylogger")
    (base_na "steal backup"))
        \end{minted}
        \vspace{20px}
      \end{minipage}      
    \end{tabular}    
    \\    
    \hline
  \end{tabular}\vspace{-2px}
  \begin{tabular}{|l|}
    \begin{tabular}{l}
      \\[-9.8px]
    \textbf{C.}\\
      \begin{minipage}{\textwidth}
        \begin{minted}[fontsize=\scriptsize]{haskell}
 or_node "obtain secret"
  (and_node "obtain secret via sysadmin"
   (base_na "bribe sysadmin")
   (seq_node "obtain password"
     (base_na "break into system")
     (base_na "install keylogger")))
   (seq_node "break in, obtain secrect"
     (base_na "break into system")
     (and_node "obtain secret inside"
       (base_na "install keylogger")
       (base_na "steal backup")))        
        \end{minted}
        \vspace{2px}
      \end{minipage}
    \end{tabular}
    \\
    \hline
  \end{tabular}
    \caption{Encrypted Data Attack from Figure~1 (A), Figure~3 (B), and Figure~2 (C) of Horne et al.~\cite{horne2017semantics}.}
    \label{fig:2}
\end{figure}
In the ideal semantics attack tree C is a sound specialization of
attack tree A, and attack tree B is a sound specialization of attack
tree A.  Attack tree C requires the attacker to break into the system
before they can steal the backup, but attack tree A does not require
this.  Then attack tree B has dropped bribing the sysadmin and simply
requires the attacker to just steal the backups.  Notice that none of
the attack trees in Fig.~\ref{fig:2} are equivalent.  So how do we
prove these specializations are sound?  We prove that they are related
through an entailment rather than an equivalence.
\begin{definition}
  \label{def:specialization}
  An attack tree $A$ is a sound specialization of an attack $B$ if and
  only if $\interp{A} \leq_4 \interp{B}$.
\end{definition}

We can now formally prove that the attack tree C is a specialization
of attack tree A, and that attack tree B is a specialization of attack
tree A from Fig.~\ref{fig:2}.
\begin{example}
  \label{ex:ex1}
  First, consider the following assignment:
\begin{center}
  \begin{math}
    \begin{array}{lllllll}
      a := \verb!"bribe sysadmin"!
      & \quad &
      b := \verb!"break into system"!
      \\
      c := \verb!"install keylogger"!
      & \quad &
      d := \verb!"steal backup"!
    \end{array}
  \end{math}
\end{center}
Then we have the following interpretations:
\begin{center}
  \footnotesize
  \begin{math}
    \begin{array}{c}
      \begin{array}{ccc}
        \begin{array}{rclcl}
        \interp{A} & = & \interp{[[AND(OR(a,dd),SEQ(b,c))]]} \\
                   & = & (a \sqcup_I d) \odot_I (b \rhd_I c)\\
      \end{array}
      & \quad &
      \begin{array}{rclcl}
        \interp{B} & = & \interp{[[SEQ(b,AND(c,dd))]]} \\
                   & = & b \rhd_I (c \odot_I d)\\      
      \end{array}
      \end{array}
      \\ \\
      \begin{array}{rclcl}
        \interp{C} & = & \interp{[[OR(AND(a,SEQ(b,c)),SEQ(b,AND(c,dd)))]]}\\
                   & = & (a \odot_I (b \rhd_I c)) \sqcup_I (b \rhd_I (c \odot_I d))\\
      \end{array}
    \end{array}        
  \end{math}
\end{center}
We reuse the same names for base attacks across the interpretations
above.  Finally, we have the following two entailments:
\begin{center}
  \begin{math}
    \footnotesize
    \begin{array}{|ll|l|}
      \hline
      \begin{array}{lll}
        \underline{\interp{C} \leq_4 \interp{A}}:\\[4px]
        \,\,\,\,\begin{array}{rl}
               & (a \odot_I (b \rhd_I c)) \sqcup_I (b \rhd_I (c \odot_I d))\\
        \leq_4 &  (a \odot_I (b \rhd_I c)) \sqcup_I (b \rhd_I (d \odot_I c))\\
        \leq_4 &  (a \odot_I (b \rhd_I c)) \sqcup_I (d \odot_I (b \rhd_I c))\\
        \leq_4 &  (a \sqcup_I d) \odot_I (b \rhd_I c)\\
        \end{array}
      \end{array}
      & \quad &
      \begin{array}{lll}
        \underline{\interp{B} \leq_I \interp{A}}:\\[4px]
        \,\,\,\,\,\begin{array}{lll}
               & b \rhd_I (c \odot_I d)\\
        \leq_4 &  b \rhd_I (c \odot_I (a \sqcup_I d))\\
        \leq_4 &  b \rhd_I ((a \sqcup_I d) \odot_I c)\\
        \leq_4 & (a \sqcup_I d) \odot_I (b \rhd_I c)\\
        \end{array}
      \end{array}\\
      \hline
    \end{array}
  \end{math}
\end{center}
Notice that neither $\interp{[[A]]} \leq_4 \interp{[[C]]}$ nor
$\interp{[[A]]} \leq_4 \interp{[[B]]}$ hold, and thus, equivalences
cannot prove the previous properties.
\end{example}
